\chapter{Algorithmen II}

Zusammenfassung der Vorlesung "`Algorithmen II"' aus dem Wintersemester 2014.\footnote{\url{http://geom.ivd.kit.edu/ws14_algo2.php}}


\section{Flussmaximierung}

\subsection{Flussnetzwerke}
\textbf{Ziel:} Berechnung des maximalen Flusses.

\subsubsection{Bestandteile}
\begin{itemize}
	\item Graph mit Quelle q und Senke s: \(G(V,E)\) mit \(E \subset V^2\)
	\item Kapazitätsfunktion \(k : V^2 \rightarrow \mathbb{R}_{+0}\), wobei \(\forall e \in V^2 \setminus E: k(e) = 0\)
\end{itemize}

\subsubsection{Flüsse}
Ein Fluss in F ist eine Funktion \(f:V^2\rightarrow\mathbb{R}\) mit den Eigenschaften:
\begin{enumerate}
	\item \(f \leq k\)
	\item \(\forall x,y \in V : f(x,y) = -f(y,x)\)
	\item \(\forall x \in V \setminus \{ q,s \}: 0 = \sum f(x,V) := \sum_{y\in V} f(x,y)\)
\end{enumerate}

\subsubsection{Bemerkungen}
\begin{itemize}
	\item Der negative Fluss soll lediglich die Darstellung vereinfachen und ist ansonsten uninteressant
	\item In alle Knoten außer in \(q\) und in \(s\) fließt immer so viel positiver Fluss rein wie raus
	\item Flussnetzwerke können zu einem Flussnetzwerk zusammengafasst werden (mit unendlichen Zu- und Abflüssen)
\end{itemize}


\subsection{Algorithmus von Ford und Fulkerson}
\begin{itemize}
	\item \textbf{Algorithmus}
	\begin{enumerate}
		\item Setze die Flüsse an alle Kanten gleich Null
		\item Solange es einen Pfad von \(q\) nach \(s\) gibt, erhöhe einen beliebigen Pfad
	\end{enumerate}
	\item Terminiert Ford-Fulkerson, ist \(f\) maximal
	\item Aufwand: \(\mathcal{O}(|E| \cdot max\{W_f | f~Fluss~in~F\})\)
\end{itemize}


\subsection{Algorithmus von Edmonds und Karp}
\begin{itemize}
	\item Idee: Anwendung von Ford-Fulkerson immer längs eines kürzesten Pfades (Breitensuche)
	\item Aufwand: \(\mathcal{O} (|E|^2 \cdot |V|)\)
\end{itemize}


\subsection{Präfluss-Pusch-Methode}
Jeder Knoten erhält zusätzlich eine Höhe und ein Reservoir, um vorübergehend beliebig viel Fluß speichern zu können.

\subsubsection{Operationen}

\paragraph{\(PUSCH(x,y)\)}
\begin{itemize}
	\item \textbf{Bedingungen}
	\begin{itemize}
		\item Überschuss bei \(x\) vorhanden: \(u(x) > 0, x \in V \setminus \{q,s\}\)
		\item \(x\) liegt höher als \(y\): \(h(x) - h(y) = 1\)
	\end{itemize}
	\item \textbf{Algorithmus}
	\begin{enumerate}
		\item Ermittle den Abfluss aus Überschuss und Restkapazität: \(min\{u(x), k_f(x,y)\}\)
		\item Erhöhe den Fluss auf der Kante \(x,y)\)
		\item Aktualisiere die Überschusswerte an \(x\) und \(y\)
	\end{enumerate}
\end{itemize}

\paragraph{\(LIFTE(x)\)}
\begin{itemize}
	\item \textbf{Bedingungen}
	\begin{itemize}
		\item Der Knoten ist weder \(q\) und \(s\)
		\item Überschuss bei \(x\) vorhanden
		\item \(x\) kann nur geliftet werden, wenn kein Pfad \(y \rightarrow x\) eine höhere Restkapazität aufweist: \(h(x) < h(y)\)
	\end{itemize}
	\item \textbf{Algorithmus}
	\begin{enumerate}
		\item Erhöhe \(x\): \(h(x) = 1 + min\{h(y)\}\)
	\end{enumerate}
\end{itemize}

\paragraph{PREFLUSS-PUSCH}
\begin{enumerate}
	\item \textbf{Initialisiere}
	\begin{enumerate}
		\item Alle Knoten außer \(q\) bekommen die Höhe \(0\) (\(h(q) = |V|)\)
		\item Alle Pfade von \(q\) bekommen die Kapazität nach \(k(x,y)\) zugewiesen, alle anderen Pfade \(0\)
	\end{enumerate}
	\item Solange es eine beliebige, erlaubte Operation gibt, führe diese aus
\end{enumerate}
