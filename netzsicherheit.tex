\chapter{Netzsicherheit: Architekturen und Protkolle}

Zusammenfassung der Vorlesung "`Netzsicherheit: Architekturen und Protokolle"' aus dem Sommersemester 2016.\footnote{\url{https://telematics.tm.kit.edu/ss2016_2928.php}}

\section{Einführung}
\begin{itemize}
	\item Smarte Welt - alles vernetzt. Vorteile beispielsweise: Bessere Integration erneuerbarer Energien, bessere Organisation des Verkehrsm "`assistiertes Leben"'
	\item Vernetzte Daten: Sensoren übermitteln die erfassten Daten an einen zentralen Datenspeicher im Internet
	\item Problem: Systeme können (beispielsweise über das Internet) angegriffen werden, Daten können von unbefugten Dritten mitgelesen werden
	\item Alternative: Vollständig verteiltes System. Probleme: Vertrauensbasis? Kontrolle? Nachvollziehbarkeit? Zuverlässigkeit?
	\item Alternative: Isoliertes System? Kein Zugriff von außen, daher theoretisch sicher. Praktisch existiert immer eine Verbindung nach außen, z.B. zum installieren von Updates
\end{itemize}


\subsection{Security vs. Safety}

\subsubsection{Begriffsdefinitionen}
\begin{itemize}
	\item (IT-)System: Gesamtheit von Komponenten, die zusammenwirken, um eine bestimmte Funktionalität zu erfüllen
	\item Komponente: Bestandteil eines Systems, das eine Teilfunktion dessen realisiert und über Schnittstellen mit anderen Komponenten kommuniziert
	\item Güter: Ressourcen die für mindestens einen Akteur einen (subjektiven) Wert besitzen
	\item Schutzziel: Anforderungen an eine Komponente oder ein System, um Güter vor Bedrohungen zu schützen
	\item Angreifermodell: Beschreibt die Fähigkeiten eines Angreifers, Angriffe auf ein System durchzuführen (beispielsweise Lokalität, Werkzeuge, kryptografische Fähigkeiten)
	\item \textbf{Safety}
	\begin{itemize}
		\item Zustand des Geschütztsein von schützenswerten Gütern vor bestimmten Gefahren
		\item Ist-Funktionalität von Komponenten stimmt mit der Soll-Funktionalität überein
	\end{itemize}
	\item \textbf{Security}
	\begin{itemize}
		\item Angriffssicherheit
		\item Bedrohung durch böswilligen Angreifer
		\item Beispielsweise Schutz der Integrität von Informationen
	\end{itemize}
\end{itemize}


\subsection{Schutzziele}
\begin{itemize}
	\item \textbf{Vertraulichkeit}
	\begin{itemize}
		\item Ein System bewahrt Vertraulichkeit, wenn es keine unautorisierte Informationsgewinnung ermöglicht
		\item Bausteine: Symmetrische oder asymmetrische Verschlüsselung
	\end{itemize}
	\item \textbf{Integrität}
	\begin{itemize}
		\item Starke Integrität: Es ist nicht möglich, Daten unautorisiert zu manipulieren
		\item Schwache Integrität: Es ist nicht möglich, Daten unautorisiert \textit{unbemerkt} zu manipulieren. Manipulation ist in vielen Fällen nicht verhinderbar, sollte dann aber nicht unbemerkt bleiben
		\item Bausteine: Tamper proof Module, Message Authentication Codes (MAC)
	\end{itemize}
	\item \textbf{Authentizität}
	\begin{itemize}
		\item Echtheit von Subjekten und/oder Daten
		\item Bausteine: Zertifikate, Signaturen, gemeinsames Geheimnis
	\end{itemize}
\end{itemize}


\subsection{Typische Angriffe}
\begin{itemize}
	\item \textbf{Angreifermodell}
	\begin{itemize}
		\item Idee: Klassifikation von Angreifern nach Ressourcen/Motivation/Fähigkeiten zur Bestimmung des Sicherheitsniveaus (Gegen welche Art von Angreifer will/kann ich mich schützen?)
		\item Dolev-Yao-Angreifer: Angreifer ist omnipräsent, kann Dateneinheiten erzeugen/versenden/modifizieren, kann allerdings nicht ver- oder entschlüsseln ohne den Schlüssel zu kennen (Angreifer einspricht "`Outsider"')
	\end{itemize}
	\item \textbf{Systematische Einordnung von Angriffen}
	\begin{itemize}
		\item Passiv: Unautorisierte Informationsgewinnung \(\rightarrow\) Vertraulichkeit
		\item Aktiv: Unautorisierte Manipulation \(\rightarrow\) Integrität/Verfügbarkeit
		\item Typische Angriffstechniken: Abhören/Zwischenschalten (beispielsweise MitM)/Manipulieren/Unterdrücken/Einfügen (beispielsweise DoS)/Replay
	\end{itemize}
\end{itemize}


\subsection{Schutzmechanismen und Bausteine}
\begin{itemize}
	\item \textbf{Kryptografische Bausteine}
	\begin{itemize}
		\item Symmetrische oder asymmetrische Verschlüsselung
		\item Integritätssicherung durch kryptografische Hashfunktion oder digitale Signatur
	\end{itemize}
	\item \textbf{Zertifikate}
	\begin{itemize}
		\item Authentifizierung eines Sachverhalts, den man nicht selbst überprüfen kann durch vertrauenswürdige Dritte (CA)
		\item Digitales Dokument, in dem eine Instanz einen bestimmten Sachverhalt mittels digitaler Signatur bestätigt
	\end{itemize}
	\item \textbf{Authentifizierung}
	\begin{itemize}
		\item Dient der Überprüfung, ob ein Kommunikationspartner tatsächlich derjenige ist, der er vorgibt zu sein
		\item Möglichkeiten: (Kombination aus) Besitz/Wissen/Biometrisches Merkmal
		\item Mechanismen und Bausteine
		\begin{itemize}
			\item Passwörter oder Passwort-Hashes: Authentifikation durch Nachweis eines Geheimnissen. Nachteile u.a.: Passwortliste notwendig (Ziel für Angreifer), Passwort muss übertragen werden (Vertraulichkeit eventuell gefährdet)m oft schlechte Wahl der Passwörter
			\item Challenge-Response-Authentifizierung: Vergleichbare Probleme wie bei Passwörtern
		\end{itemize}
	\end{itemize}
\end{itemize}



\section{Schlüsselaustausch}



\section{Vertrauensmodelle}



\section{Authentifizierung}



\section{Kerberos}



\section{Zugangsschutz}



\section{IPsec}



\section{TLS}



\section{Internetdienste}



\section{Privatsphäre}
