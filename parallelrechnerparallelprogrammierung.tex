\chapter{Parallelrechner und Parallelprogrammierung}

Zusammenfassung der Vorlesung "`Parallelrechner und Parallelprogrammierung"' aus dem Sommersemester 2017.\footnote{\url{https://www.scc.kit.edu/personen/11185.php}}

\section{Einführung}

\begin{itemize}
	\item \textbf{Klassifikation nach Flynn}
	\begin{description}
		\item[Single Instruction Single Data (SISD):] von-Neumann-Architektur (Einprozessorrechner)
		\item[Single Instruction Multiple Data (SIMD):] Vektorrechner
		\item[Multiple Instruction Single Data (MISD):] In der Praxis irrelevant. Ausnahme: Mehrere Geräte, die zur Berechnungsverifikation das selbe Datum mehrfach parallel berechnen
		\item[Multiple Instruction Multiple Data (MIMD):] Multiprozessorsystem
	\end{description}
	\item \textbf{Multiprozessorsysteme}
	\begin{description}
		\item[Speichergekoppelter:] Gemeinsamer Adresseraum; Kommunikation über gemeinsame Variablen; skalieren mit \(>1000\) Prozessoren
		\begin{description}
			\item[Uniform Memory Access Model (UMA):] Alle Prozessoren greifen gleichermaßen mit gleicher Zugriffszeit auf einen gemeinsamen Speicher zu (symmetrische Multiprozessoren)
			\item[Non-uniform Memory Access Modell (NUMA):] Speicherzugriffszeiten variieren, da Speicher physikalisch auf verschiedene Prozessoren verteilt ist (Distributed Shared Memory System (DSM))
		\end{description}
		\item[Nachrichtengekoppelt:] Lokale Adresseräume; Kommunikation über Nachrichten (No-remote Memory Access Model); theoretisch unbegrenzte Skalierung
		\begin{description}
			\item[Uniform Communication Architecture Model (UCA):] Einheitliche Nachrichtenübertragungszeit
			\item[Nin-uniform Communication Architecture Model (NUCA):] Unterschiedliche Nachrichtenübertragungszeiten in Abhängigkeit der beteiligten Prozessoren
		\end{description}
	\end{description}
\end{itemize}