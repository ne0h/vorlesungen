\chapter{Mikroprozessoren II}

Zusammenfassung der Vorlesung "`Mikroprozessoren II"' aus dem Wintersemester 2015.\footnote{\url{https://capp.itec.kit.edu/teaching/mp2/?lang=d&sem=ws15}}

\section{Einführung}

Entwurf einer Rechneranlage: Ingenieurmäßige Aufgabe der Kompromissfindung zwischen:
\begin{itemize}
	\item Zielsetzung: Einsatzgebiet, Anwendungsbereich, Leistung, Verfügbarkeit, etc.
	\item Randbedingungen: Technologie, Größe, Geld, Energieverbrauch Umwelt, etc.
	\item Gestaltungsgrundsätze: Modularität, Sparsamkeit, Fehlertoleranz, etc.
	\item Anforderungen: Kompatibilität, Betriebssystemanforderungen, Standards, etc.
\end{itemize}


\subsection{Entwurfsfragen: Zielsetzungen}

\subsubsection{Einsatzgebiete}
\begin{itemize}
	\item \textbf{Desktop Computing}
	\begin{itemize}
		\item PCs bis Workstations (\$1000 - \$10.000)
		\item Günstiges Preis-/Leistungsverhältnis
		\item Ausgewogene Rechenleistung für ein breites Spektrum von (interaktiven) Anwendungen
	\end{itemize}
	\item \textbf{Server}
	\begin{itemize}
		\item Rechen- und datenintensive Anwendungen
		\item Hohe Anforderungen an die Verfügbarkeit und Zuverlässigkeit
		\item Skalierbarkeit
		\item Große Dateisysteme und Ein-/Ausgabesysteme
	\end{itemize}
	\item \textbf{Eingebettete Systeme}
	\begin{itemize}
		\item Mikroprozessorsysteme, eingebettet in Geräte und daher nicht unbedingt sichtbar
		\item Sind auf spezielle Aufgaben zugeschnitten (hohe Leistungsfähigkeit, Spezialprozessoren)
		\item Breites Preis-/Leistungsspektrum
		\item Echtzeitanforderungen
		\item Abwägung der Anforderungen an Rechenleistung, Speicherbedarf, Kosten, Energieverbrauch, etc.
	\end{itemize}
\end{itemize}

\subsubsection{Anwendungsbereiche}
\begin{itemize}
	\item Technisch-wissenschaftlicher Bereich: Hohe Anforderungen an die Rechenleistung, insbesondere Gleitkommaverarbeitung
	\item Kommerzieller Bereich: Datenbanken, WEB, Suchmaschinen, Optimierung von Geschäftsprozessen, etc.
	\item Eingebettete Systeme: Verarbeitung digitaler Medien, Automatisierung, Telekommunikation, etc.
\end{itemize}

\subsubsection{Rechenleistung}
\begin{itemize}
	\item Ermittlung übr Benchmarks
	\item Maßzahlen für die Operationsleistung: \textit{MIPS} oder \textit{MFLOPS}
	\item \(MFLOPS = \frac{Anzahl~ausgefuehrter~Gleitkommainstruktionen}{10^6 \cdot Ausfuerhungszeit}\)
\end{itemize}

\subsubsection{Zuverlässigkeit}
\begin{itemize}
	\item Bei Ausfällen von Komponenten muss ein betriebsfähiger Kern bereit sein
	\item Verwendung redundanter Komponenten
	\item Bewertung der Ausfallwahrscheinlichkeit mittels stochastischer Verfahren
	\item Definition Verfügbarkeit: Wahrscheinlichkeit, ein System zu einem beliebigen Zeitpunkt fehlerfrei anzutreffen
\end{itemize}

\subsubsection{Energieverbrauch, Leistungsaufname}
\begin{itemize}
	\item \textbf{Mobile Geräte}
	\begin{itemize}
		\item Verfügbare Energiemenge durch Batterien und Akkumulatoren ist begrenzt \(\rightarrow\) möglichst lange mit der vorhandenen Energie auskommen
		\item Vermeiden von Überhitzungen
	\end{itemize}
	\item Green IT: Niedriger Energieverbrauch, ökologische Produktion, einfaches Recycling
\end{itemize}


\subsection{Entwicklung der Rechnertechnik}


\subsection{Entwicklung der Mikroprozessortechnik}



\section{Parallelismus auf Maschinenbefehlsebene}

\subsection{Superskalartechnik}


\subsection{Very Long Instruction Word (VLIW)}


\subsection{Explicitly Parallel Instruction Computing (EPIC)}



\section{Parallelismus auf Thread-Ebene}



\section{Multicore/Manycore}



\section{Systemstrukturen}
