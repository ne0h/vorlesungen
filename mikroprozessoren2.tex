\chapter{Mikroprozessoren II}

Zusammenfassung der Vorlesung "`Mikroprozessoren II"' aus dem Wintersemester 2015.\footnote{\url{https://capp.itec.kit.edu/teaching/mp2/?lang=d&sem=ws15}}

\section{Einführung}

Entwurf einer Rechneranlage: Ingenieurmäßige Aufgabe der Kompromissfindung zwischen:
\begin{itemize}
	\item Zielsetzung: Einsatzgebiet, Anwendungsbereich, Leistung, Verfügbarkeit, etc.
	\item Randbedingungen: Technologie, Größe, Geld, Energieverbrauch Umwelt, etc.
	\item Gestaltungsgrundsätze: Modularität, Sparsamkeit, Fehlertoleranz, etc.
	\item Anforderungen: Kompatibilität, Betriebssystemanforderungen, Standards, etc.
\end{itemize}


\subsection{Entwurfsfragen: Zielsetzungen}

\subsubsection{Einsatzgebiete}
\begin{itemize}
	\item \textbf{Desktop Computing}
	\begin{itemize}
		\item PCs bis Workstations (\$1000 - \$10.000)
		\item Günstiges Preis-/Leistungsverhältnis
		\item Ausgewogene Rechenleistung für ein breites Spektrum von (interaktiven) Anwendungen
	\end{itemize}
	\item \textbf{Server}
	\begin{itemize}
		\item Rechen- und datenintensive Anwendungen
		\item Hohe Anforderungen an die Verfügbarkeit und Zuverlässigkeit
		\item Skalierbarkeit
		\item Große Dateisysteme und Ein-/Ausgabesysteme
	\end{itemize}
	\item \textbf{Eingebettete Systeme}
	\begin{itemize}
		\item Mikroprozessorsysteme, eingebettet in Geräte und daher nicht unbedingt sichtbar
		\item Sind auf spezielle Aufgaben zugeschnitten (hohe Leistungsfähigkeit, Spezialprozessoren)
		\item Breites Preis-/Leistungsspektrum
		\item Echtzeitanforderungen
		\item Abwägung der Anforderungen an Rechenleistung, Speicherbedarf, Kosten, Energieverbrauch, etc.
	\end{itemize}
\end{itemize}

\subsubsection{Anwendungsbereiche}
\begin{itemize}
	\item Technisch-wissenschaftlicher Bereich: Hohe Anforderungen an die Rechenleistung, insbesondere Gleitkommaverarbeitung
	\item Kommerzieller Bereich: Datenbanken, WEB, Suchmaschinen, Optimierung von Geschäftsprozessen, etc.
	\item Eingebettete Systeme: Verarbeitung digitaler Medien, Automatisierung, Telekommunikation, etc.
\end{itemize}

\subsubsection{Rechenleistung}
\begin{itemize}
	\item Ermittlung übr Benchmarks
	\item Maßzahlen für die Operationsleistung: \textit{MIPS} oder \textit{MFLOPS}
	\item \(MFLOPS = \frac{Anzahl~ausgefuehrter~Gleitkommainstruktionen}{10^6 \cdot Ausfuerhungszeit}\)
\end{itemize}

\subsubsection{Zuverlässigkeit}
\begin{itemize}
	\item Bei Ausfällen von Komponenten muss ein betriebsfähiger Kern bereit sein
	\item Verwendung redundanter Komponenten
	\item Bewertung der Ausfallwahrscheinlichkeit mittels stochastischer Verfahren
	\item Definition Verfügbarkeit: Wahrscheinlichkeit, ein System zu einem beliebigen Zeitpunkt fehlerfrei anzutreffen
\end{itemize}

\subsubsection{Energieverbrauch, Leistungsaufname}
\begin{itemize}
	\item \textbf{Mobile Geräte}
	\begin{itemize}
		\item Verfügbare Energiemenge durch Batterien und Akkumulatoren ist begrenzt \(\rightarrow\) möglichst lange mit der vorhandenen Energie auskommen
		\item Vermeiden von Überhitzungen
	\end{itemize}
	\item Green IT: Niedriger Energieverbrauch, ökologische Produktion, einfaches Recycling
\end{itemize}

\subsubsection{Trends in der Rechnerarchitektur: Herausforderungen}
Weltweite Forschungsaktivitäten bzgl. ExaScale-Rechner
\begin{itemize}
	\item Verlustleistung: Überträgt man heutige (Stand 2010) Höchstleistungsrechner in den Exascale-Bereich, hätte man eine Verlustleistung von etwa 40 GW (diese kann allerdings höchstens 20-40 MW betragen)
	\item Hauptspeicher (DRAM), permanenter Speicher: Kapazität und Zugriffsgeschwindigkeit muss mit der Rechengeschwindigkeit mithalten
	\item Zuverlässigkeit und Verfügbarkeit
	\item Parallelität und Lokalität
\end{itemize}


\subsection{Entwicklung der Rechnertechnik}

\subsubsection{Halbleitertechnologie}
\begin{itemize}
	\item Mikrominiaturisierung setzt sich fort. Verkleinerung der Strukturbreiten sowie Erhöhung der Integrationsdichte: Anzahl der Transistoren verdoppelt sich alle 18 Monate)
	\item \textbf{Technologische Entwicklung bei Intel Prozessoren}
	\begin{itemize}
		\item Neuer Herstellungsprozess alle zwei Jahre mit Verdopplung der Transistorenanzahl
		\item Strukturgröße reduziert sich jedes Jahr um 30\% oder halbiwert sich alle 5 Jahre
		\item 1 Mrd. Transistoren in 2018 \(\rightarrow\) 100 Mrd. in 2021
	\end{itemize}
\end{itemize}

\subsubsection{Forschungsansätze}
\begin{itemize}
	\item Erforschung zukünftiger Fertigungstechnologien auf der Grundlage von Kohlenstoff, Nanotechnologie
	\item Beispiele: Single Molecule Diode, Single Electron Transistor, Carbon Nano Tube
\end{itemize}


\subsection{Entwicklung der Mikroprozessortechnik}

\subsubsection{Taktrate}
\begin{itemize}
	\item Bis 2000 ist die Taktrate exponentiell gestiegen
	\item Steigerung der Prozessorleistung seither durch Verbesserungen des Herstellungsprozesses, tieferen Pipelines und verbesserten Schaltkreistechnologien
\end{itemize}

\subsubsection{Steigerung der Rechenleistung durch Parallelverarbeitung}
\begin{itemize}
	\item Integration vieler Prozessorkerne auf einem Chip (Multicore/Manycore)
	\item Integration hierarchischer Speicher-/Cache-Strukturen
	\item Neue Verbindungsstrukturen (beispielsweise NoCs)
	\item Adaptive Strukturen
\end{itemize}

\subsubsection{Aufbau eines Rechners mit Multicore}
\begin{itemize}
	\item \textbf{Aufbau eines Rechners mit Multicore}
	\begin{itemize}
		\item Mehrere Prozessorkerne mit separaten Steuerwerk und Rechenwerk, teilweise auch eigener Cache (L1 und L2)
		\item Gemeinsamer Shared Cache (L3)
		\item Northbridge zur Anbindung schneller Geräte (PCIe, RAM) und Sothbridge für die restlichen Geräte (IDE, SATA, PCI, SMB, HD-Audio, etc.)
		\item Struktur: \texttt{CPU<-----Front Side Bus----->Northbridge<-----Direct Media Interface----->Southbridge}
	\end{itemize}
	\item \textbf{Speicher-/Cache-Strukturen}
	\begin{itemize}
		\item Zugriffsgeschwindigkeit der Hauptspeicherkomponenten (DRAMs) wächst nicht mit der Prozessorgeschwindigkeit: Lücke zwischen Zugriffsgeschwindkeit und Prozessorgeschwindigkeit (Memory Wall) \(\rightarrow\) Lösung: Speicherhierarchie
		\item Zuwachs Prozessorgeschwindkeit pro Jahr um 50\% gegenüber Steigerung der Zugriffsgeschwindigkeit um 7\% pro Jahr
	\end{itemize}
	\item \textbf{Verbinungsstrukturen}
	\begin{itemize}
		\item Hierarchische Mehrbusstrukturen
		\begin{itemize}
			\item Verbinden Komponenten auf verschiedenen Ebenen
			\item On-Chip Verbindungsnetzwerke: Leiten Werte zwischen den Pipelinestufen weiter und verbinden Prozessorkerne
			\item Systemverbindungsstrukturen: Verbinden Prozessoren (CMPs) mit Speicher und I/O
			\item Peripheriebusse: Verbinden I/O-Schnittstellenbausteine mit dem Systembus
			\item System-Verbindungsnetzwerke: SANs (sehr kurze Entfernungen), LANs (in Organisationen und Gebäude) und WANs (weite Entfernungen)
		\end{itemize}
		\item Punkt-zu-Punkt-Verbindungen: Quick-Path-Interconnect (QPI)
		\begin{itemize}
			\item Von Intel entwickelte Struktur zur Kommunikation zwischen Prozessoren untereinander und für die Kommunikation zwischen Prozessoren und Chipsatz
			\item Direkte Verbindungen können zwischen jedem Prozessorpaar eingerichtet werden
			\item Anbindung von PCIe und dediziertem Speicherbus
		\end{itemize}
	\end{itemize}
\end{itemize}


\section{Parallelismus auf Maschinenbefehlsebene}

\subsection{Superskalartechnik}


\subsection{Very Long Instruction Word (VLIW)}


\subsection{Explicitly Parallel Instruction Computing (EPIC)}



\section{Parallelismus auf Thread-Ebene}



\section{Multicore/Manycore}



\section{Systemstrukturen}
