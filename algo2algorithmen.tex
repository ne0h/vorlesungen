\section{Appendix A: Algorithmen und Erklärungen}

Aufstellung und Erläuterung aller Algorithmen der Vorlesung Vorlesung "`Algorithmen II"' aus dem Wintersemester 2014.\footnote{\url{http://geom.ivd.kit.edu/ws14_algo2.php}}

\subsection{Algorithmus von Ford und Fulkerson}

\begin{algorithm}[H]
	\caption{Ford-Fulkerson}
	\SetKwData{Left}{left}\SetKwData{This}{this}\SetKwData{Up}{up}
	\SetKwFunction{Union}{Union}\SetKwFunction{FindCompress}{FindCompress}
	\SetKwInOut{Input}{input}\SetKwInOut{Output}{output}

	\Input{$F(G,k,q,s)$}
	\Output{Ein maximaler Fluß $f$}
	\BlankLine

	$f \longleftarrow 0$

	\While{Es gibt einen Pfad $q \rightarrow s$ in $G_f$} {
		Erhöhe $f$ über diesem maximal
	}
\end{algorithm}

\subsection{Edmonds-Karp.Algorithmus}
	Erhöht man den Fluß in Ford-Fulkerson imer längs eines kürzesten Pfads (Breitensuche), erhält man den Edmonds-Karp-Algorithmus
	
\subsection{Die Präfluss-Pusch-Methode}
\begin{algorithm}[H]
	\caption{Push}
	\SetKwData{Left}{left}\SetKwData{This}{this}\SetKwData{Up}{up}
	\SetKwFunction{Union}{Union}\SetKwFunction{FindCompress}{FindCompress}
	\SetKwInOut{Input}{input}\SetKwInOut{Output}{output}

	\Input{$x,y$}
	\Output{}
	\BlankLine

	$d \longleftarrow \min{ü(x), k_f(x,y)}$ \newline
	$f(x,y) \longleftarrow f(x,y) +d$ \newline
	$ü(x) \longleftarrow ü(x) -d $ \newline
	$ü(y) \longleftarrow ü(y) +d$ \newline
\end{algorithm}

Push ist nur erlaubt wenn 
	\begin{itemize}
		\item $ü(x) > 0, x \in V \setminus {q,s}$
		\item $(x,y) \in E_f$
		\item $h(x) - h(y) = 1$
	\end{itemize} 


\begin{algorithm}[H]
	\caption{Lifte}
	\SetKwData{Left}{left}\SetKwData{This}{this}\SetKwData{Up}{up}
	\SetKwFunction{Union}{Union}\SetKwFunction{FindCompress}{FindCompress}
	\SetKwInOut{Input}{input}\SetKwInOut{Output}{output}

	\Input{$x$}
	\Output{}
	\BlankLine

	$h(x) := 1+ \min_{(x,y) \in E_f}{h(y}$
\end{algorithm}

Lifte ist nur erlaubt wenn 
	\begin{itemize}
		\item $x \in V \setminus {q,s}$
		\item $ü(x) > 0$
		\item $h(x) \leqslant \min_{(x,y) \in E_f}{h(y)}$
	\end{itemize} 


\begin{algorithm}[H]
	\caption{Präfluss-Pusch}
	\SetKwData{Left}{left}\SetKwData{This}{this}\SetKwData{Up}{up}
	\SetKwFunction{Union}{Union}\SetKwFunction{FindCompress}{FindCompress}
	\SetKwInOut{Input}{input}\SetKwInOut{Output}{output}

	\Input{$F(G,k,q,s$}
	\Output{Maximaler Fluss $f$}
	\BlankLine

	\For{alle $ x,y \in V $ }{
	$ h(x)  \leftarrow \begin{cases}|V| & x=q \\ 0 & \text{sonst}\end{cases}$ \newline
	$ f(x,y)  \leftarrow \begin{cases}k(x,y) & x=q \\ 0 & \text{sonst}\end{cases}$}
	\While{Es gibt erlaubte Push oder Lifte Operationen} {Führe beliebig Push oder Lifte aus}
\end{algorithm}

\subsection{"`An die Spitze"' Präfluss-Push-Algorithmus}

\begin{algorithm}[H]
	\caption{Leere}
	\SetKwData{Left}{left}\SetKwData{This}{this}\SetKwData{Up}{up}
	\SetKwFunction{Union}{Union}\SetKwFunction{FindCompress}{FindCompress}
	\SetKwInOut{Input}{input}\SetKwInOut{Output}{output}

	\Input{$x$}
	\Output{}
	\BlankLine

	\While{$ü(x)>0$}{	
		\If{$i_x > 0$}{
			$y \longleftarrow n_x(i_x)$ \newline
			\If{$(x,y) \text{puschbar}$}{
				$Pusch(x,y)$}
			\Else{
				$i_x \longleftarrow i_x -1$}
			}
		\Else{	$Lifte(x)$ \newline
				$i_x \longleftarrow Grad(x)$}
			
		}	
\end{algorithm}

\begin{algorithm}[H]
	\caption{An die Spitze}
	\SetKwData{Left}{left}\SetKwData{This}{this}\SetKwData{Up}{up}
	\SetKwFunction{Union}{Union}\SetKwFunction{FindCompress}{FindCompress}
	\SetKwInOut{Input}{input}\SetKwInOut{Output}{output}

	\Input{$F(G,k,q,s)$}
	\Output{Einen maximalen Fluss f}
	\BlankLine
	
	Initialisiere f und h wie in Präfluss-Pusch \newline
	Generie L \newline
	$x \longleftarrow Kopf(L)$ \newline
	\For{$\forall x \in V$}{
		$i_x \longleftarrow Grad(x)$}
	\While{$x \ne Nil$ }{	
		$h_{alt} \longleftarrow h(x)$ \newline
		$Leere(x)$ \newline
		\If{$h_{alt} < h(x)$}{
			Setze x an die Spitze von L
		}	
		$x \longleftarrow \text{Nachfolger von x in L}$
		}
\end{algorithm}

\subsection{Paaren in allgemeinen Graphen}
\begin{algorithm}[H]
	\caption{Paare}
	\SetKwData{Left}{left}\SetKwData{This}{this}\SetKwData{Up}{up}
	\SetKwFunction{Union}{Union}\SetKwFunction{FindCompress}{FindCompress}
	\SetKwInOut{Input}{input}\SetKwInOut{Output}{output}

	\Input{bipartiter Graph $(V_1 \stackrel{\cdot}{\cup} V_2, E)$}
	\Output{Maximale Paarung P}
	\BlankLine
	
	$V \longleftarrow V_1 \cup V_2 \cup \{ q,s \}$ \newline
	$\hat{E} \longleftarrow \{ q \} \times V_1 \cup E \cup V_2 \times \{ s \} $ \newline
	$k \longleftarrow \begin{cases}1 & e \in \hat{E} \\ 0 & \text{sonst}\end{cases}$ \newline
	$f \longleftarrow Ford-Fulkerson( F(V, \hat{E} ),h,q,s)) $ \newline
	$P \longleftarrow \{ e \in E | f(e) = 1 \} $
\end{algorithm}


\subsection{Sortieren durch stochastisches Teilen}
\begin{algorithm}[H]
	\caption{Quicksort}
	\SetKwData{Left}{left}\SetKwData{This}{this}\SetKwData{Up}{up}
	\SetKwFunction{Union}{Union}\SetKwFunction{FindCompress}{FindCompress}
	\SetKwInOut{Input}{input}\SetKwInOut{Output}{output}

	\Input{$ S : = \{ s_1, ... , s_n \} $ 	mit paarweise verschiedenen 	$ s_i \in \mathbb{Z} $ 	}
	\Output{$( \delta_1, ... , \delta_n)$ mit $\delta_1 < ... < \delta_n$ und $\{ \delta_1, ... , \delta_n \} = S $}
	\BlankLine
	
	\While{ S $\ne \emptyset$} {
		Wähle ein zufälliges Pivotelement $ y \in S $ \newline
		Zerlege $S \setminus \{ y \} $ 	in 	$  s_1 $ und $ s_2 $ , so dass 	$ s_1 < y < s_2 $ \newline
		Gib $ ( Quicksort(s_1 , y ),Quicksort(s_2))$ aus \newline
		}
\end{algorithm}


\subsection{Binäre Zerlegung des Raum}
Rekursive Zerlegung eines Polyeders des \(\mathbb{R}^3\) anhand einiger Polygone. Pro Iterationsschritt wird jeweils ein Polygon bestimmt (bevorzugt eines, das den Polyeder komplett zerlegt). Dieses teilt den Polyeder in einen linken und einen rechten Teilpolyeder, die jeweils weiter zerlegt werden.

\begin{algorithm}[H]
	\caption{BRZ}
	\SetKwData{Left}{left}\SetKwData{This}{this}\SetKwData{Up}{up}
	\SetKwFunction{Union}{Union}\SetKwFunction{FindCompress}{FindCompress}
	\SetKwInOut{Input}{input}\SetKwInOut{Output}{output}

	\Input{Ein Polyeder $P \subset \mathbb{R}^3$; orientierte, planare, disjunkte Polygone $P_1,...,P_n \subset \mathbb{R}^3$}
	\Output{Ein $RBZ$ für $P_1,...,P_n$}
	\BlankLine

	$k \longleftarrow 1$
	\BlankLine
	\tcc{Sortiere Polygonenteile aus, die außerhalb von $P$ liegen (Clipping)}
	\For {$i=1,...,n$}{
		$Q_k \longleftarrow P_1 \cap P$
		\If {$Q_k \neq \emptyset$} {
			$k \longleftarrow k + 1$
		}
	}

	$l \longleftarrow 1$

	\BlankLine
	\tcc{Falls ein Polygon $P$ komplett zerteilt, nehme dieses als Trennelement. Anderenfalls nehme das erste in der Liste}
	\If{$\exists~j: Q_j~zerlegt~P~vollstaendig$}{
		$l \longleftarrow j$
	}

	$Wurzel \longleftarrow Q_l$
	\BlankLine
	\tcc{Teile $P$}
	\If{$k \geq 3$}{
		$Q \longleftarrow P \cap li.~HR~von~Q_l$\newline
		$li. Wurzelteilbaum \longleftarrow BRZ(Q,Q_1,...,Q_{k-1})$\newline
		$Q \longleftarrow P \cap re.~HR~von~Q_l$\newline
		$re. Wurzelteilbaum \longleftarrow BRZ(Q,Q_1,...,Q_{k-1})$
	}

	\BlankLine
	\Return{Wurzel mit ihren Teilbaeumen}

\end{algorithm}


\subsection{Konstruktion konvexer Hüllen}
Die konvexe Hülle einer Teilmenge ist die kleinste konvexe Menge, die die Ausgangsmenge enthält.\footnote{\url{http://de.wikipedia.org/wiki/Konvexe_Hülle}}

\subsubsection{Annahmen}
\begin{itemize}
	\item Die Reiehenfolge der \(p_i\) sei gleichverteilt zufällig
	\item Keine vier der Ebenen \(p_i^{*}\) schneiden sich in einem Punkt
	\item Jeder Knoten hat den Grad \(3\)
\end{itemize}

\begin{algorithm}[H]
	\caption{BRZ}
	\SetKwData{Left}{left}\SetKwData{This}{this}\SetKwData{Up}{up}
	\SetKwFunction{Union}{Union}\SetKwFunction{FindCompress}{FindCompress}
	\SetKwInOut{Input}{input}\SetKwInOut{Output}{output}

	\Input{Punktmenge $P=(p_1,..,p_n) \cap A^3$}
	\Output{$\lbrack P \rbrack$}
	\BlankLine

	\tcc{Verschiebe $P$, so dass der Ursprung innerhalb der Konvexen Hülle liegt}
	$v \longleftarrow 0 - \sum_i \frac{p_i}{n}$\newline
	$P \longleftarrow P + v$
	\BlankLine

	\tcc{Schritt 2}
	$V \longleftarrow$ Knotenliste von $p_1^{\leq} \cap ... \cap p_4^{\leq}$\newline
	\For{$j > i$} {
		\eIf{$p_j^{\leq} \supset V$}{
			Entferne $p_j$ aus $P$\newline
			Aktualisiere "`gedanklich"' alle Indizes $k>j$
		}{
			Verknüpfe $p_j$ bidirektional mit einem $w_j \in V_4 \backslash p_j^{\leq}$
		}
	}
	\BlankLine

	\tcc{Schritt 3}
	\For{i = 5,...,n}{
		Durchlaufe $V$ von $w_i$ aus und setze dabei:\newline
			$N_i \longleftarrow \{ Schnittpunkte~der~Kanten~mit~p_i^{*}\}$\newline
			$W_i \longleftarrow V \cap p_i^{>}$\newline
			$V \longleftarrow (V \cup N_i) \backslash W_i$\newline
		\BlankLine
		\For{$j>i$ mit $w_j \in W_i$}{
			\eIf{$N_i \subset p_j^{\leq}$}{
				Entferne $P_j$ aus $P$\newline
				Aktualisiere "`gedanklich"' alle Indizes
			}{
				Verknüpfe $p_j$ neu mit einem $w_j \in N_i \cap p_j^{>}$
			}
		}
	}
	\BlankLine

	\Return{Verschobene Polarmenge $Q_P$}

\end{algorithm}


\subsection{Pledge Startegie}

Findet einen Weg aus einem Labyrinth.\\
Der Roboter kann erkennen, wenn er das Labyrinth verlassen hat.\\
Bei jeder Drehung wird die Änderung zum Startwinkel $ \varphi $ aktualisiert.\\

\begin{algorithm}[H]
	\caption{Pledge Strategie}
	\SetKwData{Left}{left}\SetKwData{This}{this}\SetKwData{Up}{up}
	\SetKwFunction{Union}{Union}\SetKwFunction{FindCompress}{FindCompress}
	\SetKwInOut{Input}{input}\SetKwInOut{Output}{output}

	\Input{Labyrinth $L$ und Roboter $R$}
	\Output{Weg aus dem Labyrinth}
	\BlankLine

	$\varphi \leftarrow 0$
	
	\While{$ R \in L $}{	
		gehe vorwärts bis zu einer Wand.\\
		gehe links der Wand entlang bis $ R \not\in L \text{ oder } \varphi = 0 $	
		}
\end{algorithm}


\subsection{Wanze}

Findet einen Ziel in einer Umgebung mit Hindernissen.\\

\begin{algorithm}[H]
	\caption{Wanze}
	\SetKwData{Left}{left}\SetKwData{This}{this}\SetKwData{Up}{up}
	\SetKwFunction{Union}{Union}\SetKwFunction{FindCompress}{FindCompress}
	\SetKwInOut{Input}{input}\SetKwInOut{Output}{output}

	\Input{$P_1 , ... , P_n$ disjunkte, einfache geschlossene Polygone.\\
	$ S, Z $ Start und Ziel $\in A^2 \setminus P$.\\
	Roboter $R$ der seine eigene Position $r$ kennt.}
	\Output{Weg von $S$ nach $Z$.}
	\BlankLine
	
	\While{$r \ne Z$}{	
		gehe Richtung $Z$ bis $r = Z$ oder $r \in P$\\	
		\If{$r \ne Z$} {
                    umlaufe $P_i$, suche dabei $q \in argmin || x - z ||_2 , x \in P_i$ \\
                    (die Stelle auf dem umlaufenen Polygon mit minimalem Abstand zu $Z$.)\\
                    gehe zu $q$
		}
    }
\end{algorithm}

\subsection{Geometrische Algorithmen - Türsuche 1}

Roboter steht vor einer langen Wand und sucht die Tür.\\


\begin{algorithm}[H]
	\caption{Türsuche 1}
	\SetKwData{Left}{left}\SetKwData{This}{this}\SetKwData{Up}{up}
	\SetKwFunction{Union}{Union}\SetKwFunction{FindCompress}{FindCompress}
	\SetKwInOut{Input}{input}\SetKwInOut{Output}{output}


	$i \leftarrow 1$
	
	\While{Tür noch nicht gefunden}{	
		gehe $i$ Meter der Wand entlang.\\
		gehe i Meter zurück (ändert die Laufrichtung)\\
		$ i \leftarrow i+1$\\
		}
\end{algorithm}

Weglänge: Ist die Tür n + (0,1) Meter vom Ausgangspunkt entfernt, gilt:\\
Weglänge $\ge 2 \cdot 1 + 2 \cdot 2 + ... + 2n +n = O(n^2)$.\\
$\Rightarrow$ Türsuche 1 ist nicht kompetetiv.


\subsection{Geometrische Algorithmen - Türsuche 2}

Roboter steht vor einer langen Wand und sucht die Tür.\\


\begin{algorithm}[H]
	\caption{Türsuche 2}
	\SetKwData{Left}{left}\SetKwData{This}{this}\SetKwData{Up}{up}
	\SetKwFunction{Union}{Union}\SetKwFunction{FindCompress}{FindCompress}
	\SetKwInOut{Input}{input}\SetKwInOut{Output}{output}


	$i \leftarrow 1$
	
	\While{Tür noch nicht gefunden}{	
		gehe $i$ Meter der Wand entlang.\\
		gehe i Meter zurück (ändert die Laufrichtung)\\
		$ i \leftarrow 2i$\\
		}
\end{algorithm}

Weglänge: Ist die Tür $2^{n+\delta}$ Meter vom Ausgangspunkt entfernt, gilt:\\
Weglänge $\le 2 \sum\limits_{i=0}^{n+1} 2^i + 2^{n+\delta} \le 2 ^{n+3+\delta} + 2^{n+\delta} \le 9 \cdot 2^{n+\delta}$.\\
$\Rightarrow$ Türsuche 2 ist 9-kompetetiv.
