\section{Allgemeines}
	\subsection{Welche Themen wurden in der Vorlesung behandelt?}
		\begin{itemize}
			\item Klassifikation von Geräten
			\item Privatsphäre im IoE
			\item Betrachtung des IoE in Bezug auf das OSI-Modell
			\item Sicherheitsaspekte
		\end{itemize}
	
	\subsection{Auf welche Probleme stößt man im IoE? Warum müssen diese extra behandelt werden?}
		\begin{itemize}
			\item Möglichste Energiesparend, da Batteriebetrieben
			\item Sehr beschränkte Rechenleistung
			\item Kommunikationswege nicht zuverlässig, Semi-Broadcast
			\item Meist keine Infrastruktur vorhanden, wenig zentrale Infrastruktur
			\item Geringe Bauform und muss günstig sein
			\item Greif tief in die Privatsphäre des Menschen ein, Sammelt private Daten.
			\item Omnipräsenz?
		\end{itemize}
		
	\subsection{Welche Unterschiede und Besonderheiten gibt es?}
		\begin{itemize}
			\item Dezentral
			\item Selbstorganisieren
			\item Limitierte Ressourcen
			\item Unzuverlässiger Ressourcenkanal
			\item Unsicher, knoten können zerstört oder ausgelesen werden.
		\end{itemize}
		
	\subsubsection{Was muss man deswegen besonders beachten?}
		Durch die Allgegenwärtigkeit der Knoten muss besonders auf die Privatsphäre geachtet werden.
		
	\subsubsection{Privatsphäre was ist das?}
	Die Möglichkeit vertraulich zu kommunizieren und selbst darüber zu bestimmen, wer welche Daten über mich erhält.
	
	\subsubsection{Warum ist Privatsphäre wichtig? Wo ist das Problem beim Datenschutz?}
	sensitive Daten?
	
	\subsubsection{Was sind die Prinzipien des Datenschutzes?}
	Datensicherheit, Datensparsamkeit, Rechtmässigkeit, Transparenz, Nutzerrechte, Kontrolle.
		
	\subsubsection{Was gibt es für Schutzziele?}
	
	\subsubsection{Wie realisieren wir Sicherheit im IoE?}
	
	\subsubsection{Wir hatten eine Wetterstation als Beispiel, wie funktioniert das?}
	Wetterstation sammelt Daten und überträgt diese an die Server des Herstellers. Kunde greift mit seinem Smartphone auf die gesammelten Daten auf den Servern des Herstellers zu. Die Daten liegen nicht beim Kunden sondern beim Hersteller in fremder Hand.
	\paragraph{Wo liegt hier das Problem?}
		Es wird unter anderem die Lautsärke gemessen, die lässt eventuell sogar Sprachaufzeichnungen zu. Desweiteren können über diese Daten auf das Verhalten der Bewohner schließen lassen.
	\subsubsection{Wo liegt das Problem bei Smart Metering?}
	Aus dem hochfrequenten sampling der Verbrauchswerte lassen sich ebenso Verhaltensmuster schließen.
	\paragraph{Warum will man das dann?}
		Durch die hochfrequenten Messwerte lassen sich Geräte schalten wenn Stromüberschüsse vorhanden sind außerdem kann der Netzbetreiber mit diesen Informationen sein Stromnetz besser regeln und weiß wann er mehr oder weniger erzeugen muss.
	\paragraph{Womit lässt sich dieses Problem umgehen?}
		SMART-ER
	\subsection{Was ist das Problem bei vielen Geräten auf kleinem Raum, und was macht man dagegen?}
	Das Problem sind Kollisionen und der damit verbundene Energieverbrauch durch Erkennung und Retransmission. Das Routing wird komplexer. Die Gegenmaßnahme ist Topologiekontrolle.
	
	\subsection{Was hat bei Sensorknoten den höchsten Energieverbrauch?}
	Funkschnittstelle und, wenn vorhanden, Display.
	
	\subsection{Welche Netztopologien kommen im IoE vor?}
	\begin{itemize}
		\item Einzellnes Gateway
		\item Mehrere Gateways (Redundanz)
		\item Ad\- Hoc\- Netz
	\end{itemize}
		
\section{Routing}
	\subsection{Welche Verfahren gibt es?}
		Probalistische, lokalisationsbasierte und inhaltsbasierte Verfahren.
		
	\subsection{Welches inhaltsbasierte Verfahren hatten wir?}
		Direct Diffusion
	
	\subsubsection{Wie funktioniert Direct Diffusion?}		
		Sender broadcastet Interesse nach bestimmten Daten in Form von Attribut-Wert Paaren. System speichert Richtung aus der ein Intresse kam in Form von Gradienten. Sensoren mit entsprechenden Daten schicken die Daten an den Sender. Sobal der Sender merkt, dass die Daten verfügbar sind, startet er eine Reinforcment Phase zur Verstärkung der Gradienten.
		\subsubsection{Genaures im Detail } %TODO
		

\section{MAC}
	\subsection{Welche Verfahren hatten wir auf der MAC-Schicht?}
		S-MAC, B-MAC und 802.15.4 
		
		\subsubsection{Wie lassen sich diese Klassifizieren?}
		Zentral(802.15.4) Dezentral(S-Mac), Synchron(S-Mac) Asynchron(B-Mac) %TODO
		
	\subsection{Wie funktioniert S-MAC?}
	%TODO	
	
	\subsubsection{Welche Erweiterungen gab es bei S-MAC und wie funktionieren diese?}
		\begin{itemize}
			\item ALP:
			\item MP (Message Passing):
		\end{itemize}
		
	\subsection{Wie funktioniert B-MAC?}
	
	\subsection{Wie funktioniert 802.15.4 im Beacon Modus?}
	Duty Cycling, gibt das Verhältnis zwischen wach und schlafphasen an.
	\subsubsection{Wie erhält man im Beacon Modus einen garantierten Zeitschlitz?}
	%TODO Zeit Paket Diagram aus der Vorlesung zeichen
	
	


\section{Sicherheit}
	\subsection{Warum wollen wir vertraulich kommunizieren}
	
	\subsection{Reicht Verschlüsselung um die Privatsphäre zu schützen}
	Nein, Beispiel SMART\- METER
	\subsection{Was haben wir besprochen?}
	Single Mission Key, Zufallsverteilte Schlüssellisten und Key Infection
	
	\subsection{Wie funktioniert Single Mission Key?}
	Ein Schlüssel für alle Systeme, alle nutzen diesen für Verschlüsselung und Integritätssicherung. Problem ist, dass ein kompromitiertes Gerät reicht umd die gesamte Sicherheit zu brechen.
	\subsubsection{passiert wenn ein neues System hinzukommt?}
	Es wird mit dem selben Single Mission Key konfiguriert und kann dann mit dem Rest des Netzes kommunizieren.
	
	\subsection{Was ist EGLI?}
	Eschenauer Gligor
	\subsubsection{Wei funktioniert EGLI?}
	Es gibt eine globale Schlüsselliste und lokale zufällige Teilmengen als Keyrings.
	
	%TODO Digagramm des Schlüsselaustauschs
	A sendet Zufallszahl, B antwortet mit verschlüsselter Zahl mit allen Schlüsseln, damit kennt A nun die Schlüssel(durch Entschlüsseln mit eigenen Schlüsseln) 
	
	Keypool P wird vom Benutzer zufällig erzeugt. Jedes System erhält einen Keyring R mit einer Teilmenge von P. Es wird gezeigt, dass mit guter Wahl von P und R eine Wahrscheinlichkeit von mehr als 99\% erreicht werden kann, dass zwei Systeme einen gemeinsamen Schlüssel in ihren Keyring haben. 
	\paragraph{Ist EGLI zuverlässig?}
	Nein, da es keine E2E Quittungen gibt.
	\subsubsection{Wie Kann man den Aufwand minimieren?}
	Wert selbst mitschicken und damit nur mit richtigem entschlüsseltem Antworten.
	
	\subsubsection{Was ist Zuverlässigkeit?}
	Vollständige Übertragung in der korrekten Reihenfolge ohne Dupplikate.
	\paragraph{Wie stellen wir Zuverlässigkeit sicher?}
		Time, Quittungen, Sequenznummern, FEC, E2E Quittungen, Hop by Hop Quittungen.
	\subsubsection{Was ist der Unterschied zwischen E2E und Hop 2 Hop Kommunikation?}
	
	\subsection{Was ist Privatsphäre?}
	Right to be let alone, hat sich zu Recht auf Bestimmung über Verwendung persönlicher Daten entwickelt.
	\subsubsection{Beispiele für Geräte die persönliche Daten aufzeichnen?}
	Smart Watch, SmartPhone, Thermometer, Smart Meter...
	\subsubsection{Was ist bei Stromzählern problematisch? Die werden doch heutzutage auch schon abgelesen?}
	Die zeitliche Auflösung ist das Problem, bisher gab es einen Verbrauchswert pro Jahr, jeder sind Werte im Bereich von weniger als einer Sekunde möglich. Dies lässt Rückschlüsse auf das private Leben schließen.
	\subsubsection{Warum benötigt man Smart\- Meter?}
	Dezentrale Steuerung des Stromnetzes v.a. bei volatilen regenerativen Energiequellen (dezentral Energiequellen oder Verbraucher passend ein\-  und ausschalten, z.B. schaltet sich die Waschmaschine erst bei lokalem Stromüberschuss an.)
	\subsubsection{Wie kann man Schutz der Privatsphäre bei Smart\- Metern absichern?} 
	\begin{itemize}
		\item Grundmaßnamen sind immer Pseudoanonymisierung und Verschleierung (zeitliche und örtliche Auflösung)
		\item Pseudoanonymisierung ist jedoch problematisch da hier eine Trusted Third Party notwendig ist
		\item Zeitliche Verschleierung verhindert jedoch die effiziente Nutzung der Daten
		\item Besser ist hier eine örtliche Verschleierung durch Gruppenbildung (SMART\- ER)
	\paragraph{Wie funktioniert SMART\- ER?}
	%TODO Diagramm einfügen
	Gruppenbildung und Zufallszahlen austauschen
	
	Summe und Menge der Kommunikationspartner zurückgeben( weil Gültigkeit der Werte davon abhängt, ob Kommunikationspartner gültige Werte geliefert haben)
	
	Das Problem ist die Gruppeneinteilung durch Anbieter oder Trusted Third Party, so könnte ein Anbieter einfach jeden Zähler in eine eigene Gruppe zuweisen und das ganze Verfahren wäre sinnlos. Alternativen sind Smart Meter Speed\- dating(aber: Woher bekommen die Knoten vertrauenswürdig die gloable Liste aller Teilnehmer?) oder dezentrales Aggegieren in Overlay\- Netzen.
	\end{itemize}
	
	\subsection{Was ist das besondere im Bezug auf traditionelle Sicherheit?}
	\begin{itemize}
		\item Hauptproblem: Ressourcenknappheit, d.h. asymmetrische Verfahren sind problematisch, Schlüsselaustausch muss mit symmetrischen Verfahren implementiert werden.
		\item Oft gibt es keine zentrale Infrastruktur
		\item Schlüsselaustauschverfahren für klassische Sensornetzen: EGLI, Key Infection: DICE als schlankeres DTLS
	\end{itemize}
	
	\subsection{Wie funktioniert Key Infection?}
	%TODO Diagramm des Schlüsselaustausches zeichnen
	A sendet $k_i$, B sendet $Enc_{k_i}(k)$ zurück, dies ist nun der Sitzungsschlüssel.
	Nur wer in direkter Reichweite von A war, konnte $k_i$ abhören und kennt damit den Sitzungsschlüssel, damit muss der Angreifer sehr viele räumlich verteilte Systeme korrumpieren.
\section{noch nicht zugeordnet}
	\subsection{Wie kriegt man ein Sensornetz ans Internet angebunden?}
		Mit 6LoWPAN, angepasste Version von IPv6 für das IoT. Bildet Adaptionsschicht zwischen 802.15.4 und IPv6.
		\subsubsection{Was ist 6LoWPAN?}
		IPv6 für schwache Knoten, z.B. Sensornetzen. Ermöglicht Anbindung an das "normale" Internet. Bietet bspw. Header Compression. Ein Gateway setzt dann von 6LoWPAN in normales IPv6 um.
		\subsubsection{Wie funktioniert 6LoWPAN?}
		komprimierung/dekomprimierung am Edge-Router
		\subsubsection{Welche Arten von Adressierung gibt es in IPv6?}
		Unicast, Multicast und Anycast
		\subsubsection{Wie sieht eine 6LoWPAN Datenpaket aus?}
		Dispatch\- Byte (Header\- Typ/Fragmentierung), danach Paketheader
		Paketheader kann z.B. komprimierter IP\-  bzw. UDP\- Header sein, in diesem Fall enthält er ein Bitfeld, welches angibt welche Teile des Headers komprimiert werden.
		\subsubsection{Was ist das Dispatch/Compress Byte?}
		Das Dispatch Byte ist ein verpflichtendes "Headerfeld", das angibt ob es sich um einen komprimierten/unkomprimierten IPv6 Header handelt oder ob das Paket fragmentiert ist. Danach kommt nochmal ein Byte das anzeigt welches Feld des IP Heads komprimiert ist.
		\subsubsection{Welche Header Felder werden komprimiert und wie?}
		Direkt aus PAN bzw. MAC\- Adresse abgeleitet, konkateniert mit Netzpräfix, d.h. Präfix kann bei lokalen Adressen weggelassen werden, bei direkter Kommunikation kann die Adresse sogar ganz weg gelassen werden, das sie bereits im MAC\- Header steht.
		Alle Felder die bereits duch MAC Felder abgedeckt werden und ansonsten redundante Informationen darstellen würden.
		Verkehrsklasse/Flow Label, Version, Addressen, Länge...
		Versionsfeld wird komprimiert, da immer 6.
		
		Traffic-Control und Flow Label sind meist 0
		
		Adresse bei Link-Lokaler Kommunikation,  volle Adressen nur, wenn vorher ein Kontext etabliert wurde)
		
		Next Header
		
		Länge aus Schicht 2 
		
		\paragraph{Wie sieht ein komprimierter Header aus?}
		Mehrere Bits die angeben, ob Adresse, Länger, NextHeader etc... komprimiert oder unkomprimiert vorliegen, gefolgt von den entsprechenden Feldern.
		
		\subsubsection{In welche Richtung setzt das Gateway die komprimierten Header um?}
		Von IPv6 zu 6LoWPAN: komprimierung
		Von 6LowPan zu IPv6: dekomprimierung
		
		\subsubsection{Gibt es noch andere Header die komprimiert werden können?}
		Ja, UDP. Hierbei werden weniger Ports benutzt und die Länge auf Schicht 3 berechnet. Die Sequenznummer kann komprimiert werden, das UDP nicht zuverlässig ist.
		\paragraph{Wofür braucht man überhaupt Ports?} Bindung von Programmen an die Netzwerkadresse.
		\paragraph{Wieso benötigt UDP hier weniger Ports?} UDP ist kein zuverlässiges Protokoll und belegt deswegen keinen Port um auf eine Bestätigung des Empfangs zu hören.	Ausserdem kann die Anzahl der verfügbaren Ports reduziert werden da so ein ressourcenarmes Gerät vermutlich keine $2^{16}$ Anwendungen ausführen wird.	
		
		
		
		
		
		\subsubsection{Wie lang ist eine IPv6\- Adresse?}
		128 Bit. Die ersten 64Bit sind der Präfix, welche bei link-lokaler Kommunikation nicht benötigt wird und die zweiten 64 Bit sind die Interface ID, z.B. kann hier die MAC-Adresse des Adapters verwendet werden.
		\subsubsection{Wie lange ist ein IPv6\- Header?}
		40 Byte
		\subsubsection{Wie groß ist ein 6LoWPAN Paket?}
		\subsubsection{Wie groß ist ein 802.15.4 Frame?}
		
		\subsubsection{Wie funktioniert das Routing bei 6LoWPAN?}
		6LowPan nutzt RPL
		\subsubsection{Wie funktioniert Address Autoconfig?}
		\subsubsection{Wie funktioniert Fragmentierung?}
	
	\subsection{Welche Kommunikationsformen gibt es in WPANs?}
		Unicast, Multicast und Concast
		\subsubsection{Was ist Unicast?}
			\paragraph{Was ist HHR?}
			\paragraph{Was ist E2ER?}
		\subsubsection{Was ist Multicast?}
		\paragraph{Gibt es bei Multicast ein spezielles Verfahren bezüglich der Zuverlässigkeit, wenn ja welches und wie funktioniert es?}
			Pump Slowly Fetch Quickly: %TODO
			\subparagraph{Wie funktioniert das?}
			Szenario: Sender sendet, System leitet nach Verzögerung weiter
			\subparagraph{Leiten sie immer weiter?}
			Nein, wenn mindestens 3 andere bereits weitergeleitet haben, dann nicht
			\subparagraph{Und ansonten immer?}
			Neun, nur wenn das System alle Sequenznummern erhalten hat. Fehlet eine Sequenznummer, so wird diese erstmal mit einem NACK an alle Nachbarn angefragt. Nachbarn warten zufällige Zeit pro Sequenznummer und verschicken das Paket nur nochmal, falls nicht schon ein anderes System auf das NACK reagiert hat.
			\subparagraph{Wann und wie werden NACKs verbreitet?} 1Hop Umgebung (semi-bc)
			\subparagraph{Wann werden Daten gepumpt?} Probalistisches Multicast bis zu gewissen grenzen, kein Fluten
			\subparagraph{Wenn ein Knoten anhand der Sequenznummer feststellt, dass ihm ein Paket fehlt, muss er dann seine Nachbarn kennen um das NACK zu senden?}
			Nein, das NACK wird per Broadcast mit TTL=1 gesendet.
			\subparagraph{Haben wir damit jetzt 100\% zuverlässigkeit?}
			Nein, kommt keine Dateneinheit mit falscher Sequenznummer, so sieht auch kein Knoten die Notwendigkeit ein NACK zu senden.
			\subparagraph{Wie addressiert der Sender die Empfänger?}
			Wenn alle erreicht werden sollen, wird per Broadcast transmittet, soll nur eine Teilgruppe der Teilnehmer erreicht werden, so wird per Multicast addressiert %TODO ähm?
			\subparagraph{Kann ich bei einem Broadcast die Adresse einfach weglassen?}
			Nein man benutzt eine Broadcastadresse.
			
		\paragraph{Was ist ESRT und wie funktioniert es?}
		Senke broadcastet gewünschte Rate über starke Sendeleistung, da es meistens am Stromnetz hängt.
			
		
		\subsubsection{Was ist Concast?}
		Viele Sender senden zu einem Empfänger(Senke)
		\paragraph{Welches Protokoll haben wir hier erwähnt?}
			ESRT 
			\subparagraph{Was ist ESRT?}
			
			\subparagraph{Wie funktioniert ESRT?}
			
			\subparagraph{Wie wird die neue Senderate publiziert?}
			Broadcast von der Senke an alle Quellen. Dafür muss sie aber auch alle mit ihrer Senderstärke erreichen können.
			
		\subsubsection{Was ist ein Semi\- Broadcast Medium?}
		Funk, nur eine Teilmenge aller Empfänger liegt in der Sendereichweite
		
	\subsection{Wie funktioniert LEACH?}
	Zeit wird in Runden diskreditiert, in jeder Runde werden zufällig $N \times P$ Cluster\- Heads bestimmt.
	Es gibt jeweils $\frac{1}{P}$ Runden zusammengefasst, innerhalb dieser Runden wächst die Wahrscheinlichkeit, dass sich ein Knoten als Cluster\- Head definiert, bis auf 1 an. 
	
	Nach der Wahl der Cluster\- Heads: Andere Knoten wählen ein Cluster
	
	%TODO ich hab keine Ahnung
	
	\subsubsection{Wie wird in LEACH Energie gespart?}
	Nur die Cluster\- Heads müssen jederzeit mit Gateway o.ä. kommunizieren (hohe Sendeleistung).
	Die Cluster\- Heads stellen einen Zeitplan auf und die anderen Knoten kommunizieren immer nur mit den Cluster\- Heads (Sterntopologie) und müssen nur wach sein, wenn es der Zeitplan vorsieht.
	
	\subsection{Was ist CoAP}
	CoAP ist eine leichtgewichtige Variante von HTTP(kompaktes binäres Protokoll, dass auf UDP aufbaut).
	Anfragetypen, zwei Schichten, Pakettypen.
	
	\subsection{Mac-Protokolle? Was anstatt WLAN?}
	IEEE 802.15.4, B-MAC, S-MAC
	
	
\section{Geräteklassen}
	\subsection{Was sind Nanonetze?}
	Kleinste vernetzte Nanomaschinen die jeweils nur eine Aufgabe wie Speichern, Messen, Berechnen oder Manipulieren ausführen können. Nanonetze benutzen Molekulare Kommunikation anstatt elektromagnetische.
	
	\subsection{Was ist Smart Dust?}
	Sehr kleine beschänkte Hardware im Nano bis Millimeter Bereich. Die in kooperation zusammen wirken. Bisher nur Forschungsthematik. 
	
	\subsection{Was sind klassische Sensornetze?}
	Sehr viele kleine auf Einzelanwendungen maßgeschneiderte Systeme. Sie sind batteriebetrieben und haben nur geringe Hardwareressourcen.
	
	\subsection{Was ist Physical \& Embedded Computing}
	Flexible Systeme aus Standardhardware, welche kostengünstig ist und eine hohe Energieeffizient aufweist. Im gegensatz zu klassischen Sensornetzen können diese Geräte auch einen ständigen Stromanschluss haben.
	
	\subsection{Was ist Smart\- und Submetering}
	Die zeitnahe Erfassung (und Steuerung) von Energieverbäuchen (~15min). Soll die effizientere Nutzung von Ressourcen ermöglichen. Das deutsche Modell (BSI) sieht viele Zähler und einzellne Gateways vor. 
	
	\subsection{Was ist Smart Home?}
	Hausautomation und monitoring durch Sensornetze. 
	Z.B. zur Rolladen oder Licht Steuerung.
	Dabei kommen Probleme wie einfachheit vs. Sicherheit auf. 
	\subsubsection{Wie sieht es mit Smart Home und Privacy aus?}
	Die Sensoren sind tief in den Lebensraum verflochten und können höchst private Daten sammeln.
	\subsection{Wie sieht es mit Robotern und mobilen Plattformen aus?}
	Hier handelt es sich um eine mobile Plattform mit Sensoren udn Aktoren zur Messung vor Ort.
	Sie könenn in lebensfeindlichen oder unzugänglichen Umgebungen eingesetzt werden.
	\subsection{Was sind Wearables?}
	Am Körper Tragbare Sensoren die das persönliche Verhalten, Aktivitätsdaten und Vitalwerte erfassen.
	Die Auswertung und Anzeige ist auf diesen Geräten nur rudimentär möglich.
	\subsection{Was sind Smart Phones?}
	Sie stellen die heutige Schnittstelle zwischen Mensch und IoE da, sie erlauben die Steuerung von Remote-Geräten.
	Sie bestehen aus leistungsfähiger Hardware und setzen die Existenz von leistungsfähiger Kommunikationsinfrastruktur voraus.
	\subsection{Was sind Single\- Board Computer?}
	Kostengünstige Hardware für die Entwicklungs und Prototypen Umgebung.
	Sie haben eine große Anzahl an verschiedener Bus Systeme und Anschlussmöglichkeiten.
	
	\subsection{Was ist Industrie 4.0 (Industrial Internet)?}
	Eine heterogene Umgebung an unterschiedlichsten Sensoren. Beinhaltet Leistungsfähige Steuer\- und Regelsysteme.
	Drahtlose Sensorik über Batterie oder Energy Harvesting.
	Anforderungen: Zuverlässigkeit, Robustheit, Langlebigkeit.
	
\section{Privatsphäre}
	\subsection{Was ist RFID und wo liegen die Herausforderungen für den Datenschutz?}
	RFID steht für Radio Frequency Identification. Drahtloste Übertragung von Informationen zwischen einem Transponder und einer Basistation.
	Das Problem ist das über diese Transponder eine eindutige Identifikation eines Objekts/Produkts oder einer Person möglich ist. Dieser Transponder kann auch aus großer Entfernung ausgelesen werden.
	\paragraph{Gibt es Lösungsansätze?}
	Ein möglicher Lösungsansatz wäre einen RFID Tag in einem Produkt an der Kasse duch einen "Kill" Befehl zu zerstören.
	Desweiteren könnten Kryptografische Mechanismen zur Authentifizierung genutzt werden.
	Physikalisch könnte auch einfach die Antenne abgerissen werden.
	
	\subsection{Wie sieht es mit dem Schutz der Privatsphäre aus?}
	Schutz des Persönlichkeitsrechts bei der Verarbeitung personenbezogener Daten 
	\subsubsection{Welche Daten sind schützenswert?}
	Laut BDSG sind alle Personenbezogenen Daten schützenswert allerdings sind im Kontext des Internet of Everything auch alle Metadaten schützenswert. Das heißt wer mit wem wann kommuniziert etc..
	
	\subsection{Was sind die Säulen zum Schutz der Privatsphäre?}
	\begin{itemize}
		\item Regulierung: Bundesdatenschutzgesetz etc..
		\item Selbstreuglierung: z.B. Gütesigel die den vertraulichen Umgang mit Daten bescheinigen.
		\item Selbstschutz: Privat Enhancing Technologies z.B. TOR
	\end{itemize}
	
	\subsection{Was sind allgemeine Schutzziele?}
	Anforderungen die Erfüllt werden müssen um schützenswürdige Güter vor Bedrohung zu schützen
	\begin{itemize}
		\item Confidentiality(Vertraulichkeit): Ermöglicht keine unautorisierte Informationsgewinnung
		\item Integrity(Integrität)
		\begin{itemize}
			\item Starke Integrität: Es ist unmöglich Daten unautorisiert zu verändern.
			\item Schwache Integrität: Es ist unmöglich Daten zu verändern ohne das es bemerkt wird.			
		\end{itemize}
		\item Avaliability(Verfügbarkeit): System beleibt Verfügbar und gewährt keine Einschränkung durch unautorisierten Zugriff.
	\end{itemize}

	\subsection{Spezielle Schutzziele für die Privatsphäre?}
	\begin{itemize}
		\item Unverkettbarkeit: Daten aus unterschiedlichen Kontexten sind nicht miteinander in Bezug zu setzen, z.B. durch Datenvermeidung oder Anonymisierung.
		\item Transparenz: Die Verarbeitung von Daten ist nachvollziehbar und überprüfbar. 
		\item Intervenierbarkeit: betroffene Personen können über die Erfassung und Verarbeitung ihrer Daten selbst bestimmen.
		
\end{itemize}		
	
	\subsection{Was ist ein Vertrauensmodell?}
	Vertrauen: Bewertung wie sich eine Entität in einem konkreten Sacherverhalt verhalten wird, hier die Annahme, dass Entität kein Angreifer ist.
	\begin{itemize}
		\item Vollständiges Vertrauen: uneingeschränktes Vertrauen aller Entitäten des Systems
		\item Keinerlei Vertrauen: Alle Entitäten potentielle Angreifer, PET notwendig
		\item Vertrauen in zentrale Instanz:Trusted third party
		\item Verteiltes Vertrauen: Nutzer vertraut lediglich, dass eine Teilmenge der beteiligten Entitäten nicht kooperiert.
		
	\end{itemize}
	\subsubsection{Was sind die Ziele des Angreifers?}
	\begin{itemize}
		\item Abhören von Daten
		\item Modifizieren von Daten
		\item Maskerade und Erzeugen von Daten
	\end{itemize}
	\subsubsection{Was ist das "klassiche" Angeifermodell}
	Dolec\- Yao, Outsider. Er kann alles mithören, kann Dateneinheiten erzeugen und versenden und fremde Dateneinheiten modifizieren. Er kann jedoch nicht Enschlüsseln oder Verschlüsseln ohne den Schlüssel zu kennen.
	
	\subsection{Ansätze zum Schutz der Privatspähre im IoE?}
	Diensterbringung mit dem Prinzip der Datensparsamkeit, durch Samples mit ausschlieslich benötigten Daten möglichst Anonym.
	Anonymisierung durch herabsetzen der Präzesion auf ein Minimum, hinzufügen von Störwerten.
	Zentrale Datensenken sind zu vermeiden. 
	Identität der Quelle verschleidern, z.B. durch Pseudonyme
	Unverkettbarkeit von Samples gewährleisten
	
	\subsection{Was ist bei der Privatsphäre anders im IoE?}
	Die Technologie greift viel stärker in das private leben ein und erfasst dort Daten, dadurch ist die Privatsphäre stärker gefährdet als im klassischen Internet.
	
	\subsection{Was sind IoE-spezifische Heausforderungen für die Privatsphäre?}
	\begin{itemize}
		\item Heterogene Geräte: nicht genügen Ressourcen für klassische kryptografie.
		\item Häufig keine zentrale Infrastruktur nutzbar, kein Public Key, kein zentraler Vertrauensanker.
		\item unklares Vertrauensmodell: Gegen wen schützen? wer ist mein Vertrauensanker?
		\item Kein klassisches Angreifermodell
		\item Es werden viel mehr Daten erfasst, zusammenführung beim Dienstanbieter?
		\item Kontinuierliche Datenerfassung
		\item Sensible Daten, aus dem persönlichen Umfeld/Wohnraum
		\item Veilfalt von gemessenen Phänomenen, die Verkettbarkeit erlaubt detailliere Profilbildung.
	\end{itemize}
	\subsubsection{Wo ist das Angreifermodell anders?}
	Im IoE sind die Geräte meist öffentlich zugänglich wodurch ein Angreifer physisch Zugang zu dem Gerät erhalten kann. Er kann den Speicher auslesen oder die Programmierung verändern.
	Hier wird von einem Insider Angriff ausgegangen.
	
	\subsection{Wie sieht es mit der Privatheit bei Smart-Metering aus?}
	Das periodische Senden von Messwerten lieftert detailiertes Verbrauchsprofil und damit Einblicke in die Privatsphäre.
	
	\subsubsection{Was kann man dagegen tun?}
	Der einfachste Ansatz wäre Messdaten mittels "falscher" Identitäten oder Pseudonymen zu übertragen jedoch ist dies sehr aufwendig, so ist bei einem neuen Kunden eine komplette Neuvergabe der Pseudonyme notwendig da dieser sonst direkt zugeordnet werden könnte. Desweiteren wird eine Trusted-Third-Party zur Vergabe der Pseudonyme gebraucht, welche nicht vertrauenswürdiger sein muss als das Energieunternehmen.
	Außerdem kann das Profil des Pseudonyms über externe Daten zugeordnet werden, z.B. über Kenntnis von Arbeitszeit etc..
	
	Eine weitere Möglichkeit wäre es den Energiebedarf nach aussen über einen Energiespeicher zu maskieren. Dies funktioniert aber auch nur solange der Akku genügend Kapazität besitzt, zumal dieses Verfahren sehr kostspielig ist.

\subsubsection{Gibt es einen inteligenten Ansatz?}
	Ja, das Ziel von Smart\- Metering ist das aggregieren von Daten, daher wie hoch ist der Energieverbrauch in "Karlsruhe\- Mitte" oder von allen Kunden der EnBW. Von daher die Idee, die Daten zu Aggregieren bevor sie übermittelt werden. Dies kann z.B. über viele Haushalte oder einen langen Zeitraum geschehen. 
	Über einen langen Zeitraum ist einfach jedoch nicht Ziel von Smart\- Metering über viele Haushalte hinweg ist jedoch schwierig zu realisieren.
\subsubsection{Hatten wir dazu ein Verfahren?}
Ja wir hatten SMART\- ER, Smart Meterin Protocoll with Exactness and Robustness.
Das Grundkonzept ist die Einteilung in Gruppen welche durch den Messdienstleister durchgeführt wird. 
Innerhalb der Gruppen wird kooperiert so das die Messwerte maskiert aggregiert werden können.
Pro Messintervall
\begin{itemize}
	\item Austausch von Zufallswerten innerhalb der Gruppe 
	
	Hierzu werden die Ausgehenden Zufallszahlen von M abgezogen und die eingehenden auf M addiert. 
	\item Speichern der Kommunikationspartner (Abhängigkeiten)
	
	
	L beinhaltet alle Kommunikationspartner.
	\item Berechnen maskierter Messwerte
	
	Der Messwert wird maskiert indem er auf M addiert wird.
	\item Senden maskierter Messwerte und Abhängigkeiten an Messdienstleister
	
	L und M werden an den Dienstleister übermittelt.
	\item Eventuelle Bereinigung empfangene Messwerte durch Messdienstleister.
\end{itemize}

Hier bleibt das Problem bestehen das der Dienstleister die Gruppenbildung vornimmt. Sind nun alle bis auf den Eigenen Zähler einer Gruppe korrumpiert so kann genau auf die Werte des eigenen Zählers zurück geschlossen werden.

	\paragraph{Was kann man gegen das Problem bei der Gruppenbildung unternehmen?}
	Man muss die Gruppenbildung aus der Verantwortung des Messdienstleisters entziehen. Hier gibt es zwei Ansätze:	
	\begin{itemize}
		\item Smart Meter Speed Dating: Die Zähler bilden ihre Gruppen selbständig und dezentral. 
		\item Elderberry: Baumbasierter Ansatz mit strukturiertem P2P-Overlay. Dezentrale Aggregation.
	\end{itemize}
	
\section{Recht}
	\subsection{Was sind die Funktionen des Datenschutzrechts?}
	Das Datenschutzrecht dient dem Schutz der inneren und äußeren Freiheit der Persönlichkeitsentfaltung gegen Beeinträchtigung und Gefährdung bei der Verarbeitung personenbezogener Daten
	
	\subsection{Was wird vom Datenschutzrecht abgedeckt?}
	Der Umgang mit personenbezogenen Daten, Einzelangaben über persönliche oder sachliche Verhältnisse eines bestimmten oder bestimmbaren Betroffenen. 
	
	\subsection{Was ist vom Datenschutzrecht ausgenommen?}
	Datenverarbeitung für ausschließlich persönliche und familiäre Zwecke
	
	\subsection{Wie wird ein Datenverarbeitungsprozess aufgespaltet?}
	\begin{itemize}
		\item Datenerhebung
		\item Datenverarbeitung: Speicherung, Veränderung, Übermittlung...
		\item Datennutzung
	\end{itemize}
	
	\subsection{Wer ist Verantwortlich?}
	Datenschutzrechtlich Verantwortlich ist die Stelle, die über Zwecke und Mittel einer Datenumgangshandlung entscheidet.
	
	\subsection{Dürfen einfach so personenbezogene Daten erhoben werden?}
	Nein, der Erhebung muss zugestimmt werden
	
	\subsection{Dürfen die erhobenen Daten beliebig genutzt werden?}
	Bei der Zustimmung der Erhebung werden die Daten an einen bestimmten Zweck gebunden und dürfen nur für diesen benutzt werden.
	
\section{Kommunikation}

	\subsubsection{Was sind Cyber Physical Systems?}
	Cyber Physical Systems sind vernetzte Komponenten mit Sensoren und Aktoren die physikalische Prozesse steuern. So z.B. eine Heizungssteuerung oder die Steuerung einer Fabrikanlage.
	Sie werden als Bestandteil des Internet of Everything gesehen und haben teilweise hohe Anforderungen an Safety, somit z.B. geringe Latenzen oder setzten die richtige reihenfolge von Daten vorraus.
	
	\subsection{Medienzugriff}

	\subsubsection{Welche Probleme haben wir mit Funk als Medium?}
	Funk ist sehr unzuverlässig, es kommt zu hohen Fehlerraten. Da es sich um ein geteiltes Medium handelt muss der Zugriff darauf geregelt werden.
	
	\subsubsection{Welche Anfoderungen stellen wir an die drahtlose Kommunikation?}
	\begin{itemize}
		\item Hohe Lebenszeit der vernetzten Ding: niedriger Energieverbrauch
		\item Robusheit gegenüber Topologieveränderung, z.B. "schlafende" Systeme
		\item Skaliebarkeit
		\item Selbstorganisation
		\item Sicherheit und Schutz der Privatsphäre
\end{itemize}

	\subsubsection{Was ist ein Semi-Broadcast-Medium?}
	Drahtlose Netzwerke sind ein Semi-Broadcast-Medium da die Reichweite nicht unendlich ist und die Systeme nur die Dateneinheiten in ihrer Reichweite hören können, somit können Sender Kollisionen nicht erkennen denn sie treten beim Empfänger aus.
	
	Problem der "versteckten" und "ausgelieferten" Endsysteme
	
	\paragraph{Was sind verstecke Endsysteme?}
	Verstecke Endsysteme sind die Systeme die zwar mit dem Zielendsystem kommunizieren können nicht jedoch mit dem Quellsystem.
	
	\paragraph{Was sind ausgelieferte Endsysteme?}
	Ausgelieferte Endsysteme sind solche die im Bereich einer aktiven Kommunikation zwar im Sendebereich der Quelle aber nicht im Empfangsbereich des Ziels sitzen. Dieses ausgelieferte System will mit einem vierten System kommunizieren das sich nicht im Sendereich des Quellsystems befindet aber da sich das ausgelieferte System in diesem Bereich befindet ist für dieses System das Medium blockiert.  
	
	\subsubsection{Was ist das Problem bei nahen und fernen Endsystemen?}
	Ausgangslage sind drei Endsysteme. Endsystem A ist weit von Endsystem B und C entfernt, B und C jedoch nahe beieinander.
	Da die Signalstärke quadratisch zur Entfernung abnimmt kann es sein das wenn A und B gleichzeitig senden, A von B "übertönt" wird und C A nicht hören kann.
	
	\subsubsection{Was sind die grundlegenden Einordnungen der von und betrachteten Verfahren?}
	Es wurde Zeitmultiplexing und konkurrierende Verfahren betrachtet. Wobei besonders auf den Energieverbrauch und die Latenz der Kommunikation geachtet wurde.
	
	\subsubsection{Wieso achten wir auf den Energieverbrauch?}
	Sensorknoten sind häufig Batteriebetrieben, deswegen ist es eine wichtige aber sehr beschränkte Ressource.
	
	\subsubsection{Wie sparen wir Energie?}
	Indem wir den Funktransciever so häufig wie möglich deaktivieren
	
	\subsubsection{Was sind Beispiele für unnötigen Energieverbrauch?}
	\begin{itemize}
		\item Kollision: wenn mehrere Systeme gleichzeitig senden wird eine Sendewiederholung erforderlich
		\item Unnötiges Lauschen: Transciever ist aktiv obwohl nichts empfangen wird
		\item Mithören: System empfängt Dateneinheiten die gar nicht an es gerichtet ist.
	\end{itemize}
	\paragraph{Was kann man gegen unnötigen Energieverbrauch tun?}
	Gegen Kollisionen hilft Kollisionsvermeidung und gegen unnötiges Lauschen als auch Mithören hilft Duty-Cycling.
	
	\subsubsection{Wie funktioniert Kollisionsvermeidung?}
	Zur Kollisionsvermeidung wird eine seperate Signalisierung verwendet. Entweder über einen seperaten Kanal(Out-of-Band Signalisierung) oder über den gleichen Kanal(In-Band-Signalisierung). 
	Als Beispiel für die In-Band-Signalisierung: Multiple access with collison avoidance (MACA):
	\begin{itemize}
		\item Sender sendet kurze Request to Send (RTS) Dateneinheit
		\item Empfänger antwortet mit Clear to Send (CLS) Dateneinheit
		\item Sender sendet Daten
	\end{itemize}
	
	\paragraph{Können während des MACA keine Kollisionen auftreten?}
	Doch aber mit geringerer Wahrscheinlichkeit, da RTS und CTS sehr klein sind.
	
	\paragraph{Wie sieht das mit RTS/CTS und dem versteckten Endsystem aus?}
	A und C wollen zu B senden. A sendet als erstes RTS. B antwortet mit CTS (teilt die Dauer der Belegung mit). C empfängt CTS und weißt deswegen das es nicht senden darf.
	
	\paragraph{Wie sieht das mit dem ausgelieferten Endsystem in RTS/CTS aus?}
	B will zu A senden und C will mit D kommunizieren. C empfängt RTS von B muss aber dann nicht warten, da es kein CTS von A empfängt.
	
	\subsubsection{Was ist Duty-Cycling?}
	Duty-Cycling ist die Idee den Funktransciever so oft wie möglich zu deaktivieren um Energie zu sparen, was leider die Latenz erhöht.
	
	Bei Duty-Cycling werden die Funktransciever bei Bedarf aktiviert um auf Aktivität zu prüfen oder zu Senden.
	Es wird zwischen synchronem und asynchronem Duty-Cycling unterschieden.
	\paragraph{Was ist synchrones Duty-Cycling?}
	Synchrones Duty-Cycling bezeichnet das koordinierte aufwecken der Funktransciever nach einem vorgegeben Plan.
	\paragraph{Was ist asynchrones Duty-Cycling?}
	Asynchrones Duty-Cycling bezeichnet das aufwecken der Funktransciever ohne Koordination.
	
	\subsubsection{Was ist S-MAC?}
	Sensor MAC ist ein Medienzugriffsprotokoll welches die Knoten zeitlich Synchronisiert.
	\paragraph{Wie funktioniert S-MAC?} 
	Die Idee hinter S-MAC ist das durch koordiniertes Schlafen idle listening verhindert wird, dazu erwachen die System gemeinsam an einem "rendezvous-point", also eine zeitliche Synchronisation.
	Hierzu wird ein Rahmen fester Länge genutzt der in 2 Phasen aufgeteilt wird.
	\begin{enumerate}
		\item Listen Phase: In dieser Phase findet eine Synchronisation statt und sofern gewünscht wird der Datenaustausch angestoßen. Diese Phase ist nocheinmal in 2 Phasen unterteilt, wobei pro Zeitschlitz nur maximal eine Dateneinheit(SYNC, RTS, CTS) vorkommen darf.
		\begin{itemize}
			\item Sync-Phase: Synchronisation durch SYNC Dateneinheit, welche die Zeitspanne bis zum Beginn der nächsten Sleep Phase beinhaltet.
			\item RTS/CTS Phase: In dieser Phase wir über einen zufällig gewählten Zeitschlitz über RTS ein Datenaustausch angestoßen, dannach folgt Carrier Sense + RTS/CTS 
		\end{itemize}
		\item Sleep Phase: Sollte in der Listen Phase ein Datenaustausch angestoßen worden sein, so entfällt diese Phase, ansonsten wird für die Dauer dieser Phase die Funkschnittstelle temporär deaktiviert.
	\end{enumerate}
	
	\paragraph{Wie sieht so ein S-MAC Rahmen aus?}
	Ein S-MAC Rahmen hat eine feste Länge.
	
	\begin{figure}[H]
	\centering
	%\includegraphics[scale=0.3]{internetOfEverything/smac-rahmen}
	\end{figure}
	
	\paragraph{Wie lange muss die Listenphase sein?}
	Die Listenphase muss lange genug sein um SYNC, RTS und CTS innerhalb einer geeigneter Zeitschlitze zu übertragen, dabei ergibt sich die Läng durch Parameter der MAC und PHY Schicht, z.B. der Datenrate etc. sie ist also nicht frei wählbar.
	
	\paragraph{Wie lange muss die Sleep Phase sein?}
	Die Sleep Phase ergibt sich aus dem Rest des S\- MAC Zeitrahmens. Sie ist also frei wählbar.
	
	\paragraph{Wie berechnet man den Duty Cycle?}
	$\text{Duty Cycle} = \frac{t_{Rahmen}}{t_{Listen}}$
	
	\paragraph{Wie berechnet sich die S-MAC Rahmenlänge?}
	Die Rahmenlänge ist von Listen Phase und dem Duty Cycle abhängig. Daher ergibt sich $t_{Rahmen}=t_{Listen}+t_{Sleep}$
	
	\paragraph{Wozu dient die Synchronisation?}
	Die Synchronisation soll dem System den Anfang der nächsten Sleep Phase vermitteln
	
	\paragraph{Wie funktioniert die Synchronisation?}
	Die SYNC-Dateneinheit enhält die verbleibende Zeit bis zum Beginn der nächsten Sleep Phase, dabei lernt das System den Zeitplan von Nachbarn durch regelmässigen Austausch von SYNC Dateneinheiten. 
	Falls es noch kein Nachbar bekannt ist wird ein eigener Zeitplan gewählt. Hierdurch entstehen zeitlich synchronisierte "Inseln" welche jedoch durch die SYNC Dateneinheiten den Zeitplan benachbarter Inseln lernen und so die Grenzen überbrücken können.
	
	\paragraph{Was passiert mit Systemen die an Grenzen liegen?}
	Sie empfangen und verteilen mehrere Zeitpläne, haben dementsprechend weniger Sleep Time und benötigen deswegen mehr Energie
	
	\paragraph{Gibt es Erweiterungen?}
	Ja, Message Passing und Adaptive Listening
	
	\paragraph{Was ist Message Passing?}
	Möchte man eine große Einheit von Nutzerdaten übertragen, so steigt die Bitfehlerwahrscheinlichkeit mit der Länge der Dateneinheit. Nun kann man die Nachricht als ganzes Übertragen wobei es mit hoher Wahrscheinlichkeit zu einem Fehler kommt. Eine weitere Möglichkeit wäre die große Einheit in kleiner Einheiten aufzuteilen. Die Bitfehlerwahrscheinlichkeit würde sinken aber es würde ein Overhead durch RTS/CTS entstehen.
	Hier setzt Message Passing an. Es fragmentiert die Einheit in mehrere Dateneinheiten die aber alle als Burst nach einem einzigen RTS/CTS Handshake übertragen werden und dazwischen einzellne bestätigt werden. So muss nur die defekte kleine Dateneinheit neu übermittelt werden und der Overhead durch RTS/CTS wird vermieden.
	
	\paragraph{Was ist Adaptive Listening?}
	Adaptive Listening geht das Problem an, dass pro Listen/Sleep Phase nur eine Dateneinheit weitergereicht werden kann und es deswegen in einem Multihop-Szenario zu einer höheren Verzögerung kommt. 
	Hierzu führt Adaptive Listening nach der Übertragung einer Dateneinheit eine zusätzliche Phase ein um einen neuen Datenaustausch zu initiieren, die Adaptive Listening Phase(ALP). Hierzu wacht ein nicht an einer Übertragung beteiliger Knoten nach der Übertragungszeit, welche im CTS angekündigt wurde wieder auf und hört auf ein neues RTS
	
	\paragraph{Wie kann man den durchschnittlichen Energiebedarf pro Byte berechnen?}
	Gesammtbedarch aller Systeme geteilt durch die Anzahl von der Senke empfangener Bytes.
	
	\paragraph{Wie kann man die durchschnittliche Ende zu Ende verzögerung berechnen?}
	Summe aller Ende zu Ende Verzögerungen geteilt durch die Anzahl der Dateneinheiten.
	
	\paragraph{Wie berechnet sich der Ende zu Ende Goodput?}
		Gesamzahl von der Senke empfangener Bytes geteilt durch die Zeitspanne zwischen Versenden der ersten Dateneinheit bis zum Empfang der letzten Dateneinheit an der Senke.
		
	\paragraph{Was sind die Nachteile von S-MAC?}
	Geringer Durchsatz, hohe Latenz durch lange Wartezeiten auf nächsten freien Rahmen
	
	
	
	\subsubsection{Was ist B-MAC?}
	Berkley Media Access Control ist ein Medienzugriffsprotokoll welches keine zeitliche Synchronisation der Knoten vorsieht. Es soll durch kollisionsvermeidung und effiziente Kanalnutzung bei hohen und niedrigen Datenraten einen Energieefizienten Betrieb ermöglichen. Dabei wird der Kanal periodisch geprüft anstatt auf eine zeitliche Synchronisation zu setzen. Dazu wird in zwei Zustände unterteilt
	\begin{itemize}
		\item Low Power Listening(LPL): Der Ruhezustand in dem das System die überwiegende Zeit befindet. Es wacht gelegentlich nur kurz auf um auf Daten auf dem Kanal zu überprüfen und bleibt dann Wach wenn etwas Empfangen wird, andernfalls geht es wieder schlafen.
		\item Clear Channel Assesment(CCA): Der Zustand in dem ein System etwas Senden möchte. Hierzu wird überprüft ob der Kanal frei ist, wenn frei dann wird erst eine Präambel gefolgt von den eigentlichen Daten übertragen.
	\end{itemize}
	
	\paragraph{Wozu dient die Präambel des CCA?}
	Sie stellt sicher, dass der Empfänger die Übertragung nicht verpasst. Daher das Intervall zwischen aufeinanderfolgenden Kanalüberprüfungen legt die Präambel Mindestdauer fest.
	
	\paragraph{Was ist der Vorteil von B-MAC?}
	\begin{itemize}
		\item Keine Synchronisation der Systeme notwendig
		\item Das Protokoll ist einfach und kompakt implementierbar
		\item Geringe Ende zu Ende Latenz
	\end{itemize}
	\paragraph{Welchen Nachteil hat B-MAC?}
	Das Problem der Versteckten Endgeräte wird nicht behandelt und es findet bei großen Nachrichten keine Fragmentierung statt. Auch nicht beteiligte Empfänger wachen durch die Präambel auf(Idle Listening). 
	
	\paragraph{Gibt es bei B-MAC Erweiterungen?}
	Ja es gibt X-MAC, es geht das Problem der langen Präambel des Clear Channel Assesments an, welches beim Sender einen hohen Energieverbrauch darstellt. Die Idee bei X-MAC ist statt einer großen langen Präambel viele kleine Präambelpakete zu senden. 
	Für Unicast wird hier die Zieladresse in die Präambelpakete integriert wodurch der Empfänger den Empfang quitieren kann und die Präambelphase verkürzt werden kann.
	
	
	\subsubsection{Wie stehen die beiden Zugriffsverfahren im Vergleich da?}
	B-MAC erreicht fast das 4,5 fache des Durchsatzes von S-MAC Unicast durch gerichgen Rechenaufwand pro Dateneinheit und Effektivität des Clear Channel Assesmensts.
	
	\subsubsection{Welchen Einfluss haben MAC-Prokolle auf den Energiebedarf von Kommunikationsvorgängn?}
	Die häufigste Annahme ist, das eine höhere Anzahl an Datenpaketen  , also mehr Datenvolumen, einen höheren Energiebedarf mit sich bringt. Aber das stimmt so nicht, auch häufige CCA Phasen können zu einem erhöhten Energiebedarf führen. Wohingegen lange Schlafphasen den Energiebedarf auch nicht zwangsweise reduzieren, es gibt einen anwensundsabhängigen "optimal" Wert.
	
\subsubsection{802.15.4}
	\paragraph{Was ist 802.15.4}
	IEEE 802.15.4 ist der IEEE Standard für WPAN Anwendungen. Er wird z.B. von 6LoWPAN genutzt.
	802.15.4 ist für kleine bis mittlere Datenraten und moderate Verzögerungen Ausgelegt. Es wurde so gebaut das es den Batteriebetrieb über Monate und Jahre übersteht (geringer Energiebedarf)
	
	
	PHY + MAC Protokoll im IoE
	Soll WLAN/Bluetooth ersetzen.
	
	\paragraph{Welche Eigenschaften hat es?}
	\begin{itemize}
		\item Frequenzbänder: 868/914/2450 MHz
		\item Datenrate max. 250 kbit/s
		\item Reichweite 10m
	\end{itemize}

	\paragraph{Wie groß ist eine 802.15.4 Dateneinheit?}
	Maximal 127 Byte
	
	\paragraph{Was für ein Einteilungsverfahren kommt bei 802.15.4 zum Einsatz?}
	Ein kombiniertes Zeitplan-basiertes und Konkurrenz-basiertes Verfahren.
	
	\paragraph{Ist 802.15.4 Echtzeitfähig?}
	Ja es gibt garantierte Zeitschlitze
	
	\paragraph{Auf welchem Medienzugriffsverfahren beruht 802.15.4}
	CSMA/CA, wie auch WLAN
	
	\paragraph{Welche Typen von Netzen sind möglich?}
	Es sind Peer-to-Peer-Netze (Mesh) und Stern-Netze (Star) möglich.
	
	\paragraph{Welche Klassen von Systemen gibt es?}
	\begin{itemize}
		\item Vollfunktionsfähige Systeme (Full Function Devices, FFD)
		\item Eingeschränkte Systeme (Reduced Function Device, RFD, nur in Stern-Netzen)
	\end{itemize}
	
	\paragraph{Gibt es ein spezielles Full Function Device?}
	Ja es gibt den "Koordinator", dieser Organisiert das Netz mittels periodischer Beacons. z.B. zum Synchronisieren der Systeme, Auffoderung zum Senden, Identifikation des Netzes etc...
	
	\paragraph{Welche Betriebsmodi gibt es?}
	Beacon Modus und Non-Beacon Modus
	
	\paragraph{Was ist der Beacon Modus?}
	Im Beacon Modus sind in einem Stern-Netz die Systeme einem Koordinator zugeordnet und Formieren ein Personal Area Network (PAN), welches durch den PAN-Identifikator identifiziert wird.
	
	Der Koordinator führt hierbei Buch über die Systeme und die Vergabe von Adressen, vermittelt zwischen Systemen und Peer Koordinatoren, bearbeitet Anforderungen für garantierte Zeitschlitze und sendet regelmäßig Beacons z.B. mit dem PAN-Identifikator.
	Hierbei gilt die grundlegende Annahme, dass für den Koordinator Energie keine limitierte Ressource darstellt, deswegen übernehmen sie mehr Aufgaben als andere Systeme.
	
	\paragraph{Wie sieht die Rahmenstruktur im Beacon Modus aus?}
	Der Rahmen hat feste Länge und ist unterteilt in eine inaktive und eine aktive Phase. In der inaktiven Phase können alle Systeme ihre Transciever ausschalten. Die aktive Phase dauert 16 Zeitschlitze, währen dessen der Koordinator aktive sein muss. 
	Die Zeitschlitze sind in 2 Teile aufgeteilt.
	\begin{enumerate}
		\item Contention Acces Period(CAP): Konkurrierender Zugriff durch Slotted CSMA/CA
		\item Guaranteed Time Slots(GTS): Vom System beim Koordinator reservierte Zeitschlitze, max. sieben aufeinanderfolgende. Solange bis freigegeben
	\end{enumerate}

	\begin{figure}[H]
		\centering
		%\includegraphics[scale=0.3]{internetOfEverything/802154-beacon-rahmen}
\end{figure}			

	\paragraph{Wie funktioniert der konkurrierende Zugriff während CAP?}
	Der Zugriff ist über Slotted CSMA/CA.
	Bei Carrier Sense Multiple Access wird der Kanal abgehört(CS), bevor gesendet wird. Dies bietet keinen Schutz vor versteckten Endsystemen. Bei Collision Avoidance kommt ein exponentieller Backoff zur Kollisionsvermeidung zum Einsatz.  
	
	\paragraph{Wie funktioniert die Vergabe von garantierten Zeitschlitzen?}
	Das System frägt während CAP einen Zeitschlitz beim Koordinator an, wobei hier zwischen einem Sende und Empfangsschlitz unterschieden wird. Der Koordinator quittiert diese Anfrage mit einem ACK, wobei das noch keine Zuweisung eines Zeitschlitzes darstellt. Bekommt das System vom Koordinator einen Zeitschlitz zugeteilt, so steht dieser im nächsten Beacon.
	
	\paragraph{Wie lange ist ein garantierter Zeitschlitz reserviert?}
	Solange bis das System ihn explizit freigibt oder der Koordinator feststellt, dass die Zeitschlitze eine gewisse dauer nicht genutzt wurden.
	
	\paragraph{Wie sieht eine Datenübertragung aus?}
	Bei der Datenübertagung muss zwischen den Wegen vom Koordinator zum System und umgekehrt unterschieden werden.
	\begin{itemize}
		\item System -> Koordinator: Falls das System einen Zeitschlitz zum senden reserviert hat kann es diesen Zeitschlitz direkt nutzen, wobei Dateneinheit + Inter Frame Space(IFS) + ACK  in den Zeitschlitz passen müssen. Hat das System keinen Zeitschlitz muss es während CAP mittels CSMA/CA senden.
		
		\item Koordinator -> System: Falls das System einen reservierten Zeitschlitz zum empfangen hat, nutzt der Koordinator diesen, wobei Nachricht + ACK vom System in den Zeitschlitz passen müssen. Hat das System keinen Zeitschlitz zum empfangen reserviert so kündigt der Koordinator seinen Sendewunsch im Beacon an. Er setzt die Adresse des Systems im "pending address" Feld. Das System antwortet mit Daten Anfrage, was vom Koordinator mit ACK quittiert wird. Daraufhin sendet der Koordinator die Daten und das System Quitiert den Empfang.
	\end{itemize}
	\begin{figure}[H]
		\centering
		%\includegraphics[scale=0.3]{internetOfEverything/802154-data-koordinator-system}	
	\end{figure}
	
	\paragraph{Was ist der Non-Beacon Modus?}
	Im Non-Beacon Modus gibt es keine feste Rahmenstruktur und damit auch keine garantierten Zeitschlitze. Der Zugriff auf das Medium ist duch unslotted CSMA/CA geregelt, es wird nicht Koordiniert oder Synchronisiert.
	 
	 In Peer-to-Peer Netzen können Systeme frei untereinander kommunizieren. Sie formieren ein PAN, welches durch den PAN-Identifikator identifiziert wird. Hierbei wird ein PAN Koordinator frei gewählt, dieser muss immer aktiv sein.

\subsection{Routing}
\subsubsection{Was für Verfahren haben wir behandelt?}
\begin{itemize}
	\item Probalistische Verfahren
	\item Inhaltsbasierte Verfahren
	\item Lokationsbasierte Verfahren
	\item Distanzvektorbasierte Verfahren
\end{itemize}
\subsubsection{Probalistische Verfahren}
\paragraph{Was für Verfahren hatten wir im Probalistischen Bereich?}
	Fluten und Gossiping
	
\paragraph{Wie funktioniert Fluten?}
	Der Knoten sendet jede Dateneinheit per Broadcast, über eine Broadcast-MAC-Adresse, an alle Nachbarn. Die Empfangenen Knoten leiten die Dateineinheit ihrerseit wieder per Broadcast an alle ihre Nachbarn weiter. Diese Methode ist dezental und selbstorganisierend sorgt aber für eine hohe Netzbelastung.
	\subparagraph{Was für Vorteile hat Fluten?}
	Es benötigt keine Routenfindung und keine Wartung der Topologie. Ebenfalls ist keine Routingtabelle, d.h. keine Zustandshaltung bzw. wartung notwendig.
	\subparagraph{Welche Nachteile hat Fluten?}
	Fluten erzeugt Dateneinheiten die unendlich lange im Netz kreisen und sich duplizieren, daher zu einer hohen Netzbelastung führen. Es bietet keine Zuverlässigkeit.
	\subparagraph{Wie kann man dieses Verfahren verbessern?}
	Z.B. durch Vermeidung der Duplikatbildung oder der Begrenzung der Reichweite.
	\subparagraph{Wie kann man die Duplikatbildung vermeiden?}
	Indem man jede empfangene Dateneinheit nur genau einmal weiterleitet. Hierzu müssen sich die Sensorknoten allerdings merken, welche Dateneinheiten sie bereits weitergeleitet haben und diese müssen auch eindeutig identifizierbar sein.
	\subparagraph{Wie beschränkt man die Reichweite?}
	Man führt eine Time-to-Live (TTL) ein, welche bei jedem Hop dekrementiert wird und bei 0 verworfen wird.
	Dies sorgt für eine geringere Netzlast bietet aber keine Zuverlässigkeit.

\paragraph{Was ist Gossiping?}
	Gossiping ist angelehnt an das Verbreiten von Gerüchten, Dateneinheiten werden an zufällige ausgewählte Knoten weitergeleitet, bis "ausreichen viele" Knoten die Dateneinheit erhalten haben
	\subparagraph{Wie geht man da vor?}
	Ein Knoten leitet eine Dateneinheit mit Wahrscheinlichkeit $p$ an Nachbarn weiter, hierbei wird wenn weitergeleitet wird per Broadcast an alle 1-Hop-Nachbarn weitergeleitet.
	\subparagraph{Was sind typische Werte für p?}
	Zwischen 65\% bis 75\%, bei $p<65\%$ ist die Wahscheinlichkeit hoch, dass eine Dateneinheit verworfen wird, bevor sie ihr Ziel erreicht. 
	\subparagraph{Was für Vorteile hat Gossiping?}
	Es hat einen geringeren Overhead als Fluten und es werden keine duplizierten Dateneinheiten versendet.
	\subparagraph{Welche Nachteile hat Gossiping?}
	Keine Zuverlässigkeit, da nicht alle Knoten im Netz die Dateneinheit erhalten und die Dateneinheiten sind eventuel lange unterwegs.
\paragraph{Kann man das kombinieren?}
Ja man kann auf den $k$ ersten Hops fluten und dann Gossiping mit Wahrscheinlichkeit $p$ betreiben. Dies Bezeichnet man als $GOSSIP(p,k)$, hier wären $GOSSIP(1,k)$ normales Fluten und $GOSSIP(p,0)$ normales Gossiping. Erfahrungswerte zeigen das mit $p=0,72$ und $k=4$ nahezu alle Knoten die Dateneinheiten erhalten.

\paragraph{Kann man Gossiping verbessern?}
	Ja, man kann die Wahrscheinlichkeit $p$  erhöhen, desto näher eine Dateneinheit ihrem Ziel kommt. Dazu muss natürlich jeder Knoten die Distanz zum Ziel kennen. Dies kann z.B. über Quittungen von erfolgreich empfangenen Einheiten passieren.
	
	\subsubsection{Inhaltsbasierte Verfahren}
	\paragraph{Was für inhaltsbasierte Verfahren haben wir behandelt?}
	Directed Diffusion und Rumor Routing.
	
	\paragraph{Was ist Directed Diffusion?}
	Directed Diffusion ist ein Datenzentrisches Verfahren bei dem klassiche Adressen keine Rolle spielen, hier stehen die Daten im Mittelpunkt. Das System basiert auf Fluten und ist selbst organisierend.
	Die grundlegende Idee ist das Daten durch Attribut-Wert-Paare benannt werden und eine Beobachtungsaufgabe als Interesse im Netz verbreitet wird(Diffusion). Hierdurch wird ein Gradient in den Knoten im Netz etabliert und gemessene Ereignisse folgen mehreren Pfaden zum Interessenten (Senke)(directed) wodurch sich einei Pfade verstärken. Hierbei sind die Sensorkonoten, welche die Datenanfrage beantworten können jedoch unbekannt und es können auch antworten von mehreren Sensorknoten kommen.
	\subparagraph{Wozu ist Directed Diffusion besonders geeignet?}
	Es eignet sich besonders wenn eine Datenanfrage mehrere oder regelmäßige Antworten erhalten soll, z.B. regelmäßige Temperaturmessungen.
	
	\subparagraph{Wie wird Directed Diffusion ausgeführt?}
	\begin{itemize}
		\item Senke äußert ihr Interesse an Daten, das Interesse wird im Netz geflutet.
		\item Ein Pfad zu Datenquelle wird aufgebaut. Der Gradient, also die Richtung aus der das Interesse kam wird im Knoten gespeichert.
		\item Daten werden auf dem Rückwärtspfad gesendet. Über alle verfügbaren Gradienten.
		\item Ist die Ereignie-Quelle gefunden startet die Senke die Reinforment-Phase, die Datenrate zu diesem Interesse wird erhöht der Gradient also Verstärkt. Es werden einige Pfade zwischen Quelle und Senke etabliert und nur noch diese Pfade mit geringer Verzögerung gewählt.
	\end{itemize}
	
	\paragraph{Was ist Rumor Routing?}
	Rumor Routing ist ein Datenzentrisches Routingverfahren bei dem die Initiative von der Ereignis-Quelle und der Ereignis-Senke ausgeht. Hier wird ein Kompromiss zwischen Fluten von Anfragen und Fluten von Ereignissen eingesetzt.
	Rumor Routing setzt einen Agentbasierten Ansatz ein, bei dem ein Ergeignis-Agent den Pfad zu Ereignissen etabliert. Er besitzt Informationen über Ereignisse und werden bei neuen Ereignissen mit eine gewissen Wahrscheinlichkeit erzeugt. Sie Wandern als langlebige Dateneinheit(max. Lebensauer als TTL) durch das Netz und können auf ihrem Weg informationen lernen. Auf ihrem Weg hinterlassen sie Pfadinformationen in den Knoten, als Anlehnung an Ameisen.
	\subparagraph{Welche Informationen müssen die Sensorknoten halten?}
	Die Sensorknoten müssen Weiterleitungsinformation zu bekannten Ereignissen speichern, welche sich durch ankommende Ereignis-Agenten aktualisieren. Die Einträge sollten eine limitierte Lebensdauer haben. Ebenso müssen sie ihre Nachbarschaftsinformationen speichern, diese werden z.B. durch periodische Hello-Dateneinheiten ausgetauscht.
	\subparagraph{Wie bewegen sich die Agenten durch das Netz?}
	Die Agenten werden per "Random Walk" von einem Knoten entweder per Unicast-MAC-Adresse an einen zufällig gewählten Nachbar weitergegeben oder alternativ werden mehrere Kopien an unterschiedliche Nachbarn weiteregeben, ähnlich Gossiping.
	
	\paragraph{Wo liegt der Nachteil bei Daten und Adressorientieren Routingverfahren?}
	Bei Adressorientierten Roting Verfahren sind Global eindeutige Adressen erfoderlich, bei Datenorientieten Verfahren ist der Aufwand für eine einzelne Übertraguen höher.
	
	\subsubsection{Lokationsbasierte Verfahren}
	
	\paragraph{Was will man mit lokationsbasierten Verfahren erreichen?}
	Mit lokationsbasierten Verfahren will man von klassischen Adressen abstrahieren und nur Informationen von einem System erhalten das in einer bestimmten Richtung in einer bestimmten Entfernung liegt.
	
	\paragraph{Was ist das distanzbasierte Greedy Verfahren?}
	Bei dem distanzbasiertem Greedy Verfahren geht man davon aus, dass jedes System seine eigene Position kennt. So wird die Strategie " most forward withing r " angewendet was so viel bedeutet wie das die Dateneinheit an das Nachbar-System weitergeleitet wird, das sich am nächsten am Ziel befindet.
	\subparagraph{Was für Vorteile hat dieses Verfahren?}
	Es ist garantiert frei von Schleifen
	\subparagraph{Welche Nachteile hat dieses Verfahren?}
	Da die Topologie nicht bekannt ist, wird nicht immer der kürzeste Pfad gefunden. Da als nächste Hop die Systeme mit potentiell größter Entfernung gewählt werden kann es mit steigender Wahrscheinlichkeit zu Fehlübertragungen kommen.
	
	\subparagraph{Wie kann man das Verbessern?}
	Verbessert wird dieses Verfahren durch das richtungsbasierte Greedy Verfahren. Hier ist die Idee das man als Next-Hop das System möglichst nahe an der idealen Richtung zum Ziel wählt. Als 
ideale Richtung wird hierbei die Verbindungsgerade zwischen aktuellem System und Ziel angesehen. Zwei sinnvolle Metriken sind hierbei einmal der minimale Abstand zur Verbindungsgeraden und einmal der minimale Winkel zur Verbindungsgeraden. Allerdings ist dieses Verfahren nicht garantiert Schleifenfrei.
	\subparagraph{Wo haben diese Verfahren Probleme?}
	Distan und richtungsbasiertes Greedy Verfahren haben Probleme mit Sackgassen (lokale Etremstellen)	
	\subparagraph{Wie kann man aus einer Sackgasse entkommen?}
	Mit dem Greedy Perimeter Stateless Routing bzw. der Rechte Hand Regel.
	
	\paragraph{Was ist Greedy Perimeter Stateless Routing?}
		GPSR ist eine Erweiterung des distanzbasieten Greedy Verfahrens. Es teil sich in zwei modi auf
		\begin{itemize}
			\item Greedy-Modus: Schnelle Weiterleitung von Dateneinheiten per distanzbasiertem Gerry Verfahren
			\item Perimeter-Modus: Modus zum Entkommen aus Sackgassen per Perimeter-Routing bzw. " Rechte Hand Regel ".
		\end{itemize}
		GPSR wechselt zwischen den Modi hin und her.
	\subparagraph{Wie läuft GPSR ab?}
	Es wird im Greedy Modus gestartet. Im Greedy Modus wird solange per distanzbasiertem Greedy Verfahren Weitergeleitet bis eine Sackgasse vorliegt, dann wird in den Perimeter Modus gewechselt.
	
	Perimeter Modus
	\begin{itemize}
		\item Die Position des Eintrittspunktes in den Perimeter Modus wird in der Dateneinheit vermerkt
		\item Weiterleitung um Face, welche von der Verbindungsgeraden zwischen aktuellem System und Ziel geschnitten werden
		\item Leite Dateneinheit gemäß der Rechte Hand Regel entlang des entsprechenden Perimter weiter, also um die Face Fläche herum
		\item Wechsel in den Greedy Modus, sobald das weiterleitende System dem Ziel näher ist als der Eintrittspunkt.
	\end{itemize}
	
	\subsubsection{Distanzvektorbasierte Verfahren}
	
	\paragraph{Was sind die Anforderungen für distanzvektorbasierte Verfahren?}
	Neben den üblichen Anforderungen wie geringer Energieverbrauch, geringe Datenrate etc.  sollte es für den Betrieb oberhalb eine Low-Power Medienzugriffsprotokolls  unabhängig vom MAC-Protokoll sein und u.a. über 802.14.5 arbeiten können.
	\paragraph{Was für ein Routing Protokoll gibt es hier?}
	Hier gibt es das Distanz-Vektor basierte Routingprotokoll RPL (Ripple) es soll Destination-oriented Distributed Acyclic Graphs (DODAGs) aufbauen. Es nutzt eine Objective Funktion zum Aufbau und basiert auf Metriken und Randbedingungen sogenannten Constraints.
	\subparagraph{Wie funktioniert das?}
	Verbindung von Sensor-basierten Zugangsnetzen zum Internet über dedizierte Knoten(Wurzel)
	\begin{itemize}
		\item{up} Verkehr in Richtung Internet, Sammeln von Daten (Concast)
		\item{down} Verkehr in Richtung Knoten, z.B. Code-Updates (Multicast)
	\end{itemize}
	
	Das ganze ist als gerichteter azyklischer Graph aufgebaut, wobei die Wurzel einen dedizierten Knoten darstellt. Die Wurzel ist ein LowPAN Border Router und wurde vom Systemadministrator konfiguriert.
	
	\subparagraph{Wie wird ein DODAG konstruiert?}
	Es wird bei der Wurzel gestartet. Die Knoten in der Nachbarschaft empfangen eine Nachricht und entscheiden ob sie dem Graphen beitreten oder nicht. 
	Wenn der Knoten beitritt besitzt er nun die Route zur Wurzel, welche sein Elternknoten ist. Falls der Knoten als Router arbeitet, so verbreitet er die Routinginformation in seiner Nachbarschaft, ist er nur ein Blattknoten war es das
	
	\subparagraph{Wie vielen DODAGS kann ein Knoten angehören?}
	Genau einem
	
	\subparagraph{Können sich DODAGs anpassen}
	Ja, DODAGs können sich dynamisch an Änderungen im Netz durch die Wurzel kontrolliert anpassen. Dies kann auch Topologieänderungen zu folge haben.
	
	\subparagraph{Was ist der Node Rank?}
	Der Node Rank ist ein Skalarer Wert der die relative Position eines Knotens im DODAG angibt, sozusagen die tiefe des Knotens.
	
	\subparagraph{Was ist die Objective Funktion?}
	Die Objective Funktion formuliert ein Pfadkosten Kriterium mit dem Metriken und Constraints in den Node Rank umgewandelt werden.
	Hiermit können z.B. auch Pfade über Batteriebetriebene Knoten ausgeschlossen werden.
	Jeder DODAG entspricht der Ausprägung eine Objective Funktion.
	\subparagraph{Was sind die RPL Kontrollnachrichten?}
	\begin{itemize}
		\item{DAG Information Object(DIO): }Enthält Informationen die es dem Knoten ermöglichen RPL instanzen zu entdecken, Konfigurationsparameter zu elrnen und DODAG Eltenr zu wählen. Enthalten den NodeRank des Senders.
		\item{DAG Infomation Solicitation(DIS): }Gezielte Anforderung eines DIOs von einem Knoten
		\item{Destination Adertisment Object(DAO): }Kommuniziert Informationen über Routingziele entlang eines DODAGs
	\end{itemize}
	
	\subparagraph{Was für Typen von Knoten gibt es?}
	Es gibt storing und non-storing Knoten, sie sollen Routinginformationen für die Down Richtung etablieren, hierzu senden die Blattknoten DAO Nachrichten in Richtung Wurzel. Das Problem hierbei ist der Speicheraufwand in den Knoten, daher wird zwischen storing und non-storing Knoten unterschieden. Die non-storing Knoten halten keine Routinginformationen sondern die Wurzel fügt Source-Routing-Informationen in die Dateneinheit ein. Storing nodes halten die Routinginformation selbstängit. Hierbei ist kein Mischbetrieb zulässig.
	
	\subparagraph{Wie sieht es mit RPL und Sicherheit aus?}
	Sie ist wichtig aber aufgrund der ressourcenknappheit oft problematisch. RPL bietet 3 Optionen.
	\begin{itemize}
		\item Unsecured: keine Sicherheitsmaßnamehn für Kontrollnachrichten
		\item Pre-installed: Über vorinstalliertes gemeinsames Geheimniss gesicherte RPK Nachrichten
		\item Authentifiziert: Router müssen Schlüssel von Authentication Authority erwärben, Blattknoten können über vorinstallierten Schlüssel beitreten.
	\end{itemize}
	
	\subparagraph{Welche Schutzziele gibt es in RPL?}
	Integrität, Vertraulichkeit, Schutz vor Wiedereinspielen, Schutz vor Verzögerungen
	
	\subparagraph{Wie wird ein DODAG konstruiert?}
	\begin{itemize}
		\item Knoten schicken periodisch link-lokale Multicast DIO-Nachrichten
		\item Knoten hören auf DIO-Nachrichten und verwenden die Informationen um DODAG beizutreten oder zu verwalten.
		\item Elternwahl aufgrund von DIO Informationen(z.B: Node Rank) über die Objective Function
	\end{itemize}
	Dasraus entstehen Upward Routden in Richtugn DODAG Wurzel. Wurzel muss nicht bekannt sein, Node Rank reicht aus. Aufgrund der strengen Monotonie des Baumes wird automatisch richtig geroutet.
	
	\subparagraph{Wie sieht die DOGAG Konstruktion Distanzvektorbasiert aus?}
	Es werden Pfadkosten für den günstigsten Pfad zur Wurzel bekanntgegeben. Elternknoten werden solche, die die Pfadkosten minimiere. Hierzu wird eine Elternmenge und ein bevorzugter Elternknoten gewählt. Je nach Objective Funktion können auch mehrere bevorzugte Eltern gewählt werden.
	\subparagraph{Welche Probleme können bei distanzvektorbasiertem DODAG auftreten?}
	Schleifen und Count-to-infinity.
	
	\subparagraph{Kann es günstig sein, einen hohen Node Rank zu haben?}
	Ja so stehen einem eine größere Anzahl potentieller Elternknoten zur Verfügung und man selbst wird seltener mit der Weiterleitung von Daten beauftragt, spart als Energie und vermeidet Schleifenbildung.
	
	\subparagraph{Was ist die max\_depth rule?}
	Schleifenbildung ist unvermeidbar, deswegen schränkt die max\_depth rule die Elternwahl ein. Demnach sind alle Knoten erlaubt deren Node Rank kleiner ist als der eigene bzw. den eigenen Node Rank nicht mehr als einen bestimmten Wert überschreitet. Hierdurch werden Schleifen nicht verhindert schränken aber deren Größe ein. Damit kann man Erreichen das sich ein Knoten beim Ausfall eines Links einen neuen Elternknoten mit höherem Node Rank wählt.
	Hierzu überprüft man $NewRank \leq oldRank + DAGMaxRankIncrease$
	
	\subparagraph{Wie erkennt man Schleifen bei der Datenübertragung?}
	Bei RPl benutzt man Piggybacking von Routing-Kontrollinformationen um Schleifen zu erkennen. Diese werden in Feldern im Kopf der Dateneinheit gesetzt, dies nennt man auch datapath validation.
	Hierzu benötigt man den Node Rank des Absenders und die Richtung der Dateneinheit. Als Beispielt eine Dateneinheit hat sich verirrt und up als Richtung markiert. Nun hat der Sender einen geringeren NodeRank als der Empfänger, daher erkennt man das sich die Nachricht in die falsche Richtung bewegt hat und der DODAG inkonsistent ist, er muss repariert werden.
	
	\subparagraph{Wie repariert man einen DODAG?}
	Bei der Reperatur von DODAG unterscheidet man zwischen lokalen und globalen Strategien.
	\begin{itemize}
		\item Lokale Strategie: 
			\begin{itemize}
				\item Wenn ein Link-Ausfall oder eine Inkonsistenz erkannt wurde wird versucht einen neuen Elternknoten zu wählen und daraufhin den Node Rank anzupassen. 
				\item Konnte kein neuer Elternknoten gewählt werden kommt es zu dem sogenannten \textbf{Poison-and-Wait} hier wird der Knoten aus allen Eltern-Sets und allen potentiellen Kindern gelöscht. An allen notwendigen Stellen werden neue Eltern gewählt(falls nicht möglich rekursiv poison and wait). Nach eine gewissen Zeit darf sich der Knoten neu in den DODAG einbinden.
				\end{itemize}
				
		\item Globale Strategie: Bei der globalen Strategie wird der DODAG von der Wurzel ausgehen komplett neu aufgebaut. Dies ist deutlich teurer als eine lokale Reperatur findet aber bessere Pfade.
		\end{itemize}
		\subparagraph{Welche Annahmen können über Verkehrsmuster getroffen werden?}
		Multipoint to Point also Concast stellt den Löwenanteil des Verkehrs da und hier findet RPL redundante und optimierte Pfade.
		
		Point to Multipoint also Multicast ist selten und aktives Auslesen (s. Directed Diffuison) wird als sekundär betrachtet.
		
		Point to Point also Unicast wird so gut wie nie genutzt, hier müsste im non storing mode immer über die Wurzel kommuniziert werden.
		
		\subparagraph{Wie viele Nutzdaten bleiben bei RPL?}
		vom den 128Byte des 802.15.4 Rahmen bleiben 79 Byte.
		\subparagraph{Reicht die Nutzdatenlänge oder muss fragmentiert werden?}
		DIOs werden nicht selten über 80 Byte lang, also ja es muss fragmentiert werden.
		
		\subparagraph{Was bedeutet der Verlust eines Fragments?}
		Ein verlorenes Fragment bedeutet Paketverlust, das Netz ist hoch verlustbehaftet.
		
\subsection{Topologiekontrolle}

	\paragraph{Was ist Topologiekontrolle?}
	Jede direkte Kommunikationsverbindung zwischen zwei Systemen ist Teil einer Topologie. Ein dichtes Netz führt zu einer komplexen Topologie. Die Topologiekontrolle dient daher dazu diese Komplexität zu reduzieren ohne die Konnektivität zu beeinflussen.
	
	
	\paragraph{Wie teilt man die Topologiekontrolle ein?}
	In der Topologiekontrolle unterscheidet man zwischen 
	\begin{itemize}
		\item Flachen Netzen: hier sind die Systeme gleichberechtigt.
		\item Hierachsiche Netze: hier haben die Systeme unterschiedliche Aufgaben und Fähigkeiten.
	\end{itemize}
	
	\paragraph{Wie kontrolliert man in Flachen Netzen?}
	Entweder durch Regulierung der Sendeleistung oder durch Regulierung der Zahl der Nachbarn.
	
	\subparagraph{Wie reguliert man die Sendeleistung?}
	Die Sendeleistung und die Zahl der direkten Nachbarsysteme hängen unmittelbar zusammen, also ist das Problem die richtige Sendeleistung zu finden. Eine geringere Sendeleistung spart Energie also will man eine möglichst geringe Sendeleistung mit guter Konnektivität und hohem Durchsatz. Hierzu gibt benötigt man die Position aller Systeme und kann mit Hilfe von \textbf{Minimale maximale Sendeleistung (MMS)} die Sendeleistung regulieren:
		\begin{itemize}
			\item Für jedes System $i$, setze Sendeleistung $P_i=0$
			\item Zwei Teilnetze $T_1, T_2$ mit geringstem Abstand verbinden
			\item Sendeleistung $P_i, P_j$ der zwei Systeme $i \in T_1, j \in T_2$, die am dichtesten zusammen liegen so erhöhen, dass diese miteinander kommunizieren können
			\item Abschließend Verbindungen, welche die Konnektivität nicht beeinflussen entfernen.
\end{itemize}	
Dieser Zentralisierte Ansatz ist in Sensornetzen so nicht praktikabel.

	\paragraph{Wie kann man Hierachische Netze unterteilen?}
	In Backbone-Netze und in Cluste-Netze.
	
	\subparagraph{Was machen Backbone Netze?}
	Backbone Netze teilen die Systeme in zwei Klassen. Einige der Systeme bilden das Backbone und Systeme außerhalb des Backbones dürfen nicht miteinander kommunizieren. Dabei gilt für jedes System es ist entweder Teil des Backbones oder besitzt ein Backbonemitglied als direkten Nachbarn.
	
	\subparagraph{Was machen Cluster Netze?}
		Cluster Netze unterteilen die Sensoren in Gruppen wobei jedes System genau in einer Gruppe ist bis auf Systeme die zwei oder mehr Gruppen verbinden.
		
	\subparagraph{Wie funktioniert die Clusterbildung}
	Es werden Vertreter der Gruppe gewählt, die sogenannten Cluster-Heads. Sie bilden eine unabhängige Mänge bei der zwei Cluster-Heads nicht benachbart sein dürfen. Hier wird eine maximal unabhängige Menge gesucht. Hierzu tauschen die Knoten Attribute wir z.B. Knoten-ID oder Energiereserven lokal aus.  Derjenige mit dem größten Atrributwer wird Cluster-Head. Die direkten Nachbarn werden dann aus der Betrachtugn ausgeschlossen 
	
	\subparagraph{Wie bestimmt man das Gateway?}
	Um die einzellnen Cluster zu Verbinden müssen sich Gateways finden. Hierzu benötigt jeder Cluster-Head Verbindung zu allen anderen Cluster-Heads die maximal 3 Schritte entfernt sind, wodurch die Backbone konnektivität sichergestellt ist.
	
	\subparagraph{Was ist das Problem mit den Cluster-Heads?}
	Sie fallen schneller aus, da sie höher belastet werden.
	
	\subparagraph{Wie kann man dieses Problem lösen?}
	Durch LEACH
	
	\subparagraph{Was macht LEACH?}
	Es rotiert die Cluster-Heads und sorft so für eine Verteilung der Last.
	
	\subparagraph{Wie funktioniert LEACH?}
	Die Knoten bestimmen sich selbständig mit bestimmter Wahrscheinlicheit zu einem Cluster-Head und teilen dies den benachbarten Knoten mit und alle anderen Knoten ordnen sich dem nächsten in Reichweite befindlichen Cluster-Head zu. Hierzu ist LEACH in Runden unterteilt welche 3 Phasen durchlaufen.
	\begin{itemize}
		\item Advertisment Phase: Jeder Knoten entscheidet ob er Cluster-Head der aktuellen Runde sein wird.
		\begin{itemize}
			\item $P = $ Anteil Cluster-Heads im Netz
			\item $G = $ Menge der Knoten die in den letzten $\frac{1}{P}$ Runden nicht Cluster-Head waren
			\item $T(n) =$ Jeder Knoten $n$ bestimmt Zufallszahl $z$ zwischen 0 und 1 und wird Cluster-Head falls $z < T(n)$ \\
			$T(n)= \begin{cases}
			\frac{P}{1-P(r \mod \frac{1}{P})} & n \in G \\
			0 & n \notin G
			\end{cases}
			$
			\item Jeder Knoten wird einmal innerhalb von $\frac{1}{P}$ Runden Cluster-Head
		\end{itemize}
		Cluster-Heads broadcasten ihre Entscheidung in Form von Advertismen Nachrichten an die Nachbarknoten.
		
		\item Cluster-Setup-Phase: Aufgrund erhaltener Advertismens entscheiden sich die Knoten für ein Cluster und teilen dies dem Cluster-Head mit. Der Cluster-Head sammelt eingehende Meldungen, erstellt einen Zeitplan und teil diesen den Knoten mit.
		
		\item Steady-State-Phase: Knoten kommunizierne Gemäß erhaltenem Zeitplan oder schlafen, der Cluster-Head muss ständig aktiv sein.
	\end{itemize}
	Nach einem festen Zeitintervall wiederholt sich der Vorgang.
	
\subsection{Datentransport}
	\paragraph{Welche Arten von Datentransport haben wir behandelt?}
	Unicast, Multicast und Concast.
	
	\paragraph{Was ist Unicast?}
	Einfache direkte kommunikation zwischen zwei Endsystemen. Hier liegt die Herausforderung in der Zuverlässigkeit und einer minimierung der Verzögerung. Eine Typische Komminikationsform wäre Sensor nach Aktor. Z.B. ein Lichtschalter steuert eine Lichtquelle.
	
	\paragraph{Was bedeutet Zuverlässigkeit im traditionellen?}
	Alle Dateneinheiten kommen unveränder, in der richtigen Reihenfolge und Duplikatfrei an. Es gibt keine Phantom Dateneinheiten.
	
	\paragraph{Warum nicht einfach TCP?}
	TCP erkennt die Ursache für den Paketverlust nicht und nimmt an das eine Stausituation existiert was zu schlechter Performance führt.
	TCP liefert eine 100\% Zuverlässigkeit was viele Wiederholungen und damit einen hohen Energiebedarf zur folge hat. TCP ist Verbindungsorientiert und setzt einen 3 Wege Handshake voraus was zu einer hohen Latenz beim Verbindungsaufbau führt was für spontane Übertragung von Ereignissen eher ungeeignet ist. TCP ist für Unicast ausgelegt.
	
	\paragraph{Was ist probalistische Zuverlässigkeit?}
	Nur ein Teil(\%) aller Dateneinheiten erreicht die Senke, dies verbraucht weniger Energie, als alle Dateneinheiten zu transportieren und reicht bei redundanten Informationen aus, z.B. bei periodischem Messen. Hierbei simmen die Informationen welche die Senke erreichen nicht exakt mit den erfassten Informationen überein, z.B. da Informationen im Netz aggregiert werden. Die Information die ankommen sind auch aufgrund langer Verzögerungen bei schlechten Funkverbindungen veraltet.
	
	\paragraph{Was sind pro Hop Quittungen}
	 Bei pro Hop Quittungen quittiert jeder Hop die Dateneinheit. Bei einer notwendigen Wiederholung kann das direkte Vorgängersystem das Paket erneut übertragen, hierzu ist allerdings eine Zustandshaltung auf den Zwischensysteme und viele Quittungen zwischen Nachbarn nötig. Sehr teuer, braucht Speicher und Energie.
	 \subparagraph{Was sind die Vorteile von pro Hop Quittungen?}
	 Der Energieverbrauch wird im Netz verteilt und der Sender ist nicht in jede Senderwiederholung involviert.
	 
	 \paragraph{Was sind Ende zu Ende Quittungen?}
	 Bei Ende zu Ende Quittungen gibt es auf dem Weg keine Quittungen zwischen den Zwischensystemen lediglich das Ziel quittiert den Empfang. Wird die Quittung nicht innerhalb eines festen Zeitrahmens empfangen wird die Dateneinheit über den kompletten Pfad erneut übertragen.
	 
	 \paragraph{Was ist besser?}
	 Ende zu Ende Quittungen sind immer nötig aber pro Hop Sendewiederholungen sind bei hohen Bitfehlerraten oder langen Routen sinnvoll.
	 
	 \paragraph{Kann man das besser machen?}
	 Ja durch Hop-by-Hop Reliability(HHR). Es soll die Zuverlässigkeit ohne den Einsatz von Quittungen erhöhen.
	 Hierbei wird auf Quittungen und Sendewiederholungen verzichtet, dafür wird eine Dateneinheit sofort $k$ mal gesendet. $k$ ist so zu wählen, dass Dateneinheiten mit Wahrscheinlichkeit $r$ von Senke korrekt empfangen wird $r_i=1-p_i^{k_i}$ wobei $k_i$ mit Kenntnis der Paketfehlerrate $p_i$ eines Systems $i$ zum Nachbarsystem berechnet wird.
	 
	Außerdem gibt es noch HHR mit Acknoledments, hier wird bis zu $k_i$ mal gesendet aber jedes mal auf Quittung gewartet. Es wird aufgehört zu senden, sobald die erste Quittung empfangen oder $k_i$ erreicht ist.
	
	\subparagraph{Was hat das für Vor und Nachteile?}
	HHR hat den Vorteil das Übertragungen durch den Verzicht auf Quittungen eingespart werden können, denn bei niedrigen Paketfehlerraten werden Dateneineheiten seltener übertragen. Es hat allerdings auch den Nachteil, dass die Paketwiederholung nicht Stopt, sobald die Dateneinheit empfangen wurde.
	
	HHRA hat den Vorteil das bei hohen Paketfehlerraten auf Sendewiederholungen verzichtet wird, wenn eine Dateneinheit erfolgreich quittiert wurde und den Nachteil das der Overhead der Quittungen sich bei geringen Paketfehlerraten nicht lohnt.
	
	
	
	
	\paragraph{Was ist Multicast}
	Die Senke kommuniziert mit mehreren Sensorknoten/Aktoren. z.B. für Kontroll oder Steuerungsinformationen oder Code-Updates. Hier liegt die Herausforderung in der Zuverlässigkeit.
	
	\subparagraph{Pump Slowly Fetch Quickly (PSFQ)}
	Ziel von PSFQ ist es Daten von einer Basisstation zu einer Menge an Sensoren zu transportieren, hierbei ist Datenverlust inakzeptabel, Latenz aber nicht kritisch.
	\subparagraph{Wie funktioniert PSFQ?}
	Die Basisstation pumpt Dateneinheiten ins Sensornetz und wartet zwischen dem Versand von zwei Dateneineheiten. Die Systeme speichern Dateneineheiten, fall es neue sind. Die Systeme leiten die Dateneinheiten nur weiter(pumpen) falls die Sequenznummern konsistent sind, also keine Dateneinheit fehlt.
	
	Sollte eine Sequenznummer fehlen wird nach der verlorenen Dateneinheit gefragt. Die geschieht über eine Fetch Operation, dass ist ein NACK mit einer Liste der fehlenden Sequenznummern. Daraufhin versenden die Vorgängersysteme die verlorene Dateneineheit neu, weshalb empfangene Dateneineheiten immer gepuffert werden müssen.
	
	\subparagraph{Wie Antworten die Nachbarsysteme auf ein NACK?}
	Sie starten einen Timer und hören innerhalb dieser Zeit ob ein anderes Nachbarsystem auf das NACK antwortet, ansonsten senden sie nach ablauf des Timers das NACK.
	
	\subparagraph{Ist der NACK ein Broadcast?}
	Ja er erreicht mehrere Nachbarsysteme.
	
	Es wird langsam gepumpt um Fetchen zu ermöglichen und schnell gefetcht.
	\paragraph{Was ist Concast?}
	Concast bezeichnet eine Kommunikationsform bei der viele Quellen mit einer Senke kommunizieren, als Beispiel viele Sensoren liefern Daten an eine Senke. Hier liegt die Herausforderung neben der Zuverlässigkeit auch in der Staukontrollen und der Energie-Effizient, vor allem im periodischen Fall.
	\subparagraph{Wieso ist Stau ein Problem bei Concast?}
	Wenn einige Sensoren ein Ereignis beobachten und dieses an die Senke senden wollen kann es bei einer hohen Systemdichte zu einem Engpass kommen.
	\subparagraph{Welche Auswirkungen hat ein Stau?}
	Einmal Datenverluste, diese könnte man durch erneute Übertragung korrigieren dies kostet jedoch zusätzliche Energie und zum anderen können Daten verzögert ankommen, dagegen hilft nur weniger Daten zu senden.
		
	\subparagraph{Was ist ESRT?}
	Event-to-Sink Reliable Transport (ESRT) hat das Ziel in jeder Periode $i$ mit der Länge $\tau$ sollen ca. $R$ Dateneinheiten zugestellt werden. Hierzu wird die Senderate $f_i$ zum Zeitpunkt $i$ angepasst, indem die Senke in Periode $i$ die Anzahl $r_i$ der empfangenen Dateneinheiten misst und die Zuverlässigkeit $\eta_i=\frac{r_i}{R}$ m wobei $\eta_i \in [1-\epsilon, 1+\epsilon].$ Die Senke berechnet $f_{i+1}$ basierend auf $f_i$ und $\eta_i$ und sendet ihn per Broadcast an alle Systeme.
	\subparagraph{Wie bestimmt man $R$?}
	Durch Abschätzen, $R$ muss groß genug sein, damit Informationen an der Senke genau genug sind.
	\subparagraph{Wann wird bei ESRT ein Stau erkannt?}
	Wenn weniger Daten als gewünscht $R$ die Senke erreichen.
	
	\subparagraph{Gibt es noch ein anderes Verfahren für Concast?}
	Ja die Aggregation.
	
	\subparagraph{Was ist Aggregation}
	Ziel der Aggregation ist es Daten zusammenzufassen um weniger Dateneinheiten verschicken zu müssen.
	
	\subparagraph{Welche Formen der Aggregation gibt es?}
	\begin{itemize}
		\item Keine Aggregation: Direkte Weiterleitung
		\item Paketaggregation: Eingehende Daten werden zwischengespeichert und in einem großen Paket gesammelt weitergeleitet
		\item Datenaggregation: Eingehende Daten werden zwischengespeichert und mit Aggegationsfunktiontn zu einem Aggregat zusammengefasst und weitergeleitet.
\end{itemize}	 

	\subparagraph{Was ist besser Paket oder Datenaggregation?}
	Es gibt keinen signifikanten Unterschied, deswegen ist die Paketaggregation vorzuziehen denn hier liegen alle Quelldaten vor.
	
\subsection{Systeme}
\subsubsection{IETF}
	\paragraph{Was ist 6LoWPAN?}
	Das ist IPv6 over Low-Power Wirelles Personal Area Networks, also die Nutzung von IPv6 in Umgebungen mit stark eingeschränkten Ressourcen.
	Hierdurch erhalten die Knoten eine global eindeutige IPv6 Adresse und es sind alle Knoten global erreichbar. 
	Man kann die IPv6 Autokonfigurationsmechanismen benutzen und erspart sich so Konfigurationsoverhead.
	Hierbei ist jedes LoWPAN über einen Edge Router mit dem eigentlichen Internet verbunden. Auf dem Edge Router findet die Umsetzung von 6LoWPAN in IPv6 statt.
	
	\subparagraph{Wie sieht 6LoWPAN im Schichtenmodell aus?}
	802.15.4  regelt den Medienzugriff. Hier wären auch andere Protokolle wie BTLE oder NFC möglich.
	6LowPAN ist für die Komprimierung der Paketgröße und der Fragmentierung zuständig.
	RPL kommt zum routing zum einsatz. Der Fokus liegt auch Concast und Multicast aber auch Unicast ist möglich.
	CoAP kommt als Transferprotokoll für Umgebungen mit stark limitierten Ressourcen zum einsatz.
	\begin{figure}[H]
		\centering
		%\includegraphics[scale=0.25]{internetOfEverything/6lowpan_schichtenmodell}
		
		
	\end{figure}
	
	\subparagraph{Welche Aufgabe hat 6LowPan?}
	Die Aufgaben von 6LowPAN sind IP Konnektivität über IPv6 herstellen. Dies bringt einen großen Adressraum und Autokonfigurationlösungen von IPv6. Eine Inter-Konnektivität mit anderen IP-Netzen wird möglich.
	6LowPAN muss für eine geringe Paketgröße sorgen, hierzu wird Header Compression eingesetzt damit die Daten möglichst in eine 802.15.4 Dateneinheit passen, ansonsten muss Fragmentiert und Reassembliert werden.
	
	\subparagraph{Was für Datenmengen senden Anwendungen Typischerweise?}
	<100 Byte
	
	\subparagraph{Wie sieht es mit der Adressierung in 6LowPAN aus?}
	6LowPAN bildet IPv6 Adressen auf IEEE 802.15.4 Adressen ab. Hierbei haben alle Knoten eines LoWPANs den gleichen Adresspräfix. 
	
	\subparagraph{Was für Typen von Adressen sind bei IPv6 möglich?}
	\begin{itemize}
		\item Unicast: Identifikator für ein einzelnes Interface
		\item Anycast: Identifikator für eine Menge von Interfaces, Dateneinheiten mit einer solchen Adresse werden an \textbf{ein} Interface aus dieser Menge ausgeliefert, meist das " nächstgelegende "
		\item Multicast: Identifikator für eine Menge von Interfaces, Dateneinheiten werden an \textbf{alle} Interfaces ausgeliefert
	\end{itemize}
	
	
	\subparagraph{Wie wird die IPv6 Adresse in einem 6LowPAN berechnet?}
	64 Bit Präfix des LowPan + 64 Bit Interface Identifier(MAC-Adresse)
	
	\subparagraph{Wie viele Nutzdaten bleiben in einem 802.15.4 Paket in 6LowPan?}
	Maximal 102 Byte, Rest von Header benötigt. Im schlechtesten Fall 32Byte.
	
	\subparagraph{Wie groß ist ein IPv6 Kopf?} 
	40 Byte
	
	\subparagraph{Wie groß ist ein UDP Kopf?}
	8 Byte
	
	\subparagraph{Was ist Header Compression}
	Bei Header Compression werden unnütze, redundante Informationen aus dem Header entfernt um Daten zu sparen. 
	Hierbei wird unterschieden zwischen der Komprimierung des IPv6 Kopfes und des UDP Kopfes.
	Header Compression ist hierbei eine Zustandslose Komprimierung, daher es werden nur Datenkomprimiert die sich aus der Übertragung selbst herleiten lassen.
	Hierzu wird im Dispatch Header die LOWPAN\_ HC1 angegeben darauf folgt der LOWPAN\_ HC1 Header der angibt welche Teile des IPv6 Kopfes komprimiert wurden.
	
	\subparagraph{Was ist der Dispatch Header?}
	Der Dispatch Header gibt den nachfolgenden Paketkopftyp an, wobei nach dem Dispatch Byte unterschieden wird.
	
	
	\subparagraph{Was kann im IPv6 Kopf komprimiert werden?}
		\begin{itemize}
			\item Die Versionsnummer, sie ist immer 6
			\item IPv6 Ursprung und Zieladresse bei lokaler kommunikation aus Schicht-2 Adressen ableitbar
			\item Payload Lenght, kann u.U aus Schicht 2 Payload Length oder aus Datagram\_ size im Fragmentierungskopf berechnet werden
			\item Traffic Class und Flow Label sind meist 0
			\item Next Header ist UDP, ICMP oder TCP
		\end{itemize}
			
		
		\subparagraph{Auf welche Größe lässt sich der IPv6 Kopf komprimieren}
		2 Byte + 1 Byte Hop-Limit Feld
		
	\subparagraph{Was kann im UDP-Kopf komprimiert werden}
	Port Nummer, da keine $2^{16}$ Ports benötigt werden können weniger ausreichen. UDP Payload length lässt sich unter Umständen aus Schicht 4 Payload Length berechnen.
	
	\subparagraph{Auf welche Länge  lassen sich IPv6 Kopf + UDP-Kopf komprimieren?}
	6 Byte
	
	\paragraph{Was ist CoAP?}
	Constrained Application Protocol(CoAP) ist ein Transfer-Protokoll für ressourcenbeschränkte Kommunikation. Es ist sozusagen die leichtgewichtige Alternative zu HTTP auf UDP Basis. Sie besitzt ein kompaktes Format und unterstützt Representational State Transfer(RESTful) per GET, PUT, POST und DELETE Nachricht. Ebenso unterstützt es URIs.
	
	CoAP kann am Border Router einfach in HTTP umgesetzt werden.
	
	CoAP unterstützt DTLS, die TLS Variante für UDP.
	
	\subparagraph{Was für Schichten hat CoAP?}
	\begin{itemize}
		\item Message-Schicht: Mechanismen zur zuverlässigen Nachrichtenübertragung. 4 Nachrichtentypen: Confirmable, Non-confirmable, Acknoledgement, Reset. Übertragung von Request/Responses
		\item Request/Responses: Enthalten Methodenaufrug auf ein Objekt. 4 Methoden: GET, PUT, POST, DELETE
	\end{itemize}
	
	\subparagraph{Wie kommen Antworten auf eine CoAP Nachricht?}
	Entweder Piggy-backed im ACK oder als seperate Nachricht.
	
	\subparagraph{Was für Sicherheitsmodi gibt es für CoAP?}
	\begin{itemize}
		\item NoSec: DTLS wird nicht genutzt
		\item PreSharedKex: DTLS mit vorverteilten Schlüsseln
		\item RawPublicKex: DTLS mit asymmetrischen Schlüsselpaaren
		\item Certificate: DTLS mit X.509 Zertifikat
	\end{itemize}
	
\subsubsection{ZigBee}
	\paragraph{Was ist ZigBee?}
	ZigBee ist der von der ZigBee Alliance vorangetriebene Standard für das Internet der Dinge. ZigBee ist stark anwendungsorientiert und es stehen kleine autonome Netze im Mittelpunkt welche keine Interoperabilität mit bestehender Infrastruktur bieten.
	Als Beispiele wären hier Haus und Gebäudeautomation oder Beleuchtungssteuerung zu nennen. Es baut auf einem Server/Client Modell auf, hierbei sind dem Server Attribute zugeordnet welche vom Client über standadisierte Cluster-Befehle genutzt werden. Anstelle von Ports werden sogenannte Endpunkte benutzt, wobei ein Gerät über mehrere Endpunkte verfügen kann. Die Verknüpfung zweier Endpunkte wird in einer sogenannten Binding-Tabelle netzweit gespeichert.
	
	\subparagraph{Wie viele Endpunkte sind möglich?}
	256 Endpunkte sind möglich wobei 255 den Broadcast darstellt, 0 Verwaltungsfunktion besitzt und 241-254 für zukünftige Anwendungen reserviert ist.
	
	\paragraph{Welches Medienzugriffsprotokoll nutzt ZigBee?}
	ZigBee nutzt 802.15.4 als Medienzugriffsprotokoll.
	Der Datentransport stellt in ZigBee keine eigene Schicht da, ist Verbindungslos und sorgt für die Fragmentierung großer Dateneinheiten. Ende zu Ende Quittungen sind nur bei Fragmentierung verpflichtend und es gibt eine Dupplikaterkennung über Sequenznummern. Es gibt jedoch keine Fehlererkennung, Reihenfolgetreue(außer bei Fragementierung), Staukontrolle oder Flusskontrolle(außer bei Fragmentierung).
	
	\paragraph{Wie erfolgt die Quittung bei Fragmentierung}
	Es werden nicht die einzellnen Fragemente sondern ganze Fenster (1-8 Fragmente) Quittiert und zum Quittieren werden spezielle Dateneinheiten übertragen.
	Wird eine Quittung nicht vor Ablauf eines Timers empfangen oder es wird eine unerwartete Quittung empfangen, so wird die Dateneinheit bis zu drei mal erneut übertragen.
	
	\paragraph{Wie sieht es mit Multicast aus}
	Für Multicast werden Gruppen gebildet wobei es eine Gruppentabelle gibt die GruppenIP -> Endpunkte zuordnet. Endpunkte können mehreren Gruppen angehören.
	
	\paragraph{Wie wird Adressiert}
	Nicht über IP Adressen sondern über 16-Bit Kurzadressen
	
	\paragraph{Was für Rollen gibt es?}
	\begin{itemize}
		\item Koordinator(FFD): Übernimmt Verwaltung des Netzer, beeinflusst Topologie, Funkkanal etc.
		\item 	Router (FFD): Kann Dateneinheiten im Netz weiterleiten
		\item Endgerät (FFD oder RFD): Gerät ohne Weiterleitungsfunktion
	
	\end{itemize}
	
	\paragraph{Was für Topologien gibt es in Zigbee?}
	Stern, Baum und Mesh
	
\subsubsection{Industiral Internet}
	\paragraph{Was sind die Anforderungen}
	Harte Echtzeit und Verfügbarkeit, deterministischer Verhalten und geringe Latenz. Isolierung des Verkehrs unterschiedlicher Kunden.
	
	\paragraph{Time Slotted Channel Hopping(TSCH)}
	Medienzugriffsverfahren basierend auf synchronem Zeitmultiplexing und Channel Hoppint. 
	Durch das synchrone Zeitmultiplexing wird das Medium in feste Zeitschlitze eingeteilt	mit synchronisierten Schlaf und Wachzeiten und festgelegte Zeitschlitze zur Synchronisierung.
	Durch das Channel Hopping wird die Frequenz für jede übertragene Dateneinheit gewechselt, eine Sendewiederholung findet auf einer anderen Frequenz statt, dadurch geringere Wahrscheinlichkeit auf eine Kollision.
	
	\paragraph{Was ist 802.15.4e}
	Erweiterung für Industrial Internet. Es wird Time-Slotted Channel Hopping als Betriebsmodus eingeführt.