\chapter{Virtuelle Systeme}

Zusammenfassung der Vorlesung "`Virtuelle Systeme"' aus dem Wintersemester 2016.\footnote{\url{https://os.itec.kit.edu/deutsch/3257_3261.php}}

\section{Einführung}
\begin{itemize}
	\item "`Computer sind komplex"': Vereinfachung durch Abstraktionsebenen and wohldefinierten Schnittstellen
	\item Abstraktion: Versteckt Implementierungsdetails niedrigerer Ebenen und stellt vereinfachte Schnittstellen zu Ressourcen zur Verfügung \(\rightarrow\) vereinfacht das Design auf höheren Ebenen. Beispiel: Dateien als Abstraktion einer Festplatte mit einfacher, allgemeiner Schnittstelle (\texttt{open, read, write})
	\item \textbf{Schnittstellen}
	\begin{itemize}
		\item Instruction Set Architectur (ISA): Betriebssystemunabhängige Schnittstelle für den Hardwarezugriff; beispielsweise \texttt{IA-32}
		\item Application Binary Interface (ABI): Interface für den Hardwarezugriff, das vom Betriebssystem zur Verfügung gestellt wird; beispielsweise \texttt{SYSCALL}
		\item Nachteile: Eventuell schlechte Portabilität, da höhere Ebenen von entsprechenden Schnittstellen abhängig sind; beispielsweise ist Software an bestimmte ISAs und Betriebssysteme gebunden
	\end{itemize}
	\item Generelles Problem: Software an bestimmte \texttt{ISA} und Betriebssystem gebunden; Betriebssystem an bestimmte Hardwaretypen und Systemschnittstellen gebunden \(\rightarrow\) Reduzierung der Interoperabilität
\end{itemize}


\subsection{Virtualisierung}
\begin{itemize}
	\item Abstrakt: Isomorphismus zwischen Gast und Host (Gastoperationen werden auf enstprechende Hostoperationen abgebildet)
	\item \textbf{Aufgaben der Virtualisierung}
	\begin{itemize}
		\item Zuordnung von virtuellen zu realen Ressourcen
		\item Verwendung von realen Maschinenbefehlen oder \texttt{SYSCALL} um die Anweisungen des Gastsystem auszuführens
	\end{itemize}
	\item \textbf{Process Virtual Machines}
	\begin{itemize}
		\item Simuliert ein Programm/Prozess: Führt Anwendungen für die selbe oder eine andere \texttt{API}, \texttt{ABI} oder \texttt{ISA} aus \(\rightarrow\) abstrahiert \texttt{API}, \texttt{ABI} oder \texttt{ISA}
		\item Interaktion mit dem Host-Betriebssystem zur Laufzeit via \texttt{SYSCALL}
		\item Runtime zur Verwaltung der Gastprozesse \(\rightarrow\) Gastprozesse existieren parallel zu den Hostprozessen
		\item OS Compatibility Layer (emuliert \texttt{ABI} und \texttt{API})
		\begin{itemize}
			\item Virtualisiert Software: Bibliothek um von Gast-\texttt{API} nach Host-\texttt{API} übersetzen zu können
			\item Beispiel: \texttt{Wine}
		\end{itemize}
		\item Binary Translation (emuliert \texttt{ISA} und \texttt{ABI})
		\begin{itemize}
			\item Selbes Betriebssystem, verschiedene \texttt{ISAs}: Einführung einer Zwischenschicht zur Übersetzung der \texttt{ABI}-Befehle
			\item Beispiel \texttt{Digital FX!32}: Erlaubt das Ausführen von Windows-Anwendungen, die für \texttt{x86} kompiliert worden sind, auf \texttt{Alpha}-Prozessoren
			\item Binary Optimization: Optimierung zur Laufzeit innerhalb der selben Umgebung (beispielsweise \texttt{HP Dynamo})
		\end{itemize}
		\item High-level Language Runtime (emuliert \texttt{ISA}, \texttt{ABI} und \texttt{API})
		\begin{itemize}
			\item Stellt eine komplette virtuelle Umgebung zur Verfügung
			\item Spezifizierung eines high-level Interfaces. Das Betriebssystem wird als Standard-Bibliothek abstrahiert (beispielsweise \texttt{JVM})
		\end{itemize}
	\end{itemize}
	\item \textbf{System Virtual Machines}
	\begin{itemize}
		\item Abstrahiert die \textbf{ISA} des Hosts als separate, virtuelle Maschinen \(\rightarrow\) simuliert kompletten PC
		\item Interaktion über den \textit{Virtual Machine Monitor} (VMM)
		\item Typen
		\begin{itemize}
			\item Native System VM: VMM im privilegierten Modus; Gast im Benutzermodus; beispielsweise \texttt{Xen} oder \texttt{Hyper-V}
			\item Hosted System VM: VMM und Gast im Benutzermodus; beispielsweise \texttt{VirtualBox} oder \texttt{QEMU}
			\item Dual-Mode System VM: VMM teilweise im privilegierten Modus; beispielsweise \texttt{QEMU} mit \texttt{KVM}
		\end{itemize}
		\item Vorteile System VM
		\begin{itemize}
			\item Erhöhte Kompatibilität
			\item Höhere Effizienz, da mehrere virtuelle Maschinen auf einer physikalischen laufen können
			\item Zuverlässigkeit und Verfügbarkeit wird durch Replikations- und Migrationsmöglichkeiten erhöht
			\item Sicherheit durch Isolierung zwischen den Gästen
			\item Möglichkeiten zur System-Analyse auf \texttt{ISA}-Ebene (Debugging, Malware-Analyse, Forschung)
		\end{itemize}
	\end{itemize}
\end{itemize}



\section{Emulation}
\begin{itemize}
	\item Ziel: Implementieren von Interface/Funktionalität eines Systems auf einem anderen System
	\item \textbf{Interpretation vs. Binary Translation}
	\begin{itemize}
		\item Interpretation
		\begin{itemize}
			\item Quellinstruktion wird analysiert, in Zielinstruktion überführt und ausgeführt
			\item Einfach zu implementieren; allerdings mit hohen Ausführungskosten verbunden
		\end{itemize}
		\item Binary Translation
		\begin{itemize}
			\item Betrachtet immer einen ganzen Instruktionsblock, übersetzt diesen und speichert ihn zwischen
			\item Komplexer; beim Start höhere Ausführungskosten; allerdings geringe Ausführungskosten
		\end{itemize}
	\end{itemize}
	\item \textbf{Aufbau einer Process VM}
	\begin{itemize}
		\item Emulation Engine: Emuliert die Instruktionen (wahlweise mittels Interpretation oder Binary Translation; verantwortlich für die Auswahl der Emulationstechnik)
		\item Profile Database: Sammelt dynamisch Informationen zum laufenden Programm \(\rightarrow\) bildet Basis für Emulationstechnikentscheidungen und Optimierung im Translater
		\item Code Cache: Verwaltet die zwischengespeicherten, bereits übersetzten Code-Fragmente. \textit{Code Cache Manager} entscheidet, ob Fragmente im Cache abgelegt werden
		\item Interpreter: Für virtuelle Quell-CPU und die Speicherverwaltung verantwortlich
	\end{itemize}
	\item \textbf{Techniken}
	\begin{itemize}
		\item Interpretation
		\begin{itemize}
			\item Decode-Dispatch-Interpretation: Interpreter betrachter das Quell-Programm schrittweise in einer zentralen Schleife; dekodiert die Instruktionen und ruft die Dispatch-Routine auf. Diese liest/modifiziert den Gast-Zustand
			\item Optimierungsmöglichkeiten des Interpreters: Ausführen einer einzigen Quellinstruktion benötigt ein Vielfaches "`multiple tens"' an Instruktionen auf dem Host \(\rightarrow\) extrem ineffizient
			\begin{enumerate} % TODO
				\item Threaded Interpretation: Ersetzt die Hauptschleife. Kopiere Dekoderlogik ans Ende von jeder Dispatch-Routine; verwende Sprungtabellen statt switch-case-Statements \(\rightarrow\) vermeidet teure Branches
				\item Predecoding: Aufbereitung der Quellinstruktionen (Opcode, Operanden, etc.) für einfachere Zugriffe auf die Instruktionen zur Laufzeit \(\rightarrow\) vereinfacht die Dekodierroutine
				\item Direct Threaded Interpretation (Kombination aus Erstgenannten): Speichere Adresse der Dispatch-Routine statt Opcode \(\rightarrow\) benötigt keine Sprungroutine und vermeidet Indirektionen 
			\end{enumerate}
		\end{itemize}
		\item Dynamic Binary Translation (BDT)
		\begin{itemize}
			\item Benötigt keine Dispatch-Routinen; erlaubt Optimierungen der Ziel-Instruktionen
			\item Quellinstruktionen werden direkt in Zielinstruktionen umgewandelt
			\item Probleme bei Binary Translation
			\begin{itemize}
				\item Code Discovery Problem: Finden des Einsprungpunkts bei Quell-Instruktionen ggf. schwierig (Instruktionen und Opcodes unterschiedlich lange; u.U. Daten in \textit{Instruction Stream} eingebettet; Padding; indirekte Sprünge \(\rightarrow\) Sprungziel erst zur Laufzeit bekannt)
				\item Code Location Problem: Fehlerhafte SIP-TIP-Beziehung (Source/Target Instruction Pointer). Kann beispielsweise entstehen, wenn eine Quell-Instruktion auf mehrere Zielinstruktionen abgebildet wird
			\end{itemize}
			\item Lösung: Dynamic Translation
			\begin{itemize}
				\item Zunächst Interpretation mit Code Discovery
				\item Der Code wird schrittweise übersetzt und blockweise im Code-Cache gespeichert. Zusätzlich wird das IP-Mapping im \textit{SIP-To-TIP}-Cache gespeichert
				\item Problem dabei: Wie wird der Folgeblock identifiziert? \(\rightarrow\) speichere zusätzlich den SIP zur Folgeinstruktion. Dieser kann im SIP-to-TIP-Cache nachgeschlagen werden
				\item Optimierung: \textit{Translation Block Chaining}
				\begin{itemize}
					\item Vermeidet Rücksprünge zum Emulation Manager; der nächste Block wird direkt im Block davor gespeichert. Bei Sprüngen werden verschiedene mögliche Ziele gespeichert
					\item Unconditional Jump: Direkter Sprung zum nächsten Block
					\item Conditional Jump: Jeder Folgeblock wird separat gelinkt
					\item Indirect Jump: Immer zurück zum Emulation Manager
				\end{itemize}
			\end{itemize}
		\end{itemize}
	\end{itemize}
	\item \textbf{Code-Cache-Management}
	\begin{itemize}
		\item Herausforderungen: Variabel lange Blöcke; Abhängigkeiten zwischen Blöcken; Blöcke müssen bei Verdrängung aus dem Cache neu generiert werden \(\rightarrow\) unterscheiden sich von typischen Hardware-Caches wie CPU-Caches oder TLBs
		\item Ersetzungsstrategien
		\begin{itemize}
			\item Least Recently Used (LRU): Probleme durch Verwaltungsoverhead (Nutzungsverhalten der Blöcke); bei Verdrängung Aktualisierung der Verknüpfungen mit anderen Blöcken notwendig (Speicherung von Back-Pointern); variabel große Blöcke führen zu Cache-Fragmentierung
			\item Flush When Full (FwF)
			\begin{itemize}
				\item Vorteile: Einfach; Berücksichtigt Dependencies und Fragmentierung
				\item Nachteil: Regenerierung der gelöschten Blöcke teuer
				\item Defaulteinstellung bei \texttt{QEMU} mit 16 MB Cache
			\end{itemize}
			\item Preemptive Flush
			\begin{itemize}
				\item Viele Programme weisen Phasenverhalten auf (Initialisierung, schrittweise Berechnungen, etc.) \(\rightarrow\) Instruction Working Set wechselt
				\item Idee: Flush, wenn Phasenwechsel erkannt (anhand Burst neuer Instruktionen)
			\end{itemize}
			\item First In First Out (FIFO)
			\begin{itemize}
				\item Der/die älteste(n) Blöcke werden entfernt
				\item Vorteile: Berücksichtigt zeitliche Lokalität; kein Verwaltungsoverhead zum Nutzungsverhalten der Blöcke (siehe LRU)
				\item Nachteil: Back-Pointer notwendig
				\item Variante Coarse-Grained FIFO: Code-Cache wird partitioniert, es werden immer ganze Partitionen gelöscht; keine Verkettung zwischen Partitionen
			\end{itemize}
		\end{itemize}
	\end{itemize}
	\item \textbf{Gastarbeitsspeicherverwaltung}
	\begin{itemize}
		\item Gast erhält einen Teil des Adressbereichs des Hosts; zuverlässige Isolierung notwendig (Gast darf nicht auf den Speicher, den der Host für sich nutzt, zugreifen können)
		\item Runtime verwaltet Zuordnungstabelle
	\end{itemize}
	\item \textbf{Ausnahmebehandlung}
	\begin{itemize}
		\item Direkt: Interpreter erkennt Exceptions und springt zu einem \textit{Runtime Trap Handler}
		\item Indirekt: Host-ISA erkennt die Exception. Signal wird durch das Host-OS an den Runtime Handler weitergegeben (beispielsweise Division durch \texttt{0})
	\end{itemize}
	\item \textbf{\texttt{SYSCALL}-Behandlung}
	\begin{itemize}
		\item Wrapper übersetzt Quell-System-Call zu Ziel-System-Call; eventuell muss zwischen Calling Conventions übersetzt werden
		\item Manche Calls können vom Gastsystem selbst behandelt werden (beispielsweise Signal Handler Registrierung, oder Speichermanagement)
	\end{itemize}
\end{itemize}



\section{System VMs}
\begin{itemize}
	\item Ziel: Unterstützung mehrerer Gast-Betriebsssteme auf einem System
	\item Virtual Machine Monitor (VMM): "`Besitzer"' der Hardware-Ressourcen; läuft im privilegierten Modus (Gäste im Gastmodus); verwaltet Allokation/Zugriff auf die Ressourcen
	\item VMM wechselt zur Ausführung der Gäste zwischen diesen (vergleichbar mit Timesharing zwischen Anwendungen eines Betriebssystems). Gäste dürfen dazu weder selbst Timer-Interrupts setzen noch die echten Timer-Interrupt-Werte lesen (wird vom VMM emuliert), da der VMM den Host-Timer zum Wechseln zwischen VMs und dem Host verwendet wird (Probleme durch Manipulation: "`unfaire"' Verteilung von Rechenzeit, DoS-Angriffe)
	\item \textbf{State Management}
	\begin{itemize}
		\item Alle Gästen benötigen individuelle (exklusive oder versteckte) Hardware-Ressourcen wie beispielsweise Festplatten (können rein virtuell sein); alle nicht-exklusiven Ressourcen, die virtualisiert werden, müssen entsprechend verwaltet werden (Partitionierung oder Sharing)
		\item Wechsel zwischen den States mehrerer VMs
		\begin{itemize}
			\item Indirection: Zustand der Gäste wird im RAM des VMM vorgehalten; Pointer wechselt zwischen den Gästen. Beispielsweise gut geeignet für Page Directory; ungeeignet für Registersätze
			\item Copying: Zustand der Gäste wird jeweils aus dem RAM des VMM in den Register Block des Prozessors kopiert. Beispielsweise gut geeignet für Registersätze; ungeeignet für Page Directory
		\end{itemize}
	\end{itemize}
	\item \textbf{Virtualisierung von CPUs}
	\begin{itemize}
		\item Generelle Techniken
		\begin{itemize}
			\item Full Emulation: Immer möglich; ISA des Gasts muss nicht ISA des Hosts entsprechen; langsam
			\item Direct Native Execution: Nahezu native Ausführungsgeschwindigkeit; ISA des Gasts \textit{muss} der des Hosts entsprechen
		\end{itemize}
		\item Direct Native Execution
		\begin{itemize}
			\item \textit{Trap'n'emulate}: Gastinstruktionen werden direkt ausgeführt; "`Trap"' bei privilegierten Instruktionen, diese werden vom Dispatcher emuliert. Letzterer ruft die entsprechende \textit{Interpreterroutine} auf. Beispiele für pivilegierte Instruktionen: \texttt{IN}, \texttt{OUT}, \texttt{Write TLB}
			\item Aufgabenverteilung
			\begin{itemize}
				\item Dispatcher: Übergeordnete Komponente; entscheidet welche Komponente aufgerufen werden soll
				\item Allocator: Verwaltet Ressourcen, die sich die VMs teilen
				\item Interpreter Routines: Existiert pro privilegierter Instruktion; emuliert den Befehl auf virtuellen Ressourcen
			\end{itemize}
			\item Gastbetriebssystem: Da nur der VMM im Kernelmodus läuft "`fallen"' (trappen) alle privilegierten Instruktionen in den VMM Dispatcher
			\item Gastanwendungen: Da \texttt{SYSCALL} auf \texttt{x86}-Systemen keine privilegierte Instruktion ist muss der VMM diese speziell abfangen, damit sie korrekt emuliert werden kann
			\item Weitere wichtige Instruktionen: Instruction Types
			\begin{itemize}
				\item Definitionen
				\begin{itemize}
					\item Control sensitive instructions: Alle Instruktionen, welche die Ressourcenkonfiguration verändern. Beispiel: Page Directory Pointer (\texttt{CR3}) auf \texttt{x86} aktualisieren
					\item Behavior sensitive instructions: Alle Instruktionen, deren Verhalten/Ergebnis konfigurationsabhängig ist. Beispiel: Pop flags register (\texttt{POPF}) auf \texttt{x86}
					\item Privileged instructions: Alle Instruktionen, die trappen (müssen), wenn sie im User Mode ausgeführt werden. Beispiel: Load processor status word (\texttt{LPSW}) auf \texttt{System/370}. Vorsicht: Lehrstuhlspezifische Definition; schließt \texttt{SYSCALL} hier mit ein
					\item Innocuous instructions ("`harmlose"' Instruktionen): Weder control- noch behavior-sensitiv
				\end{itemize}
				\item \textit{Efficient Virtual Machines Theorem}: Alle sensitiven Instruktionen, die eine Teilmenge der privilegierten Instruktionen bilden, können effizient ausgeführt werden
				\item Critical Instructions
				\begin{itemize}
					\item Sensitiv aber nicht priviligiert. Bei \texttt{x86} gibt es beispielsweise 17
					\item Lösung: Kritische Instruktionen werden gepatcht. Dazu scannt der VMM den Code des Gasts vor der Ausführung und wandelt alle kritischen Instruktionen in Traps um
					\item Umsetzung mittels \textit{Dynamic-binary-translation}-Techniken: \textit{Instruction stream} wird in Blöcke aufgteilt und gepatcht. Am Ende des Blocks wird eine zusätzliche Trap eingefügt, damit der VMM die Kontrolle zurückerlangt
					\item Generelles Problem mit Traps: Sehr teuer
				\end{itemize}
			\end{itemize}
		\end{itemize}
	\end{itemize}
	\item \textbf{Virtualisierung von Arbeitsspeicher}
	\begin{itemize}
		\item Betriebssysteme gehen davon aus, die vollständige Kontrolle über den Speicher zu haben. Dies beinhaltet die Verwaltung und die Fähigkeit, jede virtuelle Page auf jeden physikalischen Frame abbilden zu können
		\item VMM muss allerdings die Kontrolle über den Speicher behalten, partitioniert den Adresseraum und übernimmt das Mapping [Page \(\rightarrow\) Frame]. TLBs verstärken das Problem, da diese bei einem Miss automatisch Frames nachschlagen
		\item Abstraktionsebenen
		\begin{itemize}
			\item Physikalischer Speicher des Hosts: "`Realer"' Arbeitsspeicher. Kann von VMM mehrfach vergeben werden
			\item Physikalischer Speicher des Gasts: Speicherabstraktion, die der Host dem Gast zur Verfügung stellt. Wird vom Gast als zusammenhängender, physikalischer Speicher wahrgenommen
			\item Virtueller Speicher des Gasts: Standard \(2^{32/64}\)-virtueller Adressraum
		\end{itemize}
		\item Page-Table-Typen
		\begin{itemize}
			\item Guest virtual page table: Einmal pro Gastprozess vorhanden; [Guest virtual \(\rightarrow\) Guest physical]; vom Gastsystem verwaltet; im RAM des Gasts gespeichert
			\item Guest physical page table: Einmal pro Gast vorhanden; [Guest physical \(\rightarrow\) Host physical]; vom VMM verwaltet; im RAM des Hosts gespeichert
			\item Shadow page table: Einmal pro Gastprozess; [Guest virtual \(\rightarrow\) Host physical]; vom VMM verwaltet; im RAM des VMM gespeichert. Problem dabei: VMM muss darauf achten, dass shadow page tables und guest physical page table konsistent bleiben. Dazu mappt der VMM OS page tables read-only in guest physical page tables
		\end{itemize}
		\item Speicherallokation
		\begin{itemize}
			\item Meist triviale Allokationsstrategien mit statischer Zuweisung ohne Swapping
			\item Mehrfachvergabe mit \textit{Balloon driver}: Nicht genutzter Speicher einer VM kann anderen VMs durch den VMM zur Verfügung gestellt werden
			\item Mehrfachvergabe mit Speicherdeduplitzierung: Gleiche Frames werden per \texttt{Copy-on-write} zusammengefasst
		\end{itemize}
	\end{itemize}
	\item \textbf{Virtualisierung von Input/Output}
	\begin{itemize}
		\item Schwierig zur virtualisieren, da viele verschiedenen Typen/Geräte mit unterschiedlichen Treibern vorhanden. Darüber hinaus müssen neue Geräte kompatibel sein
		\item Vorgehen: Erstellen einer virtuellen Version des Gerätes sowie Virtualisieren der I/O-Aktivität des Gerätes
		\item Gerätetypen
		\begin{itemize}
			\item Dedicated: Geräte müssen nicht virtualisiert werden, werden allerdings trotzdem vom VMM kontrolliert, da sie im privilegierten Modus laufen (beispielsweise Maus/Monitor/Tastertur)
			\item Partitioned: Es werden mehrere, kleinere, virtualisierte Versionen angelegt; VMM übersetzt zwischen virtualiserten und physikalischem Gerät (beispielsweise Festplatten)
			\item Shared: VMM verwaltet den virtuellen Status und übersetzt [virtuelle Anfrage \(\leftrightarrow\) physikalische Anfrage] (beispielsweise Netzwerkadapter)
			\item Spooled: Grobkörnig geshared (beispielsweise Drucker)
			\item Non-existent: Nur die virtuellen Versionen werden implementiert \(\rightarrow\) VMs auf dem selben Host können kommunizieren (beispielsweise dedizierte, virtuelle Netzwerkgeräte)
		\end{itemize}
		\item Reguläre Geräteanfragen
		\begin{enumerate}
			\item Betriebssystem verwaltet Geräteressourcen (Speicherallokation, serialisieren von Anfragen, etc.)
			\item Anwendungen führen Standard-I/O-Anfragen mit \texttt{SYSCALL} durch
			\item Betriebssystem konvertiert I/O-Anfragen zu Treiberanfragen
			\item Treiber erzeugt gerätespezifische I/O-Operationen
		\end{enumerate}
		\item \texttt{System-Call}-Level
		\begin{itemize}
			\item Gast trappt in VMM, dieser interpretiert den \texttt{SYSCALL} und erzeugt Treiberanfragen \(\rightarrow\) komplette I/O-Behandlung im VMM
			\item Problem: VMM muss alle I/O-\texttt{SYSCALL} aller VMs behandlen
		\end{itemize}
		\item Treiberschnittstelle (paravirtualisiert)
		\begin{itemize}
			\item Gast-Betriebssystem verfügt über Treiber-Stubs; generische I/O-Operationen werden an den VMM übergeben und von diesem in reale Treiberanfragen umgewandelt
			\item Probleme: VMM muss Zugang zu den realen Treibern sowie einen generischen Treiber für alle Gastbetriebssysteme haben
			\item Beispiel: \texttt{XEN}
		\end{itemize}
	\end{itemize}
\end{itemize}
