\section{Appendix A: Spickzettel}

\subsection{Flussnetzwerke}
Der maximale Fluss ist
\begin{itemize}
	\item die Summe der von \(q\) ausgehenden Kanten,
	\item bei \(s\) ankommenden genutzten Kapazitäten oder
	\item der minimale Schnitt durch das Netzwerk.
\end{itemize}


\subsection{de Casteljau}
\[\frac{\Delta B_0}{t} \frac{\Delta B_1}{t-1} = C(\Delta B,t)\]


\subsection{Unterteilungsalgorithmen}
Das Differenzschema existiert nur, wenn \(\alpha(z)\) den Faktor \(1+z\) besitzt oder
\[\alpha(-1) = \sum_{i \in \mathbb{Z}} \alpha_{2i} - \sum_{i \in \mathbb{Z}}\alpha_{2i + 1} = 0\]
gilt.
	
\subsubsection{Symbol}
\[\beta(z) = \frac{\alpha(z)}{1+z}\]
\\\\
Existiert das zweite Differenzschema, so gilt \(\beta(-1)=0\).
