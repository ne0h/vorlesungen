\section{Appendix A: Spickzettel}

\subsection{Flussnetzwerke}
Der maximale Fluss ist
\begin{itemize}
	\item die Summe der von \(q\) ausgehenden Kanten,
	\item bei \(s\) ankommenden genutzten Kapazitäten oder
	\item der minimale Schnitt durch das Netzwerk.
\end{itemize}


\subsection{Stochastische Algorithmen}

\subsubsection{Algorithmen}
\begin{itemize}
	\item Las Vegas Algorithmen: Falls der Algorithmus terminiert, ist das Ergbenis korrekt (nichtdeterministisch)
	\item Monte Carlo Algorithmus: Der Algorithmus terminiert immer, allerdings ist das Ergebnis mit einer bestimmten Wahrscheinlichkeit falsch
\end{itemize}

\subsubsection{Erwartungswert}
\begin{itemize}
	\item Die Zufallsvariable \(X_i\) nimmt mit einer Wahrscheinlichkeit \(p\) einen bestimmten Wert an
	\item \(\mathbb{E}\lbrack \sum X_i \rbrack = \sum \mathbb{E} \lbrack X_i \rbrack\)
\end{itemize}


\subsection{de Casteljau}
\[\frac{\Delta B_0}{t} \frac{\Delta B_1}{t-1} = C(\Delta B,t)\]


\subsection{Unterteilungsalgorithmen}
Das Differenzschema existiert nur, wenn \(\alpha(z)\) den Faktor \(1+z\) besitzt oder
\[\alpha(-1) = \sum_{i \in \mathbb{Z}} \alpha_{2i} - \sum_{i \in \mathbb{Z}}\alpha_{2i + 1} = 0\]
gilt.
	
\subsubsection{Symbol}
\[\beta(z) = \frac{\alpha(z)}{1+z}\]
\\\\
Existiert das zweite Differenzschema, so gilt \(\beta(-1)=0\).

\subsection{Minimax Theorem}
Sei $M=\begin{pmatrix}
5 & 6 \\
7 & 4 
\end{pmatrix} $ dann ist $ min_y =  \begin{pmatrix} 
x & 1-x 
\end{pmatrix}
\begin{pmatrix}
5 & 6 \\
7 & 4 
\end{pmatrix} = 0 $ Nullsetzen.\newline $max_x = 
\begin{pmatrix}
5 & 6 \\
7 & 4 
\end{pmatrix}
\begin{pmatrix}
y \\
1-y 
\end{pmatrix} = 0 $ ebenfalls Nullsetzen.
Die Nullstellen entsprechen dem Ergebnis. \newline Also ist $f(x,y)=(max_x,min_y)$ bzw. $(\begin{pmatrix}
max_x \\
1 - max_x
\end{pmatrix}
,
\begin{pmatrix}
min_y \\
1 - min_y
\end{pmatrix} )$ und der Wert ergibt sich durch Einsetzen in $ max_x \cdot min_y \cdot M=f(x,y) = 
\begin{pmatrix} 
x & 1-x 
\end{pmatrix}
\begin{pmatrix}
5 & 6 \\
7 & 4 
\end{pmatrix}
\begin{pmatrix}
y \\
1-y 
\end{pmatrix}$ und von rechts nach links ausrechnen.

TODO: Türsuche, Sternsuche

