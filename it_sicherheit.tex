\chapter{IT-Sicherheit für vernetzte Systeme}
Zusammenfassung der Vorlesung  "`IT-Sicherheit für vernetzte Systeme"' von Professor Hartenstein aus dem Wintersemester 2013\footnote{\url{http://dsn.tm.kit.edu/pastcourses_3385.php}}.

\section{Einführung}

\subsection{Management}

\subsubsection{Handlungsorientiertes Managementkonzept}
Prozesse der Entscheidungsfindung und Entscheidungsdurchsetzung.
\begin{itemize}
	\item Gesamtheit der Handlungen
	\item Bestmögliches Erreichen der Ziele einer Institution
	\item Auf die beteiligten Interessensgruppen gerichtet
\end{itemize}
\textbf{Strukturierung in Aufgabenbereiche:}
\begin{itemize}
	\item Grundsatz- und Zielbildung
	\item Planung
	\item Organisation
	\item Kontrolle
\end{itemize}

\subsubsection{Personenorientiertes Managementkonzept}
Fokus liegt auf den Aufgaben sowie den Befugnissen und Aufsicht/Kontrolle der handelnden Personen.\\
Die Personengruppen sind durch Gesetze, Satzungen oder Aufträgen mit Rechten, Pflichten und Verantwortung ausgestattet.

\subsubsection{Handhabungssorientiertes Managementkonzept}
\begin{itemize}
	\item Management-Techniken und Methoden
	\item Qualitativer und quantitativer Art
	\item Prognose-, Planungs- und Entscheidungstechniken
	\item Einbeziehung/basis von Managementinformationssystemen
\end{itemize}


\subsection{IT-Sicherheitsmanagement}
\begin{itemize}
	\item Allumfassender Ansatz
	\item Ziel: Agemessenes Sicherheitsniveau
	\item Verlangt: Abschätzung des Risikos
	\item Immerwährender Prozess: Planung/Organisation - Umsetzung/Durchsetzung - Aufrechterhaltung und Erfolgskontrolle
\end{itemize}

\subsubsection{Informationssicherheit}
\begin{itemize}
	\item Hauptaspekte: Vertraulichkeit, Verfügbarkeit und Integrität zum Erhalt eines kontinuierlichen Geschäftsbetriebs
	\footnote{\url{http://standards.iso.org/ittf/PubliclyAvailableStandards/c041933_ISO_IEC_27000_2009.zip}}
	\item Umsetzung eines geeigneten Maßnahmenkatalogs
	\item Festgelegter Risikomanegementprozess
	\item Umsetzung mit Hilfe eines ISMS
	\item Die Maßnahmen müssen festgelegt, umgesetzt, überwacht, überprüft und verbessert werden
	\item Ziel: Erreichen des spezifischen Unternehmenssicherheitsniveaus
\end{itemize}


\subsection{IT-Sicherheitsmanagement: Prinzipien}

\subsubsection{Prinzipien: Die 6 Ps}
Siehe handlungsorientiertes Management.
\begin{itemize}
	\item Richtlinien (Policy)
	\item Planung (Planning)
	\item Programme (Programs)
	\item Schutzmaßnahmen (Protection)
	\item Menschen (People)
	\item Projektmanegement
\end{itemize}

\subsection{Strukturierung der Vorlesung}
\begin{itemize}
	\item Konzepte
	\item Standards
	\item Vorgehensweisen
	\item Fallstudien und -beispiele
\end{itemize}



\section{Bedrohungsszenarien und IT-Schutzziele}

\subsection{Potentielle Bedrohungen und Angreifer}

\subsubsection{Klassifizierung und Angreifer}
\begin{itemize}
	\item Zielsetzung (Wann ist der Angriff ein Erfolg?)
	\item Fokussierung (Muss ein bestimmtes Ziel angegriffen werden?)
	\item Ressourcen (Welche Ressourcen stehen dem Angreifer zur Verfügung?)
	\item Motivationsstärke (Wie wichtig ist dem Angreifer das Erreichen der Zielsetzung?)
\end{itemize}

\subsubsection{Typische Zielsetzungen eines Angreifers}
\begin{itemize}
	\item Zugriff auf Ressourcen
	\item Zugriff auf Informationen
	\item Manipulation eines Dienstes
	\item Sabotage
\end{itemize}

\subsubsection{Ressourcen des Angreifers}
\begin{itemize}
	\item Personelle Ressourcen
	\item Finanzielle Ressourcen
	\item Technische Ressourcen
	\item Insiderwissen
	\item Einfluss auf das Ziel
\end{itemize}

\subsubsection{Motivationsstärke des Angreifers}
\begin{itemize}
	\item Persönliche, nicht-finanzielle Interessen: Ansehen, politische Überzeugung, Interesse. Sicherheitsmmaßnahmen so weit erhöhen, dass der Angriff aufwendig wird.
	\item Finanzieller Gewinn: Rentabilität. Sicherheit so weit erhöhen, dass der Angriff unrentabel wird.
	\item Politisches oder strategisches Ziel: Sicherheit muss erhäht werden, dass Angriff unmöglich ist (praktisch kaum zu bewältigen).
\end{itemize}

Angriffe können gezielt oder ungezielt sein. Für einen gezielten Angriff ist ein höheres Sicherheitsniveau erforderlich. (Anderenfalls muss man nur besser sein als die anderen.)

\subsubsection{Beispiele von Angreifern}
\begin{itemize}
	\item Staatliche Institutionen: Informationen oder Sabotage
	\item Gezielte Hacker-Angriffe: Manipulation von Diensten. Bsp: Drogenschmuggel, Ausspähen von Häfen.
	\item Ungezielte Hacker-Angriffe: Meist Zugriff auf Informationen oder Ressourcen
	\item Gelegenheitsangriffe
\end{itemize}


\subsection{Schutzziele der IT-Sicherheit}

\subsubsection{IT-Sicherheit / Informationssicherheit}
Schützen von Informationen bzw. Daten einschließlich zugehöriger Komponenten wie etwa der Systeme und Hardware, die diese Information nutzt, speichert oder überträgt.

Differenzierung zur allgemeinen Sicherheit erfolgt über die sogenannten Schutzziele.

\subsubsection{Schutzziele}

Die klassischen Triaden: Confidentiality, Integrity, Availability (CIA).

\begin{itemize}
	\item Authentizität: Echtheit und Glaubwürdigkeit eines Objekts (Übergeordnetes Schutzziel)
	\item Vertraulichkeit: Keine unautorisierte Informationsgewinnung.
	\item Integrität: Keine Möglichkeit, Daten unautorisiert zu manipulieren.
	\item Verfügbarkeit (Abgrenzung gegenüber Robustheit).
	\item Datenschutz
	\item Anonymität und Pseudonymität
	\item Verbindlichkeit
\end{itemize}



\section{Der IT-Sicherheitsprozess}

\subsection{Was ist ein IT-Sicherheitsprozess?}

Ergänzung der Schutzziele durch Richtlinien und Gesetze, Sicherheitsbewusstsein, Schulungen, Planung und Organisation.

\begin{itemize}
	\item Was ist ein angemessenes Sicherheitsniveau?
	\item Welche Ressourcen müssen geschützt werden?
	\item Wie werden die Ressourcen geschützt?
\end{itemize}

\subsubsection{Umsetzung von IT-Sicherheit}
\begin{itemize}
	\item Zugangs- und Zugriffskontrolle
	\item Kryptokonzept und Montoring
	\item Redundanz, Backup und Archivierung
	\item Patches und Upgrades
	\item Firewalls, Intrusion Detection und Intrusion Prevention
\end{itemize}

\subsubsection{Klassisches Prozessmodell: Wasserfall-Modell}
\begin{enumerate}
	\item Voruntersuchung
	\item Analyse
	\item Grobentwurf
	\item Feinentwurf
	\item Realisierung
	\item Betrieb und Wartung
\end{enumerate}
Jeweils nur ein Sprung zum Vorgänger bzw. Nachfolger möglich.

\subsubsection{Inkrementelles Modell}
\begin{itemize}
	\item Voruntersuchung/Analyse
	\item Definition
	\item Entwurf
	\item Realisierung
	\item Betrieb
\end{itemize}
Inkrementelles Modell, daher nach \textit{Betrieb} ein Sprung zu \textit{Definition}.

\subsubsection{Spirallmodell}
Die Spirale besteht aus den folgenden Teilschritten, die jeweils immer wieder abgearbeitet werden und weiter nach Außen gehen.
\begin{itemize}
	\item Analyse, Identifikation der Ziele
	\item Evaluierung der Alternativen, Risikoanalyse
	\item Entwicklung und Verifikation (Wasserfallmodell, inkr. Modell, etc.)
	\item Planung der nächsten Schritte, Überprüfung
\end{itemize}

\subsection{Internationale Umsetzung: ISO 2700x}

\begin{itemize}
	\item Ziel: Normierung des IT-Sicherheitsprozesses, Katalog von "`Best Practise"' - Vorschlägen, Detaillierte Behandlung ausgewählter Problemfelder (z.B. Risikoanalyse), Grundlage für Zertifizierung
	\item Aber: Kein Anspruch auf Vollständigkeit!
\end{itemize}

\subsubsection{ISO 27000: Informationssicherheits-Managementsystem (ISMS)}
\begin{itemize}
	\item Ganzheitlicher Prozess
	\item Risikoanalyse
	\item Risikoakzeptanz (Risiko vs. Kosten)
	\item Risikomanagement (aktive Prävention und Neubewertung)
\end{itemize}

\subsubsection{ISO 27001: Prozessmodell}
\begin{itemize}
	\item Planen: Definition des Anwendungsbereichs und der Sicherheitsleitlinie, Organisation der Risikoanalyse, Auswahl von Maßnahmen, Koordination mit Management
	\item Durchführen (Do): Risikobehandlung und Umsetzung der Maßnahmen, Schulungen und Bewusstseinsbindung
	\item Prüfen (Check); Überwachung der Maßnahmen, Prüfen der Wirksamkeit, Überprüfung der Risikoeinschätzung, Interne Befragungen (Audits)
	\item Handeln (Act): Umsetzung identifizierter Verbesserungen, Verbeugung neuer Risiken, Sicherstellen, dass die Verbesserungen ihre Ziele erreichen
\end{itemize}

\subsubsection{Verantwortung des Managements}
\begin{itemize}
	\item Zielbildung: Festlegung der Sicherheitsleitlinie und des Sicherheitsniveaus
	\item Organisation: Bestimmung von Verantwortlichkeiten, Bereitstellung der Ressourcen
	\item Schlung und Bewusstsein
	\item Kontrolle: Regelmäßige Überprüfung, Analyse und Bewertung
\end{itemize}

\subsubsection{ISO 27002: Sicherheitsleitlinie}
Verbindliche Grundsätze für die IT-Sicherheit.
\begin{itemize}
	\item Bedeutung: Bedeutung der Sicherheit definieren, Absicht des Managements zu erklären
	\item Zielsetzung
	\item Strategie: Rahmen für Risikoeinschätzung und Risikimanegement, Berücksichtigung gesetzlicher und vertraglicher Anforderungen
	\item Organisation: Verantwortlichkeiten festlegen, Verweise auf weiterführende Dokumente
\end{itemize}

\subsubsection{Aufbau und Struktur am Beispiel des KIT}
\begin{itemize}
	\item Präambel
	\item Geltungsbereich
	\item Zielsetzung
	\item Organisation
	\item IT-Sicherheitsprozess am KIT
	\item Sachverwandte Themen
\end{itemize}
Knapp formuliert (5 Seiten).

\subsubsection{Maßnahmen zur behandlung identifizierter Risiken}
\begin{itemize}
	\item Bahandlung mittels Maßnahme (Risikoreduktion)
	\item Tolierierung des Risikos (Restrisiko)
	\item Vermeidung/Eleminierung spezieller Risiken
	\item Übertragung des Risikos auf Dritte (z.B. Versicherung)
\end{itemize}


\subsection{Deutsche Umsetzung: BSI-Grundschutz}
\begin{itemize}
	\item Empfehlungen von Standardsicherheitsmaßnahmen für typische IT-Systeme
	\item Organisatorische, personelle, infrastrukturelle und technische Standard-Sicherheitsmaßnahmen
	\item Baukastenprinzip
	\item Werkzeugunterstützung
	\item Bedarfsgerechte Weiterentwicklung (...aber leider etwas zu langsam)
\end{itemize}

\subsubsection{IT-SIcherheitsprozess nach BSI GSK}
\begin{itemize}
	\item Strategische Ebene: Initiierung, Erstellung der Sicherheitsleitlinie, Einrichtung des IT-Sicherheitsmanagements
	\item Taktische Ebene: Erstellung eines IT-Sicherheitskonzepts
	\item Operative Ebene: Umsetzung und Aufrechterhaltung im laufenden Betrieb
\end{itemize}

\subsubsection{Entwicklung eines IT Sicherheitskonzeots nach GSK}
\begin{enumerate}
	\item IT-Strukturanalyse: Ermittlung des zu schützenden IT-Verbungs
	\item Schutzbedarfsfeststellung
	\item IT-Grundschutzanalyse: Modellierung über Baukastensystem und Basis-Sicherheitscheck
	\item Ergänzende Sicherheitsanalyse (bei hohem Schutzbedarf): Risikoanalyse, Penetrationstest, Differenz-Sicherheitsanalyse (etablierte Musterlösungen hochschutzbedürftiger Systeme)
	\item Realisierungsplanung
\end{enumerate}

\subsubsection{Schutzbedarfsfeststellung}
\begin{itemize}
	\item Klassisches Vorgehen: Ermittlung der bedrohten Objekte und Bedrohungen, dann Risikoanalyse
	\item GSK vereinfacht: Pauschale Schutzbedarfsfeststellung über die Schadensauswirkungen
	\item Schutzbedarfsfeststellung für \textit{jedes} Schutzziel separat.
	\item Mögliche Prinzipien: Maximumsrinzip (Schutzbedarf orientiert sich am höchsten Schadenspotential), Kumulationseffekt (Schutzbedarf des Systems ist höher als Schutzbedarf aller Anwendungungen kumuliert), Verteilungseffekt (Invers zu Kumulationseffekt)
\end{itemize}

\subsubsection{Gefährdungskatalog}
\begin{itemize}
	\item Elementare Gefährdungen
	\item Höhere Gewalt
	\item Organisatorische Mängel
	\item Menschliche Fehlhandlung
	\item Technisches Versagen
	\item Versätzliche Handlungen
\end{itemize}

\subsubsection{Vergleich zu ISO 2700x}
\begin{itemize}
	\item BSI GSK konkretisiert das recht allgemeine ISO 2700x-Paket
	\item Vorteile: Detaillierter, Berücksichtungeg deutscher Besonderheiten (Rechtslage, etc.)
	\item Nachteile: Grundschutz recht starr, BSI Maßnahmen oft zu konkret und nicht aktuell
	\item Fazit: Nut für mittelgroße IT-Verbunde geeignet, sonst "`Overkill"'
\end{itemize}



\section{Rechtliche Anforderungen}

\subsection{Motivation}

\subsubsection{Fragen aus dem Betrieb}
\begin{itemize}
	\item Welche Gesetze sind einschlägig?
	\item Welche (Log-)Daten darf ich speichern und wie lange?
	\item Welche Schutzmaßnahmen sind vorgeschrieben?
\end{itemize}

\subsubsection{Mögliche Gesetze}
\begin{itemize}
	\item Bundesdatenschutzgesetz
	\item Telemediengesetz und Telekommunikationsgesetz
	\item Kreditwesengesetz, HGB und KonTraG
	\item Signaturgesetz
\end{itemize}

\subsubsection{Herausforderungen bei der IT-Compliance}
\begin{itemize}
	\item Vollständigkeit nur schwer erreichbar
	\item Abwägung unterschiedlicher Anforderungen nötig
	\item Gewichtung und Angemessenheit muss gewahrt werden
\end{itemize}

\subsection{Abgrenzung von Regelungsbereichen}
Die Abgrenzung von Regelungsbereichen, d.h. die Wechselwirkung von verschiedenen Gesetzen ist zu beachten.\\
Beispiel: Welche Regelungsbereiche sind für den Bereich elektronischer Kommunikationsnetzen und -diensten einschlägig?\\
Abgrenzung/Zuordnung in der Praxis oft umstritten.\\

\subsubsection{Exemplarische Abgrenzung: Ebenen der Kommunikation}

\begin{tabularx}{\columnwidth}{|X|X|l}
	\hline
	\textbf{Inhaltsebene}: z.B. Abwicklung von Bankgeschäften, Versandhandel & \textbf{Offline-Recht}: Bundesdatenschutzgesetz, Bereichsspezifische Gesetze \\
	\hline
	\textbf{Interaktionsebene}: Teledienste und Mediendienste & \textbf{Online-Recht}: Telemediengesetz \\
	\hline
	\textbf{Transportebene}: Telekommunikationsdienste & \textbf{Telekommunikationsrecht}: Telekommunikationsgesetz \\
	\hline
\end{tabularx}

\subsubsection{Telekommunikationsgesetz}
\begin{itemize}
	\item Fernmeldegeheimnis
	\item Datenschutz
	\item TK-Überwachung und Voratsdatenspeicherung
	\item Nicht verantwortlich für den Inhalt
	\item Sicherheitskonzept notwendig, wenn TK-Anlagen betrieben werden
\end{itemize}

\subsubsection{Telemediengesetz}
\begin{itemize}
	\item Datenschutz
	\item Impressumspflicht
	\item Regelmäßig verantwortlich für den Inhalt
\end{itemize}


\subsection{Datenschutz}
Schutz von personenbezogener Daten\footnote{Bundesdatenschutzgesetz}. Datensicherheit ist meistens vorausgesetzt, bzw. notwendig.

\subsubsection{Unterscheidung zwischen 3 Phasen}
\begin{itemize}
	\item Erhebung
	\item Verarbeitung, Speicherung, Übermittlung
	\item Nutzung
\end{itemize}

\subsubsection{Grundsätze des BDSG}
\begin{itemize}
	\item Präventives Verbot mit Erlaubnisvorbehalt
	\item Zweckbindungsgebot
	\item Transparenzgebot: Auskunftspflicht und Berichtungsrecht
	\item Gebot Verhältnismäßigkeit, Datenvermeidung und Datensparsamkeit
	\item Technische und organisatorische Schutzmaßnahmen
\end{itemize}


\subsection{Aufsichtsrechtliche Regelungen für IT-Finanzdienstleister}

\subsubsection{MaRisk}
\begin{itemize}
	\item Abgeleitet aus dem KWG
	\item Mindestanforderungen an das Risikomanagement
	\item Herausgegeben von der BaFin
	\item Präventiv ausgerichtet mit risikoorientiertem Aufsichtsprozess
	\item Verwendung con internat. und nat. Normen (ISO 2700x, BSI GSK)
	\item Vergleichbare Maßnahmen und vereinfachte Prüfung
	\item Auswirkungen auf die Software-Entwicklung: Programmierrichtlinie, Testverfahren, Funktionstrennung von Entwicklung und Abnahme/Test, Dokumentation und Versionierung
	\item Umfangreiches Notfallmanagement
\end{itemize}


\subsection{Aktuelles: Geplantes IT-Sicherheitskonzept}

\subsubsection{Kernpunkten des Referentenentwurfs}
\begin{itemize}
	\item Pflicht zur Erfüllung von Mindeststandards
	\item Pflicht zur Meldung erheblicher Sicherheitsvorfälle
	\item Jährliche Berichtspflicht des BSI
\end{itemize}


\subsection{Zusammenfassung}
\begin{itemize}
	\item Zahlreiche Gesetze mit (in-)direktem Bezug zur IT-Sicherheit
	\item Datenschutz
	\item Regulierung im Finanzsektor durch KWG und MaRisk
	\item Weitere Regulierungen auf nat. und europäischer Ebene zu erwarten (z.B. IT-Sicherheitsgesetz)
\end{itemize}

\section{Risikoanalyse und -management}

\subsection{Motivation}
\begin{itemize}
	\item Risiken sind per se nichts Schlechts
	\item Risiken müssen allerdings konsequent gemanagt werden
	\item "`Geht ein Unternehmen kein Risiko ein, wird es sofort von offensiveren Wettbewerbern überholt"'\footnote{Bärentango: Mit Risikomanagement Projekte zum Erfolg führen von Tom DeMarco}
\end{itemize}


\subsection{Risikomanagement}
\[Risiko = Eintrittswahrscheinlichkeit\cdot Schadenspotential\]

\subsubsection{Methoden der Risikoanalyse}
\begin{itemize}
	\item Qualitative Einstufung
	\item Quantitative Einstufung
\end{itemize}

Risikomanagement bildet eine fundierte Basis für Entscheidungen und dient somit als Steuerinstrument, um Ressourcen effektiv zu nutzen.\\
Die Vorgehensweise orientiert sich an Plan-Do-Check-Act.

\subsection{Spezifika des IT-Risikomanegement}
\begin{itemize}
	\item Risikomanagement bildet einen wesentlichen Baustein im IT-Sicherheitsmanagement
	\item Integration in das Unternehmens Risikomanagement
	\item Definiert in ISO 27005
\end{itemize}

\subsubsection{Der IT-Sicherheitsrisikoprozess}
\begin{enumerate}
	\item \textbf{Festlegung der Rahmenbedingungen}
	\begin{itemize}
		\item Methodik festlegen (ISO 27005 gibt nur den Rahmen vor)
		\item Evaluierungskriteren entwickeln
		\item Potentielle Auwirkungs(-kategorien) festlegen
		\item Fokus und Grenzen festlegen
		\item Organisation im Unternehmen (Rollen zuweisen, etc.)
	\end{itemize}
	\item \textbf{Risikoidentifizierung}
	\begin{itemize}
		\item Identifikation von Szenarien, die zu einem Schaden führen können
		\item Identifikstion von Assets, Bedrohungen, umgesetzter Maßnahmen, Schwachstellen und potentielle Auwirkungen
	\end{itemize}
	\item \textbf{Risikoanalyse}
	\begin{itemize}
		\item Bestimmung des Schadenspotentials (z.B. durch statitische Daten)
		\item Bestimmung der Eintrittswahrscheinlichkeit
		\item Qualitative oder quantitative Bewertung möglich,
	\end{itemize}
	\item \textbf{Risikobewertung}
	\begin{itemize}
		\item Priorisierung der Risiken auf Basis der Evaluationskriterien und Akzeptanzkriterien
		\item Ist kompetenzgerecht vorzunehmen
	\end{itemize}
	\item \textbf{Risikobehandlung}
	\begin{itemize}
		\item Mögliche Methoden zur Risikobehandlung s.o.
		\item Ggf. sind Maßnahmen vorzuehmen
		\item Entscheidungen kompetenzgerecht
	\end{itemize}
\end{enumerate}

Iterativer Prozess, daher im Anschluss ein Sprung zu (1).

\subsubsection{Qualitative vs. quantitativer IT-Risikobewertung}
\begin{itemize}
	\item Quantitativ: Hinterlegung von konkreten Werten
	\item Qualitativ: Konkrete meist schwierig zu erlangen, daher Bewertung in Stufen (von Eintrittswahrscheinlichkeit und Schadenspotential $\rightarrow$ Matrix)
	\item ISO 27005 rät zu einer qualitativen Bewertung
\end{itemize}


\subsection{OCTAVE}

\subsubsection{Fokussierung auf zwei Aspekte}
\begin{enumerate}
	\item Operationelle Risiken
	\item Sicherheitsrisiken
\end{enumerate}

\textbf{OCTAVE ist eine Evaluierungsmethode und kein Prozess.}

\subsubsection{Die OCTAVE-Methoden}
\begin{enumerate}
	\item OCTAVE-Methode (2001): Erste und umfangreiche Methode. Aus der Organisationssicht und der Technologiesicht wird ein Strategieplan entwickelt.
	\item OCTACE-S: Analog zu (1), allerdings für kleine Organisationen (< 100 Mitarbeitern)
	\item OCTAVE-Allegro (2007): Weiterentwicklung von (1), Fokussierung auf Information Assets. Wesentliche Unterschiede sind die vereinfachte Nutzung, die eingeschränkte Technologiesicht und das Information-Asset-Konzept mit einer quantitativen Risikoanalyse.
\end{enumerate}

\subsubsection{OCTAVE Allegro: Prozess}
\begin{enumerate}
	\item Risikomanagementkriterien
	\item Entwicklung von Information-Asset-Profilen und Identifikation von Kontainern
	\item Identifikation von Bedrohungen
	\item Identifikation und Behandlung von Risiken
\end{enumerate}


\subsection{IT-Risikomanegement in der Praxis}

\subsubsection{IT-Risikomanagement-Methode der Fiducia}
\begin{itemize}
	\item 4-stufige, qualitative Risikobewertung (Kein - Standard - Erweitert - Hoch)
	\item Ableitung der Eintrittswahrscheinlichkeiten aus dieser Einstufung
	\item Risikomatrix mit 3 Risikoklassen (Akzeptabel - Meldepflichtig - Genehmigungspflichtig)
\end{itemize}

\subsubsection{Höhere Qualität bei der Entwicklung}
Trotz initial gesteigertem Aufwand,

\begin{itemize}
	\item effezienterer Ressourceneinsatz
	\item erhöhte Rebostheit und weniger Support
	\item gesteigerte Wiederverwendbarkeit
\end{itemize}


\subsection{Zusammenfassung}
\begin{itemize}
	\item Risikoorientiertes Vorgehen notwendig für effektives IT-Sicherheitsmanagement
	\item Risken bestimmen sich durch Eintrittswahrscheinlichkeit und Schadenspotential
	\item IT-Risiken werden i.d.R. qualitativ bewertet
	\item Die Einstufungen von Eintrittswahrscheinlichkeit und Schadenspotential in der Praxis eine Herausforderung
	\item Eintrittswahrscheinlichkeit kann aus der Schwachstelle und Bedrohung berechnet werden
\end{itemize}



\section{Krpytografische Verfahren}

\subsection{Kryptografische Verfahren}
Der Schutz sensibler Informationen in potentiell bösartigen Systemen ist nicht ohne weiteres gewährleistet.

\subsubsection{Kerkoff's Prinzip}
Die Sicherheit eines Verfahrens soll alleine auf der Geheimhaltung eines Schlüssels beruhen.

\subsection{Symmetrische Verschlüsselungsverfahren}
\begin{itemize}
	\item Algorithmen werden so gestaltet, dass bekannte Kryptoanalyseangriffe keine Bedrohung sind
	\item Sicherheit im allgemeinen nicht beweisbar
	\item Hohe Effiziens, da meist Bit-Shifts oder XORs
\end{itemize}

\subsubsection{Asymmetrische Verschlüsselungsverfahren}
\begin{itemize}
	\item Jeweils verschiedene Schlüssel zum Ver- und Entschlüsseln
	\item Basieren vorwiegend auf mathematischen Problemen
	\item Um Größenordnungen langsamer als symmetrische Verfahren, daher meist hybride Verfahren
\end{itemize}

\subsubsection{Message Authentication Codes}
\begin{itemize}
	\item Verwendung zum Detektieren von Manipulationen an Daten
	\item Basis: Vorher ausgetauschter, symmetrischer Schlüssel
	\item Kann nicht ohne Kenntnis des Schlüssels gefälscht werden
	\item Beispiel HMAC: Nutzen einer Hashfunktion und eines Schlüssels (Problem: Kollisionen der Hashfunktion)
	\item Nachteile: Komplexes Schlüsselmanagement, Verbindlichkeit nicht gegeben
\end{itemize}

\subsubsection{Digitale Signatur}
\begin{itemize}
	\item Digitale Signatur = "`elektronische Unterschrift"'
	\item Unterschriftverfahren zwar angreifbar, in der Praxis jedoch zumeist ausreichend (kaum zu erwartender Gewinn, Risiko einer Entdeckung hoch, harte Sanktionen und hoher zeitlicher Aufwand)
	\item Digitale Welt: Risiko und zeitlicher Aufwand sehr gering $\rightarrow$ Digitale Signaturen müssen sehr viel sicherer sein als Unterschriften
	\item Berechnung über asymmetrisches Schlüsselpaar
\end{itemize}

\subsubsection{Zufallszahlengenerator (PRNG)}
\begin{itemize}
	\item Zufallszahlen dürfen nicht prädizierbar sein
	\item Probleme bei herkömmlichen PRNGs: Gleichförmigkeit, Unabhängigkeit, Unkorreliertheit
	\item Sicherheit bei Kryptografischen PRNGs: Hohes Maß an Entropie nötig, so dass der Angreifer das Seed nicht prädizieren kann
\end{itemize}

\subsection{Anwendung der Verfahren}

\subsubsection{Basic Authentication}
\begin{itemize}
	\item Client schickt Geheimnis P an den Server
	\item Angriffsmöglichkeiten: Abhören und Replay-Attack, MitM, Angriff auf den Server, der die Geheimnisse hält
\end{itemize}

\subsubsection{Challenge-Response-Verfahren}
\begin{itemize}
	\item Client schickt Identifier C; Server schickt neugenerierte Zufallszahl R; Client verschlüsselt R mit Schlüssel K und schickt Ergebnis
	\item Angriffsmöglichkeiten: Abhören und Krypotanalyse, MitM, Angriff auf den Server, der die Schlüssel hält
\end{itemize}

\subsubsection{Challenge-Reponse mit PK-Verfahren}
\begin{itemize}
	\item Client schickt ID; Server schickt neugenerierte Zufallszahl R, Client signiert R und schickt Ergebnis
	\item Angriffsmöglichkeiten: Abhören und Kryptoanalyse, MitM
\end{itemize}

\subsubsection{Mutual Authentication (konzeptionell)}
\begin{itemize}
	\item Server schickt seinen Public Key; Client verschickt mit seinem Public Key verschlüsselten, generierten Session Key; Client schickt ID; Server schickt neugenrierte Zufallszahl R; Client signiert R und schickt Ergebnis
	\item Problem: Echtheitsprüfung der Public Keys
	\item Angriffsmöglichkeiten: Abhören und Kryptoanalyse, MitM mit falschen Public Keys
\end{itemize}

\subsubsection{Forward Secrecy}
\begin{itemize}
	\item Mutual Authentication zur Etablierung eines Session Keys; Aushandeln eines Sessions Keys per Diffie-Hellman; Austausch der Daten; Löschen der Keys
	\item Mutual Authentication und Session Key für Authentizität und Integrität
	\item Session Key für Vertraulichkeit
\end{itemize}

\subsubsection{Digitale Zertifikate}
\begin{itemize}
	\item Bindung bestimmte Eigenschaften an digitale Identitäten
	\item Können Daten zur Identifikation enthalten
	\item Überprüfbar bzgl. Integrität und Authentizität
	\item Zentrale Vergabestelle notwendig
	\item Praxisüblicher Standard: X.509
\end{itemize}

\subsubsection{Vergleich mit der Realwelt}

\begin{tabularx}{\columnwidth}{|X|X|l}
	\hline
	\textbf{Digitale Welt} & \textbf{Realwelt} \\
	\hline
	Digitale Signatur & Unterschrift \\
	Privater Schlüssel & Schriftbild \\
	Zertifikat & Personalausweis \\
	\hline
\end{tabularx}


\subsection{Zusammenfassung}
\begin{itemize}
	\item Hautziele von Verschlüsselung: Vertraulichkeit, Integrität, Authentizität und Verbindlichkeit (siehe Schutzziele)
	\item Grundlegende kryptografische Verfahren können zu komplexen Protokollen kombiniert werden
\end{itemize}



\section{Schlüsselmanagement und Zugangskontrolle}

\subsection{Schlüsselmanagement}

\subsubsection{Public Key Infrastruktur}
\begin{itemize}
	\item Verwaltung der öffentlichen Schlüsseln
	\item Bestandteile: Certification Authority (Verwalten und Ausstellen der Zertifikate), Registration Authority (Vorgelagerte Instanz zur Prüfung der Antragstellenden), Validation Authority (Prüfung von Zertifikaten, OCSP-Protokoll)
	\item Zur Signierung durch die CA: Signieren mit dem privaten Schlüssel
	\item Senden einer Nachricht: Signieren mit privatem Schlüssel; Senden der Nachricht mit Zertifikat im Anhang; Prüfen der Signatur mit öffentlichem Schlüssel aus dem Zertifikat
	\item Eine weltweite CA würde nicht skalieren, daher Root CAs mit Intermediate CAs als Zwischenstellen
	\item Widerruf von Zertifikaten über Certificate Revocation Lists (CRL)
\end{itemize}

\subsubsection{Kerberos}
\begin{itemize}
	\item Zentralisierung durch Key Distribution Center (KDC) mit SSO. Mögliche Nachteile: Single Point of Failure, Leistungsengpass
	\item Entwickelt im MIT (Athena-Projekt) und dort seit 1986 im Einsatz
	\item Verwendung von Tickets (sowohl benutzer- als auch serverbezogen)
	\item Zweistufiges Protokoll: Client kommuniziert nur selten mit dem KDC, Passwort wird nur selten genutzt
	\item Interrealm-Authentifizierung möglich
\end{itemize}


\subsection{Grundlagen der Zugangs- und Zugriffskontrolle}

\subsubsection{Digitale Identität am Beispiel des nPA}
\begin{itemize}
	\item Spezielle Funktionen: Fingerabdruck, eID (Internet Ausweis), Digitale Unterschrift
	\item Datensicherheit: Übertragung nur bei berechtigten Diensten (Zertifikat), Übermittlung nur nach PIN-Eingabe, verschlüsselte Übertragung, Karte muss auf dem Leser liegen
	\item Wettbewerb: Bereits eingeführte Verfahren (z.B. Near-Field-Communication), exisitierende Dienste, eventuell ungeeignet für Smartphones
\end{itemize}

\subsubsection{Möglichkeiten zur Authentifikation}
\begin{itemize}
	\item Wissensbasiert: PIN, Passwörter, Challenge-Response
	\item Biometrie: Fingerabdruck, Iris, Unterschrift, Tippverhalten
	\item Besitzbasiert: Ausweis, Magnetkarte, Zertifikat
	\item Mehr-Faktor-Authentifikation: Kombinationen daraus
\end{itemize}


\subsection{Zusammenfassung}
\begin{itemize}
	\item Zugangskontrolle ermöglicht Identifikation und Authentifikation
	\item Zugriffskontrolle als Basis für autorisierten Zugriff auf Ressourcen
	\item Aber: Der Faktor Mensch...
\end{itemize}


\section{Zugriffskontrolle}

\section{IDM1 - Provisioning}

\section{IDM2 - Föderatives IAM}

\section{Business Continuity}

\section{Secure Data Outsourcing}

\section{Secure Data Sharing}

\section{Anonymität}

\section{Online Social Networks}